% !TeX root = eserciziterzo.tex
% !BIB TS-program = biber
% !TeX encoding = UTF-8
% !TeX spellcheck = it_IT

\documentclass[openany,a4paper]{book}%
\usepackage{base}
%\geometry{top=2cm,bottom=2cm,left=2cm,right=2cm}
\usepackage[big]{layaureo}
\input{../Mod_base/grafica}
\input{../Mod_base/matematica}
\input{../Mod_base/tabelle}
\DeclareCaptionFormat{grafico}{\textbf{Grafico \thefigure}#2#3}
\DeclareCaptionFormat{esempio}{\textbf{Esempio \thefigure}#2#3}
\usepackage{imakeidx}
\makeindex[options=-s ../Mod_base/oldclaudio.sti]
\input{../Mod_base/pagina}
\input{../Mod_base/indice}
\input{../Mod_base/date}
\input{../Mod_base/loghi}
\input{../Mod_base/unita_misura}
\input{../Mod_base/utili}
\input{../Mod_base/stand_class}
\usepackage{qrcode}
\newcommand{\HRule}{\rule{\linewidth}{0.5mm}}
\usepackage{placeins} 
\makeatletter
\renewcommand\frontmatter{%
	\cleardoublepage
	\@mainmatterfalse
	\pagenumbering{arabic}}
\renewcommand\mainmatter{%
	\cleardoublepage
	\@mainmattertrue}
\makeatother

%%%%%%%%%%%%%%%%%%%%%%%%%%%%%%%%
%%%lunghezza arrotondamenti%%%%%
\newcommand{\lungarrotandamento}{4}
%%%%%%%%%%%%%%%%%%%%%%%%%%%%%%%%%%
\includeonly{%
terzo/NotazioneScientifica,
terzo/complessibase,
%terzo/complessiopposti,
%terzo/complessiconiugati,
terzo/complessisomma,
%terzo/complessidifferenza,
%terzo/complessiprodotto,
%terzo/complessidivisione,
terzo/complessimoduloargomento,
terzo/angoli,
terzo/goniometria,
terzo/FunzGonioCalc, 
terzo/EquazioniGoniometriche,
terzo/triangoli,
terzo/Soluzioni
}
\usepackage[grumpy,mark,markifdirty,raisemark=0.95\paperheight]{gitinfo2}
\usepackage[toc,page]{appendix}

\renewcommand{\appendixtocname}{Appendice}

\renewcommand{\appendixpagename}{Appendice}
\usepackage{tkz-berge}
\usepackage[italian]{varioref}
\usepackage{hyperxmp}
\usepackage[pdfpagelabels]{hyperref}
\usepackage[italian]{cleveref}
\input{../Mod_base/tcolorboxgest}
\title{Esercizi svolti terzo}
\author{Claudio Duchi}
\date{\datetime}
\hypersetup{%
	pdfencoding=auto,
	urlcolor={blue},
	pdftitle={Esercizi svolti},
	pdfsubject={terzo},
	pdfstartview={FitH},
	pdfpagemode={UseOutlines},
	pdflicenseurl={http://creativecommons.org/licenses/by-nc-nd/3.0/},
	pdflang={it},
	pdfmetalang={it},
	pdfkeywords={goniometria, trigoniometria, numeri complessi},
	pdfcopyright={Copyright (C) 2018, Claudio Duchi},
	pdfcontacturl={http://breviariomatematico.altervista.org},
	pdfcontactpostcode={},
	pdfcontactphone={},
	pdfcontactemail={claduc},
	pdfcontactcountry={Italy},
	pdfcontactcity={Perugia},
	pdfcontactaddress={},
	pdfcaptionwriter={Claudio Duchi},
	pdfauthortitle={},%
	pdfauthor={Claudio Duchi},
	linkcolor={blue},
	colorlinks=true,
	citecolor={red},
	breaklinks,
	bookmarksopen,
	verbose,
	baseurl={http://breviariomatematico.altervista.org}
}
\listfiles
\begin{document}
\frontmatter
		\begin{titlepage}
	\begin{center}
		\input{../Mod_base/Lgrande}\\[1cm]
		\textsc{\LARGE Claudio Duchi}\\[1.5cm]
		\HRule \\[0.4cm]
		{ \huge \bfseries ESERCIZI SVOLTI DI MATEMATICA}\\[0.4cm]
		{\LARGE \textsc{TERZO MANUTENZIONE}}
		\HRule \\[1.5cm]
		\vfill
		\begin{tikzpicture}
		\renewcommand*{\VertexBallColor}{green!50!black}
		\GraphInit[vstyle=Art]
		\grComplete[RA=5]{18}
		\end{tikzpicture}
	\end{center}
	{\centering
	Release:\gitReln\ (\gitAbbrevHash)\ Autore:\gitAuthorName\ 
	\gitCommitterDate \\
}
\end{titlepage}
\setcounter{page}{2} 
\input{../Mod_base/copyright}
\tableofcontents 
%\addcontentsline{toc}{chapter}{\listtablename}
%\listoftables

\addcontentsline{toc}{chapter}{\listfigurename}
\listoffigures
\renewcommand\lstlistlistingname{Esempi e contro esempi}
\addcontentsline{toc}{chapter}{\lstlistlistingname}
\addcontentsline{toc}{section}{Esempi}
\lstlistoflistings%{}
{
%	https://tex.stackexchange.com/questions/318486/number-freestyle-causes-an-overlay-in-the-list-of-tcolorboxes/318512#318512
	\makeatletter
	\renewcommand{\l@tcolorbox}{\@dottedtocline{1}{0pt}{3em}}
	\makeatother
\tcblistof[\section*]{thm}{Esempi}
\addcontentsline{toc}{section}{Contro esempi}
\tcblistof[\section*]{cthm}{Contro esempi}
}
\mainmatter%
 

\chapter{Notazione scientifica}
\label{cha:Notazionescietifica}
 
\section{Conversioni}


\begin{esempiot}{Da decimale a esponenziale}{decexp1}
	Convertire il numero \num{0.0035} in notazione scientifica
\end{esempiot}
La virgola va' spostata di tre posti verso destra per cui
\[\num{0.0035}=\num[scientific-notation=true]{0.0035}\]
\begin{esempiot}{Da decimale a esponenziale}{decexp2}
	Convertire il numero \num{7255.45} in notazione scientifica
\end{esempiot}
La virgola va' spostata di tre posti verso sinistra per cui
\[\num{7255.45}=\num[scientific-notation=true]{7255.45}\]
\tcbstartrecording

\begin{exercise}
	Convertire il numero \num{0.372} in notazione scientifica\index{Notazione!scientifica}
	\tcblower
	La virgola va' spostata di un posto verso destra per cui
	\[\num{0.372}=\num[scientific-notation=true]{0.372}\]
\end{exercise}
%\tcbstoprecording
%% \newpage
%\section{Soluzioni esercizi Notazione scientifica}
%\tcbinputrecords
\tcbstartrecording
\chapter{Numeri complessi}
\label{cha:numericomplessibase}
 \section{Il piano complesso}
% 24/02/2018 :: 17:21:26 ::  \tcbstartrecording
 \begin{exercise}
	Disegnare nel piano complesso\index{Piano!complesso} il numero $z=1+2\uimm$\index{Numero!complesso}
	\tcblower
	Disegnare nel piano complesso il numero $z=1+2\uimm$
	\begin{center}
		\includestandalone[width=.6\textwidth]{terzo/grafici/Piano_complesso_11}
		\captionof{figure}{Piano complesso}\label{fig:disegnopianocomplesso11}
	\end{center}
\end{exercise}
 \begin{exercise}
Disegnare nel piano complesso\index{Piano!complesso} il numero $z=2+3\uimm$\index{Numero!complesso}
\tcblower
Disegnare nel piano complesso il numero $z=2+3\uimm$
\begin{center}
\includestandalone[width=.6\textwidth]{terzo/grafici/Piano_complesso_01}
\captionof{figure}{Piano complesso}\label{fig:disegnopianocomplesso01}
\end{center}
 \end{exercise}
  \begin{exercise}
  	Disegnare nel piano complesso il numero $z=-4+2\uimm$
  	\tcblower
  	Disegnare nel piano complesso il numero $z=-4+2\uimm$
  	\begin{center}
  		\includestandalone[width=.6\textwidth]{terzo/grafici/Piano_complesso_02}
  		\captionof{figure}{Piano complesso}\label{fig:disegnopianocomplesso02}
  	\end{center}
  \end{exercise}

  \begin{exercise}
Disegnare nel piano complesso il numero $z=2-7\uimm$
	\tcblower
Disegnare nel piano complesso il numero $z=2-7\uimm$
	\begin{center}
		\includestandalone[width=.6\textwidth]{terzo/grafici/Piano_complesso_12}
		\captionof{figure}{Piano complesso}\label{fig:disegnopianocomplesso12}
	\end{center}
\end{exercise}
\begin{exercise}[no solution]
	Disegnare nel piano complesso il numero $z=1+2\uimm$
\end{exercise}
\begin{exercise}[no solution]
Disegnare nel piano complesso il numero $z=+2\uimm$
\end{exercise}
\begin{exercise}[no solution]
	Disegnare nel piano complesso il numero $z=1$
\end{exercise}
\begin{exercise}[no solution]
	Disegnare nel piano complesso il numero $z=-3+2\uimm$
\end{exercise}
\begin{exercise}[no solution]
	Disegnare nel piano complesso il numero $z=-\frac{1}{2}+4\uimm$
\end{exercise}
\begin{exercise}[no solution]
	Disegnare nel piano complesso il numero $z=-3-2\uimm$
\end{exercise}
\begin{exercise}[no solution]
	Disegnare nel piano complesso il numero $z=-3+\uimm$
\end{exercise}
%\begin{exercise}[no solution]
%	Disegnare nel piano complesso il numero $z=-4+2\uimm$
%\end{exercise}

% \tcbstoprecording
%% \newpage
% \section{Soluzioni il piano complesso}
% \tcbinputrecords

 \section{Opposto di un numero complesso}
%\tcbstartrecording
\begin{exercise}
	Trovare e disegnare l'opposto\index{Numero!complesso!opposto} di $z=1+2\uimm$
	\tcblower
	Trovare e disegnare l'opposto di $z=1+2\uimm$
	
	L'opposto è $-z=-1-2\uimm$
	\begin{center}
		\includestandalone[width=.6\textwidth]{terzo/grafici/Piano_complesso_03}
		\captionof{figure}{Opposto di un complesso}\label{fig:disegnopianocomplesso03}
	\end{center}
\end{exercise}
\begin{exercise}[no solution]
	Trovare e disegnare l'opposto di $z=2+4\uimm$
\end{exercise}
\begin{exercise}
	Trovare e disegnare l'opposto di $z=3-2\uimm$
	\tcblower
	Trovare e disegnare l'opposto di $z=3-2\uimm$
	
	L'opposto è $z=-3+2\uimm$
	\begin{center}
		\includestandalone[width=.6\textwidth]{terzo/grafici/Piano_complesso_04}
		\captionof{figure}{Opposto di un complesso}\label{fig:disegnopianocomplesso04}
	\end{center}
\end{exercise}
\begin{exercise}
	Trovare e disegnare l'opposto di $z=1-2\uimm$
	\tcblower
	Trovare e disegnare l'opposto di $z=1-2\uimm$
	
	L'opposto è $z=-1+2\uimm$
	\begin{center}
		\includestandalone[width=.6\textwidth]{terzo/grafici/Piano_complesso_13}
		\captionof{figure}{Opposto di un complesso}\label{fig:disegnopianocomplesso13}
	\end{center}
\end{exercise}

\begin{exercise}[no solution]
	Trovare e disegnare l'opposto di  $z=5-7\uimm$
\end{exercise}
\begin{exercise}
	Trovare e disegnare l'opposto di $z=-1-3\uimm$
	\tcblower
	Trovare e disegnare l'opposto di $z=-1-3\uimm$
	
	L'opposto è $z=1+3\uimm$
	\begin{center}
		\includestandalone[width=.6\textwidth]{terzo/grafici/Piano_complesso_14}
		\captionof{figure}{Opposto di un complesso}\label{fig:disegnopianocomplesso14}
	\end{center}
\end{exercise}
\begin{exercise}[no solution]
	Trovare e disegnare l'opposto di  $z=-1$
\end{exercise}
\begin{exercise}[no solution]
	Trovare e disegnare l'opposto di  $z=3\uimm$
\end{exercise}
\begin{exercise}[no solution]
	Trovare e disegnare l'opposto di  $z=1-4\uimm$
\end{exercise}
\begin{exercise}[no solution]
	Trovare e disegnare l'opposto di  $z=1+4\uimm$
\end{exercise}
%
\begin{exercise}[no solution]
	Trovare e disegnare l'opposto di $z=-2-4\uimm$
\end{exercise}
\begin{exercise}[no solution]
	Trovare e disegnare l'opposto di $z=-1+2\uimm$
\end{exercise}
\begin{exercise}[no solution]
	Trovare e disegnare l'opposto di  $z=-5+7\uimm$
\end{exercise}
\begin{exercise}[no solution]
	Trovare e disegnare l'opposto di  $z=1+3\uimm$
\end{exercise}
\begin{exercise}[no solution]
	Trovare e disegnare l'opposto di  $z=1$
\end{exercise}
\begin{exercise}[no solution]
	Trovare e disegnare l'opposto di  $z=-3\uimm$
\end{exercise}
\begin{exercise}[no solution]
	Trovare e disegnare l'opposto di  $z=-1+4\uimm$
\end{exercise}
\begin{exercise}[no solution]
	Trovare e disegnare l'opposto di  $z=-1-4\uimm$
\end{exercise}
 \section{Coniugato di un numero complesso}
%\tcbstartrecording
\begin{exercise}
	Trovare e disegnare il coniugato\index{Numero!complesso!coniugato} $\conj{z}$ di $z=1-2\uimm$
	\tcblower
	Trovare e disegnare il coniugato $\conj{z}$ di $z=1-2\uimm$
	
	$\conj{z}=1+2\uimm$
	\begin{center}
		\includestandalone[width=.6\textwidth]{terzo/grafici/Piano_complesso_05}
		\captionof{figure}{Complesso coniugato}\label{fig:disegnopianocomplesso05}
	\end{center}
\end{exercise}
\begin{exercise}[no solution]
	Trovare e disegnare il coniugato $\conj{z}$ di $z=3+4\uimm$
\end{exercise}
\begin{exercise}[no solution]
	Trovare e disegnare il coniugato $\conj{z}$ di $z=-2+2\uimm$
\end{exercise}
\begin{exercise}[no solution]
	Trovare e disegnare il coniugato $\conj{z}$ di $z=-8+\uimm$
\end{exercise}
\begin{exercise}[no solution]
	Trovare e disegnare il coniugato $\conj{z}$ di $z=-3$
\end{exercise}
\begin{exercise}[no solution]
	Trovare e disegnare il coniugato $\conj{z}$ di $z=-2\uimm$
\end{exercise}
\begin{exercise}[no solution]
	Trovare e disegnare il coniugato $\conj{z}$ di $z=-3-8\uimm$
\end{exercise}
\begin{exercise}[no solution]
	Trovare e disegnare il coniugato $\conj{z}$ di $z=-2+5\uimm$
\end{exercise}
\begin{exercise}
	Trovare e disegnare il coniugato $\conj{z}$ di  $z=-1-2\uimm$
	\tcblower
	Trovare e disegnare il coniugato $\conj{z}$ di  $z=-1-2\uimm$
	
	$\conj{z}=-1+2\uimm$
	\begin{center}
		\includestandalone[width=.6\textwidth]{terzo/grafici/Piano_complesso_15}
		\captionof{figure}{Complesso coniugato}\label{fig:disegnopianocomplesso15}
	\end{center}
\end{exercise}
\begin{exercise}
	Trovare e disegnare il coniugato $\conj{z}$ di $z=-3-2\uimm$
	\tcblower
	Trovare e disegnare il coniugato $\conj{z}$ di $z=-3-2\uimm$
	
	$\conj{z}=-3+2\uimm$
	\begin{center}
		\includestandalone[width=.6\textwidth]{terzo/grafici/Piano_complesso_06}
		\captionof{figure}{Complesso coniugato}\label{fig:disegnopianocomplesso06}
	\end{center}
\end{exercise}
\begin{exercise}[no solution]
	Trovare e disegnare il coniugato $\conj{z}$ di $-3+4\uimm$
\end{exercise}
\section{Modulo di un numero complesso}
\begin{exercise}
	Trovare il modulo di $z=3+4\uimm$
	\tcblower
	Trovare il modulo di $z=3+4\uimm$
	$\norm{z}=\sqrt{9+16}=\sqrt{25}=5$
\end{exercise}
\begin{exercise}[no solution]
	Trovare il modulo di $z=-2+2\uimm$
\end{exercise}
\begin{exercise}[no solution]
	Trovare il modulo di $z=-8+\uimm$
\end{exercise}
\begin{exercise}[no solution]
	Trovare il modulo di $z=-3$
\end{exercise}
\begin{exercise}[no solution]
	Trovare il modulo di $z=-2\uimm$
\end{exercise}
 \begin{exercise}
	Trovare il modulo di $z=-3-8\uimm$
	\tcblower
	Trovare il modulo di $z=-3-8\uimm$
	$\norm{z}=\sqrt{9+64}=\sqrt{73}$
\end{exercise}
\begin{exercise}[no solution]
	Trovare il modulo di $z=-2+5\uimm$
\end{exercise}
%\tcbstoprecording
%% \newpage
% \section{Soluzioni coniugato di un numero complesso}
% \tcbinputrecords
 \tcbstoprecording
 \newpage
 \section{Soluzioni il piano complesso}
 \tcbinputrecords

 \section{Opposto di un numero complesso}
 %\tcbstartrecording
 \begin{exercise}
Trovare e disegnare l'opposto\index{Numero!complesso!opposto} di $z=1+2\uimm$
\tcblower
Trovare e disegnare l'opposto di $z=1+2\uimm$
\begin{center}
\includestandalone[width=.6\textwidth]{terzo/grafici/Piano_complesso_03}
\captionof{figure}{Opposto di un complesso}\label{fig:disegnopianocomplesso03}
\end{center}
 \end{exercise}
 \begin{exercise}[no solution]
 Trovare e disegnare l'opposto di $z=2+4\uimm$
 \end{exercise}
  \begin{exercise}
  Trovare e disegnare l'opposto di $z=3-2\uimm$
  	\tcblower
  	 Trovare e disegnare l'opposto di $z=3-2\uimm$
  	\begin{center}
  		\includestandalone[width=.6\textwidth]{terzo/grafici/Piano_complesso_04}
  		\captionof{figure}{Opposto di un complesso}\label{fig:disegnopianocomplesso04}
  	\end{center}
  \end{exercise}
 \begin{exercise}[no solution]
 	Trovare e disegnare l'opposto di $z=1-2\uimm$
 \end{exercise}
%\tcbstoprecording
%% \newpage
% \section{Soluzioni opposto del complesso}
% \tcbinputrecords
 
 \section{Coniugato di un numero complesso}
 %\tcbstartrecording
 \begin{exercise}
Trovare e disegnare il coniugato\index{Numero!complesso!coniugato} $\conj{z}$ di $z=1-2\uimm$
\tcblower
Trovare e disegnare il coniugato $\conj{z}$ di $z=1-2\uimm$

$\conj{z}=1+2\uimm$
\begin{center}
\includestandalone[width=.6\textwidth]{terzo/grafici/Piano_complesso_05}
\captionof{figure}{Complesso coniugato}\label{fig:disegnopianocomplesso05}
\end{center}
 \end{exercise}
 \begin{exercise}[no solution]
Trovare e disegnare il coniugato $\conj{z}$ di $z=-1-2\uimm$
 \end{exercise}
  \begin{exercise}
  Trovare e disegnare il coniugato $\conj{z}$ di $z=-3-2\uimm$
  	\tcblower
  	  Trovare e disegnare il coniugato $\conj{z}$ di $z=-3-2\uimm$
  	  
  	  $\conj{z}=-3+2\uimm$
  	\begin{center}
  		\includestandalone[width=.6\textwidth]{terzo/grafici/Piano_complesso_06}
  		\captionof{figure}{Complesso coniugato}\label{fig:disegnopianocomplesso06}
  	\end{center}
  \end{exercise}
 \begin{exercise}[no solution]
 		  Trovare e disegnare il coniugato $\conj{z}$ di $-3+4\uimm$
 \end{exercise}
%\tcbstoprecording
%% \newpage
% \section{Soluzioni coniugato di un numero complesso}
% \tcbinputrecords
 
\tcbstartrecording
\chapter{Operazioni con i numeri complessi}
 \section{Somma di due numeri complessi}
 %\tcbstartrecording
  \begin{exercise}[no solution]
 	Dati i numeri $z_1=-8-3\uimm$ e $z_2=4+6\uimm$, trovare la loro somma e  disegnare il numero nel piano complesso.
 \end{exercise}
 \begin{exercise}[no solution]
	Dati i numeri $z_1=-3-2\uimm$ e $z_2=1-6\uimm$, trovare la loro somma e  disegnare il numero nel piano complesso.
\end{exercise}
 \begin{exercise}
Dati i numeri $z_1=1-2\uimm$ e $z_2=3+4\uimm$, trovare la loro somma\index{Numero!complesso!somma} e  disegnare il numero nel piano complesso.
\tcblower
Dati i numeri $z_1=1-2\uimm$ e $z_2=3+4\uimm$, trovare la loro somma e  disegnare il numero nel piano complesso.
\[z_1+z_2=1=1-2\uimm+3+4\uimm=4+2\uimm \]
\begin{center}
\includestandalone[width=.6\textwidth]{terzo/grafici/Piano_complesso_07}
\captionof{figure}{Somma di numeri complessi}\label{fig:disegnopianocomplesso07}
\end{center}
 \end{exercise}
 \begin{exercise}[no solution]
Dati i numeri $z_1=-3+2\uimm$ e $z_2=1+6\uimm$, trovare la loro somma e  disegnare il numero nel piano complesso.
 \end{exercise}
 \begin{exercise}
 	Dati i numeri $z_1=-2+2\uimm$ e $z_2=-2-2\uimm$, trovare la loro somma e  disegnare il numero nel piano complesso.
 	\tcblower
 	Dati i numeri $z_1=-2+2\uimm$ e $z_2=-2-2\uimm$, trovare la loro somma e  disegnare il numero nel piano complesso.
 \[z_1+z_2=1=-2+2\uimm-2-2\uimm=-4\]
 	\begin{center}
 		\includestandalone[width=.6\textwidth]{terzo/grafici/Piano_complesso_08}
 		\captionof{figure}{Somma di numeri complessi}\label{fig:disegnopianocomplesso08}
 	\end{center}
 \end{exercise}
 \begin{exercise}
	Dati i numeri $z_1=-3+2\uimm$ e $z_2=1+6\uimm$, trovare la loro somma\index{Numero!complesso!somma} e  disegnare il numero nel piano complesso.
	\tcblower
	Dati i numeri $z_1=-3+2\uimm$ e $z_2=1+6\uimm$, trovare la loro somma e  disegnare il numero nel piano complesso.
	\[z_1+z_2=1=-3+2\uimm+1+6\uimm=-2+8\uimm \]
	\begin{center}
		\includestandalone[width=.6\textwidth]{terzo/grafici/Piano_complesso_16}
		\captionof{figure}{Somma di numeri complessi}\label{fig:disegnopianocomplesso16}
	\end{center}
\end{exercise}
 \begin{exercise}[no solution]
	Dati i numeri $z_1=-2+3\uimm$ e $z_2=6+1\uimm$, trovare la loro somma e  disegnare il numero nel piano complesso.
\end{exercise}
 \begin{exercise}[no solution]
	Dati i numeri $z_1=-5+3\uimm$ e $z_2=-4+\uimm$, trovare la loro somma e  disegnare il numero nel piano complesso.
\end{exercise}
 \begin{exercise}[no solution]
	Dati i numeri $z_1=+2\uimm$ e $z_2=-1+5\uimm$, trovare la loro somma e  disegnare il numero nel piano complesso.
\end{exercise}
 \begin{exercise}[no solution]
	Dati i numeri $z_1=-4+\uimm$ e $z_2=4+\uimm$, trovare la loro somma e  disegnare il numero nel piano complesso.
\end{exercise}
 \begin{exercise}[no solution]
	Dati i numeri $z_1=-3+2\uimm$ e $z_2=-3-2\uimm$, trovare la loro somma e  disegnare il numero nel piano complesso.
\end{exercise}
%\tcbstoprecording
%% \newpage
% \section{Soluzioni somma di due numeri complessi}
% \tcbinputrecords
  \section{Differenza di due numeri complessi}
 % \tcbstartrecording
 \begin{exercise}
 	Dati i numeri $z_1=2-3\uimm$ e $z_2=1-2\uimm$, trovare la loro differenza\index{Numero!complesso!differenza} e  disegnare il risultato  nel piano complesso.
 	\tcblower
 	Dati i numeri $z_1=-3-2\uimm$ e $z_2=-4-1\uimm$, trovare la loro differenza e  disegnare il risultato  nel piano complesso.
 	\[z_1-z_2=z_1=-3-2\uimm-(-4-1\uimm) =1+3\uimm \]
 	\begin{center}
 		\includestandalone[width=.6\textwidth]{terzo/grafici/Piano_complesso_09}
 		\captionof{figure}{Differenza di numeri complessi}\label{fig:disegnopianocomplesso09}
 	\end{center}
 \end{exercise}
 \begin{exercise}[no solution]
 	Dati i numeri $z_1=-3+2\uimm$ e $z_2=1+6\uimm$, trovare la loro differenza e  disegnare il risultato  nel piano complesso.
 \end{exercise}
 \begin{exercise}
 	Dati i numeri $z_1=3+2\uimm$ e $z_2=-4+\uimm$, trovare la loro differenza e  disegnare il risultato  nel piano complesso.
 	\tcblower
 	Dati i numeri $z_1=-2+2\uimm$ e $z_2=-2-2\uimm$, trovare la loro differenza e  disegnare il risultato  nel piano complesso.
 	\begin{center}
 		\includestandalone[width=.6\textwidth]{terzo/grafici/Piano_complesso_10}
 		\captionof{figure}{Differenza di numeri complessi}\label{fig:disegnopianocomplesso08a}
 	\end{center}
 \end{exercise}
 \begin{exercise}[no solution]
 	Dati i numeri $z_1=-3+2\uimm$ e $z_2=1+6\uimm$, trovare la loro differenza e  disegnare il risultato  nel piano complesso.
 \end{exercise}
 \section{Prodotto di numeri complessi}
%\tcbstartrecording
\begin{exercise}
	Eseguire\index{Numero!complesso!prodotto}la seguente moltiplicazione $(2-5\uimm)(3-4\uimm)$ fra numeri complessi.
	\tcblower
	Eseguire la seguente moltiplicazione $(2-5\uimm)(3-4\uimm)$ fra numeri complessi.
	\begin{align*}
	(2-5\uimm)(3-4\uimm)=&6-8\uimm-15\uimm-20\\
	=&-14-23\uimm
	\end{align*}
\end{exercise}
\begin{exercise}
	Eseguire la seguente moltiplicazione $(1-7\uimm)(1+5\uimm)$ fra numeri complessi.
	\tcblower
	Eseguire la seguente moltiplicazione $(1-7\uimm)(1+5\uimm)$ fra numeri complessi.
	\begin{align*}
	(1-7\uimm)(1+5\uimm)=&1+5\uimm-7\uimm-20\\
	=&36-2\uimm
	\end{align*}
\end{exercise}
\begin{exercise}
	Eseguire la seguente moltiplicazione $(1-5\uimm)(1+5\uimm) $ fra numeri complessi.
	\tcblower
	Eseguire la seguente moltiplicazione $(1-5\uimm)(1+5\uimm)$ fra numeri complessi.
	\begin{align*}
	(1-5\uimm)(1+5\uimm)=&1+5\uimm-5\uimm+25\\
	=&26
	\end{align*}
\end{exercise}
\begin{exercise}
	Eseguire la seguente moltiplicazione $(1-3\uimm)(6-2\uimm)(3+2\uimm)$ fra numeri complessi.
	\tcblower
	Eseguire la seguente moltiplicazione $(1-3\uimm)(6-2\uimm)(3+2\uimm)$ fra numeri complessi.
	\begin{align*}
	(1-3\uimm)(6-2\uimm)(3+2\uimm)=&(1-2\uimm-18\uimm-6)(3+2\uimm)\\
	=&-20\uimm(3+2\uimm)\\
	=&40-60\uimm\\
	\end{align*}
\end{exercise}

%\tcbstoprecording
%% \newpage
% \section{Soluzioni esercizi numeri complessi}
% \tcbinputrecords
 \section{Divisione di  numeri complessi}
%\tcbstartrecording
\begin{exercise}
	Eseguire\index{Numero!complesso!divisione}la seguente divisione $\dfrac{3-2\uimm}{1-5\uimm}$ fra numeri complessi.
	\tcblower
	Eseguire la seguente divisione  $\dfrac{3-2\uimm}{1-5\uimm}$ fra numeri complessi.
	\begin{align*}
	\dfrac{3-2\uimm}{1-5\uimm}=&\dfrac{3-2\uimm}{1-5\uimm}\dfrac{1+5\uimm}{1+5\uimm}\\
	=&\dfrac{3+15\uimm-2\uimm+10}{1+25}\\
	=&\dfrac{13+13\uimm}{26}\\
	=&\dfrac{13}{26}+\dfrac{13}{26}\uimm\\
	\end{align*}
\end{exercise}
\begin{exercise}
	Eseguire\index{Numero!complesso!divisione}la seguente divisione $\dfrac{5-2\uimm}{3+4\uimm}$ fra numeri complessi.
	\tcblower
	Eseguire la seguente divisione  $\dfrac{5-2\uimm}{3+4\uimm}$ fra numeri complessi.
	\begin{align*}
	\dfrac{5-2\uimm}{3+4\uimm}=&\dfrac{5-2\uimm}{3+4\uimm}\dfrac{3-4\uimm}{3-4\uimm}\\
	=&\dfrac{15-20\uimm-6\uimm-8}{9+16}\\
	=&\dfrac{7-26\uimm}{9+16}\\
	=&\dfrac{7}{25}-\dfrac{26}{25}\uimm\\
	\end{align*}
\end{exercise}
\begin{exercise}
	Eseguire le seguenti operazioni  $\dfrac{1-\uimm}{3\uimm}\dfrac{5-6\uimm}{1-\uimm}$ fra numeri complessi.
	\tcblower
	Eseguire le seguenti operazioni  $\dfrac{1-\uimm}{3\uimm}\dfrac{5-6\uimm}{1-\uimm}$ fra numeri complessi.
	\begin{align*}
	\dfrac{1-\uimm}{3\uimm}\dfrac{5-6\uimm}{1-\uimm}=&\dfrac{5-6\uimm-5\uimm-6}{3+3\uimm}\\
	=&\dfrac{-1-11\uimm}{3+3\uimm}\\
	=&\dfrac{-1-11\uimm}{3+3\uimm}\dfrac{3-3\uimm}{3-3\uimm}\\
	=&\dfrac{-3+3\uimm-33\uimm-33}{9+9}\\
	=&\dfrac{-36-30\uimm}{18}\\
	=&-2-\dfrac{5}{3}\uimm
	\end{align*}
\end{exercise}

\tcbstoprecording
 \newpage
 \section{Soluzioni esercizi numeri complessi}
 \tcbinputrecords
 \section{Differenza di due numeri complessi}
% \tcbstartrecording
 \begin{exercise}
Dati i numeri $z_1=2-3\uimm$ e $z_2=1-2\uimm$, trovare la loro differenza\index{Numero!complesso!differenza} e  disegnare il risultato  nel piano complesso.
\tcblower
Dati i numeri $z_1=-3-2\uimm$ e $z_2=-4-1\uimm$, trovare la loro differenza e  disegnare il risultato  nel piano complesso.
\[z_1-z_2=z_1=-3-2\uimm-(-4-1\uimm) =1+3\uimm \]
\begin{center}
\includestandalone[width=.6\textwidth]{terzo/grafici/Piano_complesso_09}
\captionof{figure}{Differenza di numeri complessi}\label{fig:disegnopianocomplesso09}
\end{center}
 \end{exercise}
 \begin{exercise}[no solution]
Dati i numeri $z_1=-3+2\uimm$ e $z_2=1+6\uimm$, trovare la loro differenza e  disegnare il risultato  nel piano complesso.
 \end{exercise}
 \begin{exercise}
 	Dati i numeri $z_1=3+2\uimm$ e $z_2=-4+\uimm$, trovare la loro differenza e  disegnare il risultato  nel piano complesso.
 	\tcblower
 	Dati i numeri $z_1=-2+2\uimm$ e $z_2=-2-2\uimm$, trovare la loro differenza e  disegnare il risultato  nel piano complesso.
 	\begin{center}
 		\includestandalone[width=.6\textwidth]{terzo/grafici/Piano_complesso_10}
 		\captionof{figure}{Differenza di numeri complessi}\label{fig:disegnopianocomplesso08a}
 	\end{center}
 \end{exercise}
 \begin{exercise}[no solution]
 	Dati i numeri $z_1=-3+2\uimm$ e $z_2=1+6\uimm$, trovare la loro differenza e  disegnare il risultato  nel piano complesso.
 \end{exercise}

%\tcbstoprecording
%% \newpage
% \section{Soluzioni differenza di due numeri complessi}
% \tcbinputrecords
 
 \section{Prodotto di  numeri complessi}
 %\tcbstartrecording
 \begin{exercise}
Eseguire\index{Numero!complesso!prodotto}la seguente moltiplicazione\[(2-5\uimm)(3-4\uimm) \] fra numeri complessi.
\tcblower
Eseguire la seguente moltiplicazione $(2-5\uimm)(3-4\uimm)$ fra numeri complessi.
\begin{align*}
(2-5\uimm)(3-4\uimm)=&6-8\uimm-15\uimm-20\\
=&-14-23\uimm
\end{align*}
 \end{exercise}
 \begin{exercise}
	Eseguire la seguente moltiplicazione\[(1-7\uimm)(1+5\uimm) \]fra numeri complessi.
	\tcblower
	Eseguire la seguente moltiplicazione $(1-7\uimm)(1+5\uimm)$ fra numeri complessi.
	\begin{align*}
	(1-7\uimm)(1+5\uimm)=&1+5\uimm-7\uimm-20\\
	=&36-2\uimm
	\end{align*}
\end{exercise}
 \begin{exercise}
	Eseguire la seguente moltiplicazione\[(1-5\uimm)(1+5\uimm) \] fra numeri complessi.
	\tcblower
	Eseguire la seguente moltiplicazione $(1-5\uimm)(1+5\uimm)$ fra numeri complessi.
	\begin{align*}
	(1-5\uimm)(1+5\uimm)=&1+5\uimm-5\uimm+25\\
	=&26
	\end{align*}
\end{exercise}
\begin{exercise}
	Eseguire la seguente moltiplicazione\[(1-3\uimm)(6-2\uimm)(3+2\uimm)\] fra numeri complessi.
	\tcblower
	Eseguire la seguente moltiplicazione $(1-3\uimm)(6-2\uimm)(3+2\uimm)$ fra numeri complessi.
	\begin{align*}
	(1-3\uimm)(6-2\uimm)(3+2\uimm)=&(1-2\uimm-18\uimm-6)(3+2\uimm)\\
	=&-20\uimm(3+2\uimm)\\
	=&40-60\uimm\\
	\end{align*}
\end{exercise}

%\tcbstoprecording
%% \newpage
% \section{Soluzioni esercizi numeri complessi}
% \tcbinputrecords
 
 \section{Divisione di  numeri complessi}
 %\tcbstartrecording
 \begin{exercise}
Eseguire\index{Numero!complesso!divisione}la seguente divisione \[\dfrac{3-2\uimm}{1-5\uimm} \] fra numeri complessi.
\tcblower
Eseguire la seguente divisione  $\dfrac{3-2\uimm}{1-5\uimm}$ fra numeri complessi.
\begin{align*}
\dfrac{3-2\uimm}{1-5\uimm}=&\dfrac{3-2\uimm}{1-5\uimm}\dfrac{1+5\uimm}{1+5\uimm}\\
=&\dfrac{3+15\uimm-2\uimm+10}{1+25}\\
=&\dfrac{13+13\uimm}{26}\\
=&\dfrac{13}{26}+\dfrac{13}{26}\uimm\\
\end{align*}
 \end{exercise}
 \begin{exercise}
	Eseguire\index{Numero!complesso!divisione}la seguente divisione \[\dfrac{5-2\uimm}{3+4\uimm} \] fra numeri complessi.
	\tcblower
	Eseguire la seguente divisione  $\dfrac{5-2\uimm}{3+4\uimm}$ fra numeri complessi.
	\begin{align*}
	\dfrac{5-2\uimm}{3+4\uimm}=&\dfrac{5-2\uimm}{3+4\uimm}\dfrac{3-4\uimm}{3-4\uimm}\\
	=&\dfrac{15-20\uimm-6\uimm-8}{9+16}\\
	=&\dfrac{7-26\uimm}{9+16}\\
	=&\dfrac{7}{25}-\dfrac{26}{25}\uimm\\
	\end{align*}
\end{exercise}
 \begin{exercise}
	Eseguire le seguenti operazioni  \[\dfrac{1-\uimm}{3\uimm}\dfrac{5-6\uimm}{1-\uimm}\] fra numeri complessi.
	\tcblower
	Eseguire le seguenti operazioni  $\dfrac{1-\uimm}{3\uimm}\dfrac{5-6\uimm}{1-\uimm}$ fra numeri complessi.
	\begin{align*}
	\dfrac{1-\uimm}{3\uimm}\dfrac{5-6\uimm}{1-\uimm}=&\dfrac{5-6\uimm-5\uimm-6}{3+3\uimm}\\
	=&\dfrac{-1-11\uimm}{3+3\uimm}\\
	=&\dfrac{-1-11\uimm}{3+3\uimm}\dfrac{3-3\uimm}{3-3\uimm}\\
	=&\dfrac{-3+3\uimm-33\uimm-33}{9+9}\\
	=&\dfrac{-36-30\uimm}{18}\\
	=&-2-\dfrac{5}{3}\uimm
	\end{align*}
\end{exercise}

\tcbstoprecording
% \newpage
 \section{Soluzioni esercizi numeri complessi}
 \tcbinputrecords
 
\section{Modulo e argomento}
\begin{esempiot}{Trovare modulo ed argomento di un numero complesso}{numcomarg1}\index{Numero!complesso!modulo}\index{Numero!complesso!argomento}
	Dato  il numero complesso \[z=2+3\uimm\] determinarne modulo $\abs{z}$ ed l'argomento $\phi$ con $\ang{0}\leq\phi<\ang{360}$
\end{esempiot}
Rappresento il numero complesso nel piano di Gauss
\begin{center}
	\includestandalone[width=.6\textwidth]{terzo/grafici/argomentocomplesso1}
	\captionof{figure}{Modulo ed argomento uno}\label{fig:moduloargomentouno}
\end{center}
Formule coinvolte:
\begin{align*}
\abs{z}=&\sqrt{a^2+b^2}\\
\phi=&\arctan\left(\dfrac{b}{a}\right)
\end{align*}
Quindi per il modulo
\begin{align*}
z=&2+3\uimm\\
\abs{z}=&\sqrt{2^2+3^2}\\
=&\sqrt{4+9}\\
=&\sqrt{13}\\
\simeq&\num[round-precision=\lungarrotandamento,round-mode=places]{3.6055551275}
\end{align*}
Utilizzando la calcolatrice
 \begin{center}
	\begin{tabular}{ll}
		\tasto{2}\tastoquadrato\tastopiu\tasto{3}\tastoquadrato\tastouguale&13\\
	\tastoradicequadrata\tastoans\tastouguale&$\simeq\num[round-precision=\lungarrotandamento,round-mode=places]{3.6055551275}$
		\end{tabular}
\end{center}
Per l'argomento

Verifico la calcolatrice \testgradi
\begin{align*}
z=&2+3\uimm\\
\phi=&\arctan\left(\dfrac{3}{2}\right)\\
\simeq&\ang[round-precision=\lungarrotandamento,round-mode=places]{56.30993247}
\end{align*}
Utilizzando la calcolatrice
\begin{center}
	\begin{tabular}{ll}
		\tasto{3}\tastodiv\tasto{2}\tastouguale&1.5\\
\tastoitan\tastoans\tastouguale&$\simeq\ang[round-precision=\lungarrotandamento,round-mode=places]{56.30993247}$
	\end{tabular}
\end{center}

\begin{esempiot}{Trovare modulo ed l'argomento di un numero complesso}{}
	Dato  il numero complesso \[z=-2+3\uimm\] determinarne modulo $\abs{z}$ ed argomento $\phi$ con $\ang{0}\leq\phi<\ang{360}$
\end{esempiot}
Rappresento il numero complesso nel piano di Gauss
\begin{center}
	\includestandalone[width=0.6\textwidth]{terzo/grafici/argomentocomplesso2}
	\captionof{figure}{Modulo ed argomento due}\label{fig:moduloargomentodue}
\end{center}
Formule coinvolte:
\begin{align*}
\abs{z}=&\sqrt{a^2+b^2}\\
\phi=&\arctan\left(\dfrac{b}{a}\right)
\end{align*}
Quindi per il modulo
\begin{align*}
z=&-2+3\uimm\\
\abs{z}=&\sqrt{(-2)^2+3^2}\\
=&\sqrt{4+9}\\
=&\sqrt{13}\\
\simeq&\num[round-precision=\lungarrotandamento,round-mode=places]{3.6055551275}
\end{align*}
Utilizzando la calcolatrice
\begin{center}
	\begin{tabular}{ll}
		\tastoparentesisin\tasto{-2}\tastoparentesides\tastoquadrato\tastopiu\tasto{3}\tastoquadrato\tastouguale&13\\
		\tastoradicequadrata\tastoans\tastouguale&$\simeq\num[round-precision=\lungarrotandamento,round-mode=places]{3.6055551275}$
	\end{tabular}
\end{center}
Per l'argomento

Verifico la calcolatrice \testgradi
\begin{align*}
z=&-2+3\uimm\\
\phi=&\arctan\left(-\dfrac{3}{2}\right)\\
\simeq&\ang[round-precision=\lungarrotandamento,round-mode=places]{-56.30993247}
\intertext{Correggo}
\simeq&\ang{180}\ang[round-precision=\lungarrotandamento,round-mode=places]{-56.30993247}\\
\simeq&\ang[round-precision=\lungarrotandamento,round-mode=places]{123.6900675}
\end{align*}
Utilizzando la calcolatrice
\begin{center}
	\begin{tabular}{ll}
		\tasto{3}\tastodiv\tastoparentesisin\tasto{-2}\tastoparentesides\tastouguale&\num{-1.5}\\
		\tastoitan\tastoans\tastouguale&$\simeq\ang[round-precision=\lungarrotandamento,round-mode=places]{-56.30993247}$\\
		\tasto{180}\tastopiu\tastoans\tastouguale&$\simeq\ang[round-precision=\lungarrotandamento,round-mode=places]{123.6900675}$
	\end{tabular}
\end{center}
\begin{esempiot}{Trovare modulo ed l'argomento di un numero complesso}{}
	Dato  il numero complesso \[z=-2-3\uimm\] determinarne modulo $\abs{z}$ ed argomento $\phi$ con $\ang{0}\leq\phi<\ang{360}$
\end{esempiot}
Rappresento il numero complesso nel piano di Gauss
\begin{center}
	\includestandalone[width=.6\textwidth]{terzo/grafici/argomentocomplesso3}
	\captionof{figure}{Modulo ed argomento tre}\label{fig:moduloargomentotre}
\end{center}
Formule coinvolte:
\begin{align*}
\abs{z}=&\sqrt{a^2+b^2}\\
\phi=&\arctan\left(\dfrac{b}{a}\right)
\end{align*}
Quindi per il modulo
\begin{align*}
z=&-2-3\uimm\\
\abs{z}=&\sqrt{(-2)^2+(-3)^2}\\
=&\sqrt{4+9}\\
=&\sqrt{13}\\
\simeq&\num[round-precision=\lungarrotandamento,round-mode=places]{3.6055551275}
\end{align*}
Utilizzando la calcolatrice
\begin{center}
	\begin{tabular}{ll}
		\tastoparentesisin\tasto{-2}\tastoparentesides\tastoquadrato\tastopiu\tastoparentesisin\tasto{-3}\tastoparentesides\tastoquadrato\tastouguale&13\\
		\tastoradicequadrata\tastoans\tastouguale&$\simeq\num[round-precision=\lungarrotandamento,round-mode=places]{3.6055551275}$
	\end{tabular}
\end{center}
Per l'argomento

Verifico la calcolatrice \testgradi
\begin{align*}
z=&-2-3\uimm\\
\phi=&\arctan\left(\dfrac{-3}{-2}\right)\\
\simeq&\ang[round-precision=\lungarrotandamento,round-mode=places]{56.30993247}
\intertext{correggo}
\simeq&\ang{180}+\ang[round-precision=\lungarrotandamento,round-mode=places]{56.30993247}\\
\simeq&\ang[round-precision=\lungarrotandamento,round-mode=places]{236.30993247}
\end{align*}
Utilizzando la calcolatrice
\begin{center}
	\begin{tabular}{ll}
		\tasto{-3}\tastodiv\tasto{-2}\tastouguale&1.5\\
		\tastoitan\tastoans\tastouguale&$\simeq\ang[round-precision=\lungarrotandamento,round-mode=places]{56.30993247}$\\
		\tasto{180}\tastopiu\tastoans\tastouguale&$\simeq\ang[round-precision=\lungarrotandamento,round-mode=places]{236.3099325}$\\
	\end{tabular}
\end{center}

\begin{esempiot}{Trovare modulo ed l'argomento di un numero complesso}{}
	Dato  il numero complesso \[z=2-3\uimm\] determinarne modulo $\abs{z}$ ed argomento $\phi$ con $\ang{0}\leq\phi<\ang{360}$
\end{esempiot}
Rappresento il numero complesso nel piano di Gauss
\begin{center}
	\includestandalone[width=.6\textwidth]{terzo/grafici/argomentocomplesso4}
	\captionof{figure}{Modulo ed argomento quattro}\label{fig:moduloargomentoquattro}
\end{center}
Formule coinvolte:
\begin{align*}
\abs{z}=&\sqrt{a^2+b^2}\\
\phi=&\arctan\left(\dfrac{b}{a}\right)
\end{align*}
Quindi per il modulo
\begin{align*}
z=&2-3\uimm\\
\abs{z}=&\sqrt{2^2+(-3)^2}\\
=&\sqrt{4+9}\\
=&\sqrt{13}\\
\simeq&\num[round-precision=\lungarrotandamento,round-mode=places]{3.6055551275}
\end{align*}
Utilizzando la calcolatrice
\begin{center}
	\begin{tabular}{ll}
		\tasto{2}\tastoquadrato\tastopiu\tastoparentesisin\tasto{-3}\tastoparentesides\tastoquadrato\tastouguale&13\\
		\tastoradicequadrata\tastoans\tastouguale&$\simeq\num[round-precision=\lungarrotandamento,round-mode=places]{3.6055551275}$
	\end{tabular}
\end{center}
Per l'argomento

Verifico la calcolatrice \testgradi
\begin{align*}
z=&2-3\uimm\\
\phi=&\arctan\left(\dfrac{-3}{2}\right)\\
\simeq&\ang[round-precision=\lungarrotandamento,round-mode=places]{-56.30993247}
\intertext{correggo}
\simeq&\ang{360}\ang[round-precision=\lungarrotandamento,round-mode=places]{-56.30993247}\\
\simeq&\ang[round-precision=\lungarrotandamento,round-mode=places]{303.6900675}
\end{align*}
Utilizzando la calcolatrice
\begin{center}
	\begin{tabular}{ll}
		\tasto{-3}\tastodiv\tasto{2}\tastouguale&-1.5\\
		\tastoitan\tastoans\tastouguale&$\simeq\ang[round-precision=\lungarrotandamento,round-mode=places]{-56.30993247}$\\
			\tasto{360}\tastopiu\tastoans\tastouguale&$\simeq\ang[round-precision=\lungarrotandamento,round-mode=places]{303.6900675}$\\
		\end{tabular}
\end{center}
\chapter{Angoli}
\label{cha:angolibase}

\section{Conversioni radianti gradi}
\begin{esempiot}{Conversioni radianti gradi}{}
Convertire $\alpha=\ang{45;58;25}$ in radianti\index{Radianti}
\end{esempiot}
Prima convertiamo in gradi decimali\index{Radianti}\index{Grado!Sessagesimale}
\[\alpha=\ang{45}+\left(\dfrac{58}{60}\right)^{\si{\degree} }+\left(\dfrac{25}{3600}\right)^{\si{\degree} }=\ang{45}+\left(\dfrac{58\cdot60+25}{3600}\right)^{\si{\degree} } =\ang{45}+\left(\dfrac{3505}{3600}\right)^{\si{\degree} }\approx\ang[round-precision=\lungarrotandamento,round-mode=places]{45.97361111}\]
\[\rho=\dfrac{\pi}{180}\alpha\approx\dfrac{\pi}{180}\cdot\ang[round-precision=\lungarrotandamento,round-mode=places]{45.97361111}\approx\SI[round-precision=\lungarrotandamento,round-mode=places]{0.802390882}{\radian}\]
\stampapuntini
\begin{esempiot}{Conversioni radianti gradi}{}
	Convertire $\alpha=\ang{70;48;25}$ in radianti
\end{esempiot}
Prima convertiamo in gradi decimali \index{Radiante}\index{Grado!Sessagesimale}\index{Grado!Sessadecimale}
\[\alpha=\puntini{\ang{70}}+\left(\dfrac{48}{60}\right)^{\si{\degree} }+\left(\dfrac{25}{3600}\right)^{\si{\degree} }=\ang{70}+\left(\dfrac{\puntini{48\cdot60}+25}{3600}\right)^{\si{\degree} } =\ang{70}+\left(\dfrac{\puntini{2905}}{3600}\right)^{\si{\degree} }\approx\ang[round-precision=\lungarrotandamento,round-mode=places]{70.806944}\]
\[\rho=\dfrac{\puntini{\pi}}{180}\alpha\approx\dfrac{\puntini{\pi}}{180}\cdot\ang[round-precision=\lungarrotandamento,round-mode=places]{70.806944}\approx\puntini{\SI[round-precision=\lungarrotandamento,round-mode=places]{1.2358143}{\radian}}\]
\nonstampapuntini
\begin{esempiot}{Conversioni radianti gradi}{}
	Convertire\SI[round-precision=\lungarrotandamento,round-mode=places]{2.856}{\radian} in gradi sessagesimali
\end{esempiot}
\[\alpha=\dfrac{180}{\pi}\cdot\rho=\dfrac{180}{\pi}\cdot\SI[round-precision=\lungarrotandamento,round-mode=places]{2.856}{\radian}\approx\ang[round-precision=\lungarrotandamento,round-mode=places]{163.6367463}\]
Iniziamo con 
$\alpha=\ang{163}+\ang[round-precision=\lungarrotandamento,round-mode=places]{0.6367463}$
\begin{align*}
\alpha^{\si{\degree} }&=\ang{163}\\ 
\alpha^{\si{\arcminute}}&=\ang[round-precision=\lungarrotandamento,round-mode=places]{0.6367463;;}\cdot 60=\ang[round-precision=\lungarrotandamento,round-mode=places]{;38.20477736;}=\ang{;38;}\\
\alpha^{\si{\arcsecond}}&=\ang[round-precision=\lungarrotandamento,round-mode=places]{;0.204777358;}\cdot 60\approx\ang{;;12}\\
\end{align*}
abbiamo quindi
\[\alpha=\ang[round-precision=\lungarrotandamento,round-mode=places]{163.6367463}=\ang{163;38;12}\]
\stampapuntini
\begin{esempiot}{Conversioni radianti gradi}{}
	Convertire \SI[round-precision=\lungarrotandamento,round-mode=places]{0.823310}{\radian} in gradi sessagesimali
\end{esempiot}
\[\alpha=\dfrac{180}{\pi}\cdot\rho=\dfrac{180}{\pi}\cdot\SI[round-precision=\lungarrotandamento,round-mode=places]{0.823310}{\radian}\approx\puntini{\ang[round-precision=\lungarrotandamento,round-mode=places]{47.17218823}}\]
Iniziamo con 
$\alpha=\ang{47}+\ang[round-precision=\lungarrotandamento,round-mode=places]{0.17218823}$
\begin{align*}
\alpha^{\si{\degree}}&=\puntini{\ang{47}}\\ 
\alpha^{\si{\arcminute}}&=\ang[round-precision=\lungarrotandamento,round-mode=places]{0.17218823;;}\cdot 60=\puntini{\ang[round-precision=\lungarrotandamento,round-mode=places]{;10.33129385;}}=\puntini{\ang{;10;}}\\
\alpha^{\si{\arcsecond}}&=\puntini{\ang[round-precision=\lungarrotandamento,round-mode=places]{;0.331293854;}\cdot 60\approx\ang{;;19}}\\
\end{align*}
abbiamo quindi
\[\alpha=\ang[round-precision=\lungarrotandamento,round-mode=places]{47.17218823}=\ang{47;10;19}\]
\section{Da grado sessagesimale a sessa-decimale}
 \tcbstartrecording
 \begin{exercise}
Trasformare $\alpha=\ang{52;38;28}$ in forma sessa-decimale\index{Grado!Sessadecimale}\index{Grado!Sessagesimale}
\tcblower
\begin{align*}
\alpha=&\ang{52}+\left(\dfrac{38}{60}\right)^{\si{\degree} }+\left(\dfrac{28}{3600}\right)^{\si{\degree} }\\
=&\ang{52}+\left(\dfrac{38\cdot60+28}{3600}\right)^{\si{\degree} }\\
=&\ang{52}+\left(\dfrac{2308}{3600}\right)^{\si{\degree} }\approx\ang[round-precision=\lungarrotandamento,round-mode=places]{52,641111}
\end{align*}
\end{exercise}
 \begin{exercise}
 	Trasformare  $\alpha=\ang{75;55;35}$ in forma sessa-decimale\index{Grado!Sessadecimale}\index{Grado!Sessagesimale}
 	\tcblower
 	\begin{align*}
 	\alpha=&\ang{75}+\left(\dfrac{55}{60}\right)^{\si{\degree} }+\left(\dfrac{35}{3600}\right)^{\si{\degree} }\\
 	=&\ang{75}+\left(\dfrac{55\cdot60+35}{3600}\right)^{\si{\degree}}\\
 	=&\ang{75}+\left(\dfrac{3335}{3600}\right)^{\si{\degree}}\approx\ang[round-precision=\lungarrotandamento,round-mode=places]{75,92638}
 	\end{align*}
 \end{exercise}
  \begin{exercise}
  	Trasformare  $\alpha=\ang{38;21;10}$ in forma sessa-decimale\index{Grado!Sessadecimale}\index{Grado!Sessagesimale}
  	\tcblower
  	\begin{align*}
  	\alpha=&\ang{38}+\left(\dfrac{21}{60}\right)^{\si{\degree} }+\left(\dfrac{10}{3600}\right)^{\si{\degree} }\\
  	=&\ang{38}+\left(\dfrac{21\cdot60+10}{3600}\right)^{\si{\degree}}\\
  	=&\ang{38}+\left(\dfrac{1270}{3600}\right)^{\si{\degree}}\approx\ang[round-precision=\lungarrotandamento,round-mode=places]{38,35277778}
  	\end{align*}
  \end{exercise}
  \begin{exercise}
  	Trasformare  $\alpha=\ang{74;58;27}$ in forma sessa-decimale\index{Grado!Sessadecimale}\index{Grado!Sessagesimale}
  	\tcblower
  	\begin{align*}
  	\alpha=&\ang{74}+\left(\dfrac{58}{60}\right)^{\si{\degree} }+\left(\dfrac{27}{3600}\right)^{\si{\degree} }\\
  	=&\ang{74}+\left(\dfrac{58\cdot60+27}{3600}\right)^{\si{\degree}}\\
  	=&\ang{74}+\left(\dfrac{3507}{3600}\right)^{\si{\degree}}\approx\ang[round-precision=\lungarrotandamento,round-mode=places]{74,9741666667}
  	\end{align*}
  \end{exercise}
  \begin{exercise}
  	Trasformare  $\alpha=\ang{128;31;5}$ in forma sessa-decimale\index{Grado!Sessadecimale}\index{Grado!Sessagesimale}
  	\tcblower
  	\begin{align*}
  	\alpha=&\ang{128}+\left(\dfrac{31}{60}\right)^{\si{\degree} }+\left(\dfrac{5}{3600}\right)^{\si{\degree} }\\
  	=&\ang{128}+\left(\dfrac{31\cdot60+5}{3600}\right)^{\si{\degree}}\\
  	=&\ang{128}+\left(\dfrac{1865}{3600}\right)^{\si{\degree}}\approx\ang[round-precision=\lungarrotandamento,round-mode=places]{128,51666600667}
  	\end{align*}
  \end{exercise}
  \begin{exercise}
  	Trasformare  $\alpha=\ang{77;40;10}$ in forma sessa-decimale\index{Grado!Sessadecimale}\index{Grado!Sessagesimale}
  	\tcblower
  	\begin{align*}
  	\alpha=&\ang{77}+\left(\dfrac{40}{60}\right)^{\si{\degree} }+\left(\dfrac{10}{3600}\right)^{\si{\degree} }\\
  	=&\ang{77}+\left(\dfrac{40\cdot60+10}{3600}\right)^{\si{\degree}}\\
  	=&\ang{77}+\left(\dfrac{2410}{3600}\right)^{\si{\degree}}\approx\ang[round-precision=\lungarrotandamento,round-mode=places]{77,66944444}
  	\end{align*}
  \end{exercise}
\tcbstoprecording
\newpage
\section{Soluzioni da grado sessagesimale a sessa-decimale}
\tcbinputrecords
\section{Da grado sessadecimale a sessagesimale}
\tcbstartrecording
\begin{exercise}
	Convertire $\alpha=\ang{75.84}$ in gradi sessagesimali\index{Grado!Sessadecimale}\index{Grado!Sessagesimale}
	\tcblower
	Iniziamo con\index{Grado!Sessadecimale}\index{Grado!Sessagesimale}
	$\alpha=\ang{75}+\ang{0.84}$ in gradi sessagesimali
	\begin{align*}
	\alpha^{\si{\degree}}&=\ang{75}\\ 
	\alpha^{\si{\arcminute}}&=\ang{0.84}\cdot 60=\ang{;50.4;}=\ang{;50;}\\
	\alpha^{\si{\arcsecond}}&=\ang{;0.4;}\cdot 60=\ang{;;24}\\
	\end{align*}
	abbiamo quindi
	\[\alpha=\ang{75.84}=\ang{75;50;24}\]
\end{exercise}
\begin{exercise}
Convertire $\alpha=\ang{45.35}$ in gradi sessagesimali\index{Grado!Sessadecimale}\index{Grado!Sessagesimale}
	\tcblower
Iniziamo con 
$\alpha=\ang{45}+\ang{0.35}$
\begin{align*}
\alpha^{\si{\degree}}&=\ang{45}\\ 
\alpha^{\si{\arcminute}}&=\ang{0.35}\cdot 60=\ang{;21.0;}=\ang{;21;}\\
\alpha^{\si{\arcsecond}}&=\ang{;0;}\cdot 60=\ang{;;0}\\
\end{align*}
abbiamo quindi
\[\alpha=\ang{45.35}=\ang{45;21;0}\]
\end{exercise}
\begin{exercise}
	Convertire $\alpha=\ang{74.9742}$ in gradi sessagesimali\index{Grado!Sessadecimale}\index{Grado!Sessagesimale}
	\tcblower
	Iniziamo con 
	$\alpha=\ang{74}+\ang{0.9742}$
	\begin{align*}
	\alpha^{\si{\degree}}&=\ang{74}\\ 
	\alpha^{\si{\arcminute}}&=\ang{0.9742}\cdot 60=\ang{;58.452;}\\
	\alpha^{\si{\arcsecond}}&=\ang{;0.452;}\cdot 60=\ang{;;27.12}\\
	\end{align*}
	abbiamo quindi
	\[\alpha=\ang{74.9742}=\ang{74;58;27}\]
\end{exercise}
\begin{exercise}
	Convertire $\alpha=\ang{128.5167}$ in gradi sessagesimali\index{Grado!Sessadecimale}\index{Grado!Sessagesimale}
	\tcblower
	Iniziamo con 
	$\alpha=\ang{128}+\ang{0.5167}$
	\begin{align*}
	\alpha^{\si{\degree}}&=\ang{128}\\ 
	\alpha^{\si{\arcminute}}&=\ang{0.5167}\cdot 60=\ang{;31.002;}\\
	\alpha^{\si{\arcsecond}}&=\ang{;0.002;}\cdot 60=\ang{;;0.12}\\
	\end{align*}
	abbiamo quindi
	\[\alpha=\ang{128.5167}=\ang{128;31;0}\]
\end{exercise}
\begin{exercise}
	Convertire $\alpha=\ang{77.6694}$ in gradi sessagesimali\index{Grado!Sessadecimale}\index{Grado!Sessagesimale}
	\tcblower
	Iniziamo con 
	$\alpha=\ang{77}+\ang{0.6694}$
	\begin{align*}
	\alpha^{\si{\degree}}&=\ang{77}\\ 
	\alpha^{\si{\arcminute}}&=\ang{0.6694}\cdot 60=\ang{;40.164;}\\
	\alpha^{\si{\arcsecond}}&=\ang{;0.164;}\cdot 60=\ang{;;9.84}\\
	\end{align*}
	abbiamo quindi
	\[\alpha=\ang{77.6694}=\ang{77;40;9}\]
\end{exercise}
\tcbstoprecording
\newpage
\section{Soluzioni da grado sessadecimale a sessagesimale}
\tcbinputrecords
\chapter{Goniometria}
\label{cha:goniometriaEss}
\section{Trovare seno coseno e tangente}\index{Seno}\index{Coseno}\index{Tangente}
\begin{esempiot}{Trovare seno coseno e tangente}{exemplum1}
	Se $\sin\alpha=\dfrac{3}{5}$ con $\dfrac{\pi}{2}\leq\alpha\leq\pi$ determinare coseno e tangente.
\end{esempiot}
Se $\sin\alpha=\dfrac{3}{5}$ disegno la figura\nobs\vref{fig:esempio1}. Con il vincolo $\dfrac{\pi}{2}\leq\alpha\leq\pi$ il Punto $P$ è nel secondo quadrante, di conseguenza il coseno è negativo, come è negativa la tangente.
\begin{align*}
\sin\alpha&=\dfrac{3}{5}\\
\cos\alpha&=-\sqrt{1-\sin^2\alpha}=\\
&=-\sqrt{1-\dfrac{9}{25}}=\\
&=-\sqrt{\dfrac{25-9}{25}}=\\
&=-\sqrt{\dfrac{16}{25}}=\\
&=-\dfrac{4}{5}\\
\end{align*}
Di conseguenza la tangente è:
\[\tan\alpha=\dfrac{\sin\alpha}{\cos\alpha}=\dfrac{\dfrac{3}{5}}{-\dfrac{4}{5}}=\dfrac{3}{5}\cdot\left(-\dfrac{5}{4}\right)=-\dfrac{3}{4}\]
\begin{esempiot}{Trovare seno coseno e tangente}{}\index{Seno}\index{Coseno}\index{Tangente}
	Se $cos\alpha=\dfrac{7}{9}$ con $\dfrac{3\pi}{2}\leq\alpha\leq 2\pi$ determinare seno e tangente.
\end{esempiot}
L'angolo è nel quarto quadrante come lo mostra la figura\nobs\vref{fig:esempio2}, quindi seno e tangente sono negativi
\begin{align*}
\cos\alpha&=\dfrac{7}{9}\\
\sin\alpha&=-\sqrt{1-\cos^2\alpha}=\\
&=-\sqrt{1-\dfrac{49}{81}}=\\
&=-\sqrt{\dfrac{81-49}{81}}=\\
&=-\sqrt{\dfrac{32}{81}}=\\
&=-\dfrac{4\sqrt{2}}{9}\\
\end{align*}
\[\tan\alpha=\dfrac{\sin\alpha}{\cos\alpha}=\dfrac{-\dfrac{4\sqrt{2}}{9}}{\dfrac{7}{9}}=-\dfrac{4\sqrt{2}}{9}\cdot\dfrac{9}{7}=-\dfrac{4\sqrt{2}}{7}\]
\begin{esempiot}{Trovare seno coseno e tangente}{}\index{Seno}\index{Coseno}\index{Tangente}
	Se $sin\alpha=-\dfrac{9}{10}$ con $\pi\leq\alpha\leq\frac{3\pi}{2}$ determinare coseno e tangente.
\end{esempiot}
Se $\sin\alpha=-\dfrac{9}{10}$ disegno la figura\nobs\vref{fig:esempio3}. Con il vincolo $\pi\leq\alpha\leq\dfrac{3\pi}{2}$ il punto $P$ è nel terzo quadrante, di conseguenza il coseno è negativo, mentre  la tangente è positiva.
\begin{align*}
\sin\alpha&=-\dfrac{9}{10}\\
\cos\alpha&=-\sqrt{1-\sin^2\alpha}=\\
&=-\sqrt{1-\dfrac{81}{100}}=\\
&=-\sqrt{\dfrac{100-81}{100}}=\\
&=-\sqrt{\dfrac{19}{100}}=\\
&=-\dfrac{\sqrt{19}}{100}\\
\end{align*}
Di conseguenza la tangente è:
\[\tan\alpha=\dfrac{\sin\alpha}{\cos\alpha}=\dfrac{-\dfrac{9}{10}}{-\dfrac{\sqrt{19}}{10}}=\left(-\dfrac{9}{10}\right)\cdot\left(-\dfrac{\sqrt{19}}{10}\right)=\dfrac{9}{\sqrt{19}}=\dfrac{9\sqrt{19}}{19}\]
\begin{esempiot}{Trovare seno coseno e tangente}{}\index{Seno}\index{Coseno}\index{Tangente}
	Se $cos\alpha=-\dfrac{5}{6}$ con $\dfrac{\pi}{2}\leq\alpha\leq \pi$ determinare seno e tangente.
\end{esempiot}
L'angolo è nel secondo quadrante come lo mostra la figura\nobs\vref{fig:esempio4}, quindi seno è positivo e la tangente è negativa
\begin{align*}
\cos\alpha&=-\dfrac{5}{6}\\
\sin\alpha&=\sqrt{1-\cos^2\alpha}=\\
&=\sqrt{1-\dfrac{25}{36}}=\\
&=\sqrt{\dfrac{36-25}{36}}=\\
&=\sqrt{\dfrac{11}{36}}=\\
&=\dfrac{\sqrt{11}}{6}\\
\end{align*}
\[\tan\alpha=\dfrac{\sin\alpha}{\cos\alpha}=\dfrac{\dfrac{\sqrt{11}}{6}}{-\dfrac{5}{6}}=\dfrac{\sqrt{11}}{6}\cdot\left(-\dfrac{6}{5}\right)=-\dfrac{\sqrt{11}}{5}\]
\begin{figure}
\begin{subfigure}[b]{.5\linewidth}
\centering
\includestandalone[width=.8\textwidth]{terzo/grafici/EquaElementareSeno1}
\captionsetup{format=esempio}
\caption{Seno noto}\label{fig:esempio1}
\end{subfigure}%
\begin{subfigure}[b]{.5\linewidth}
\centering
\includestandalone[width=.8\textwidth]{terzo/grafici/EquaElementareCoseno1}
\captionsetup{format=esempio}
\caption{Coseno noto}\label{fig:esempio2}
\end{subfigure}%
\quad
\begin{subfigure}[b]{.5\linewidth}
\centering
\includestandalone[width=.8\textwidth]{terzo/grafici/EquaElementareSeno2}
\captionsetup{format=esempio}
\caption{Seno noto}\label{fig:esempio3}
\end{subfigure}%
\begin{subfigure}[b]{.5\linewidth}
\centering
\includestandalone[width=.8\textwidth]{terzo/grafici/EquaElementareCoseno2}
\captionsetup{format=esempio}
\caption{Coseno noto}\label{fig:esempio4}
\end{subfigure}%
\quad
\begin{subfigure}[b]{.5\linewidth}
	\centering
	\includestandalone[width=.8\textwidth]{terzo/grafici/EquaElementareTangente1}
	\caption{Tangente nota}\label{fig:esempio5}
\end{subfigure}%
\begin{subfigure}[b]{.5\linewidth}
	\centering
	\includestandalone[width=.8\textwidth]{terzo/grafici/EquaElementareTangente2}
	\caption{Tangente nota}\label{fig:esempio6}
\end{subfigure}%
\caption{Trovare seno, coseno e tangente}
\end{figure}
\begin{esempiot}{Trovare seno coseno e tangente}{}\index{Seno}\index{Coseno}\index{Tangente}
Se $\tan\alpha=\dfrac{1}{8}$ con $\pi\leq\alpha<\dfrac{3\pi}{2}$ determinare seno e coseno.
\end{esempiot}
Come dalla figura\nobs\vref{fig:esempio5} l'angolo è nel terzo quadrante quindi seno e coseno sono negativi.
\[\cos\alpha=-\dfrac{1}{\sqrt{1+\tan^2\alpha}}=-\dfrac{1}{\sqrt{1+\dfrac{1}{64}}}=-\dfrac{1}{\sqrt{\dfrac{64+1}{64}}}=-\dfrac{1}{\sqrt{\dfrac{65}{64}}}=-\dfrac{1}{\dfrac{\sqrt{65}}{8}}=-\dfrac{8}{\sqrt{65}}=-\dfrac{8\sqrt{65}}{65} \]
\[\sin\alpha=\tan\alpha\cdot\cos\alpha=\dfrac{1}{8}\cdot\left(-\dfrac{8\sqrt{65}}{65}\right)=-\dfrac{\sqrt{65}}{65} \]
\begin{esempiot}{Trovare seno coseno e tangente}{}
	Se $\tan\alpha=\dfrac{1}{8}$ con $\pi\leq\alpha<\dfrac{3\pi}{2}$ determinare seno e coseno.
\end{esempiot}
Come dalla figura\nobs\vref{fig:esempio6} l'angolo è nel terzo quadrante quindi seno e coseno sono negativi.
\[\cos\alpha=-\dfrac{1}{\sqrt{1+\tan^2\alpha}}=-\dfrac{1}{\sqrt{1+\dfrac{9}{25}}}=-\dfrac{1}{\sqrt{\dfrac{25+9}{25}}}=-\dfrac{1}{\sqrt{\dfrac{34}{25}}}=-\dfrac{1}{\dfrac{\sqrt{34}}{5}}=-\dfrac{5}{\sqrt{34}}=-\dfrac{5\sqrt{34}}{34} \]
\[\sin\alpha=\tan\alpha\cdot\cos\alpha=\left(-\dfrac{3}{5}\right)\cdot\left(-\dfrac{5\sqrt{34}}{34}\right)=\dfrac{3\sqrt{34}}{34} \]
\begin{esempiot}{Trovare seno coseno e tangente}{}\index{Seno}\index{Coseno}\index{Tangente}
	Se $\tastoisin\alpha=\dfrac{1}{8}$ con $\pi\leq\alpha<\dfrac{3\pi}{2}$ determinare seno e coseno.
\end{esempiot}
Come dalla figura\nobs\vref{fig:esempio5} l'angolo è nel terzo quadrante quindi seno e coseno sono negativi.
\[\cos\alpha=-\dfrac{1}{\sqrt{1+\tan^2\alpha}}=-\dfrac{1}{\sqrt{1+\dfrac{1}{64}}}=-\dfrac{1}{\sqrt{\dfrac{64+1}{64}}}=-\dfrac{1}{\sqrt{\dfrac{65}{64}}}=-\dfrac{1}{\dfrac{\sqrt{65}}{8}}=-\dfrac{8}{\sqrt{65}}=-\dfrac{8\sqrt{65}}{65} \]
\[\sin\alpha=\tan\alpha\cdot\cos\alpha=\dfrac{1}{8}\cdot\left(-\dfrac{8\sqrt{65}}{65}\right)=-\dfrac{\sqrt{65}}{65} \]
\begin{esempiot}{Trovare seno coseno e tangente}{}
	Se $\tan\alpha=\dfrac{1}{8}$ con $\pi\leq\alpha<\dfrac{3\pi}{2}$ determinare seno e coseno.
\end{esempiot}
Come dalla figura\nobs\vref{fig:esempio6} l'angolo è nel terzo quadrante quindi seno e coseno sono negativi.
\[\cos\alpha=-\dfrac{1}{\sqrt{1+\tan^2\alpha}}=-\dfrac{1}{\sqrt{1+\dfrac{9}{25}}}=-\dfrac{1}{\sqrt{\dfrac{25+9}{25}}}=-\dfrac{1}{\sqrt{\dfrac{34}{25}}}=-\dfrac{1}{\dfrac{\sqrt{34}}{5}}=-\dfrac{5}{\sqrt{34}}=-\dfrac{5\sqrt{34}}{34} \]
\[\sin\alpha=\tan\alpha\cdot\cos\alpha=\left(-\dfrac{3}{5}\right)\cdot\left(-\dfrac{5\sqrt{34}}{34}\right)=\dfrac{3\sqrt{34}}{34} \]

\chapter{Funzioni goniometriche usando la calcolatrice}
\label{cha:ValFunzGonioCalc}
Quando si parla di calcolatrice ci si riferisce a una calcolatrice scientifica. Devono essere presenti i tasti:
\begin{center}
 \begin{tabular}{ccc}
\tastosin&\tastocos&\tastotan \\ 
\end{tabular} 
\end{center}
In genere per le funzioni inverse si usa una combinazione di tasti \tastoshift 
\begin{center}
 \begin{tabular}{ccc}
 \tastoisin&\tastoicos&\tastoitan \\ 
 \end{tabular} 
\end{center}
La calcolatrice deve permettere di gestire i radianti e gradi. La calcolatrice deve avere  il numero $\pi$ \tastopgreco presente. Un tasto molto utile è il tasto \tastoans\ (dall'inglese answer risposta) che richiama l'ultimo risultato ottenuto. In una calcolatrice è facile trovare tre unità di misura per gli angoli i radianti $RAD$, i gradi centesimali $GRAD$ e i gradi sessagesimali $DEG$. Dobbiamo sempre essere in grado di verificare come è impostata la calcolatrice in modo da ottenere risultati corretti.
\begin{table}
	\centering
	\begin{tabular}{lll}
\toprule
Unità di misura		& Sigla &Inglese\\ 
\midrule
Radianti		&RAD &Radians \\ 
Gradi centesimali		&GRAD &Gradian \\ 
Gradi Sessagesimali		&DEG &Degree \\ 
\bottomrule
	\end{tabular} 
	\caption{Calcolatrice angoli}\label{tab:calcolatrice_angoli}
\end{table}
\section{Trovare valore funzione}
\subsection{Angolo in gradi}
\begin{esempiot}{Trovare valore funzione}{}
Calcolare \[\sin\ang{38;28;50}\] 
\end{esempiot}
Controllare che la calcolatrice sia impostata in gradi sessagesimali\index{Grado!Sessagesimale}.
Basta verificare che \testgradi In caso contrario modificare le impostazioni. 

Si inizia convertendo i gradi in forma sessadecimale\index{Grado!Sessagesimale}\index{Grado!Sessadecimale}\index{Seno}

\begin{align*}
&\phantom{=}\ang{38}+\left(\dfrac{28}{60}\right)^{\si{\degree}}+\left(\dfrac{50}{3600}\right)^{\si{\degree} }=\\
=&\ang{38}+\left(\dfrac{28\cdot60+50}{3600}\right)^{\si{\degree}}=\\
=&\ang{38}+\left(\dfrac{1680+50}{3600}\right)^{\si{\degree}}=\\
=&\ang{38}+\left(\dfrac{1730}{3600}\right)^{\si{\degree}}=\\
=&\ang[round-precision=\lungarrotandamento,round-mode=places]{38.4805556}
\end{align*}
usando la calcolatrice

\begin{center}
\begin{tabular}{ll}
\tasto{28}\tastoper\tasto{60}\tastouguale& 1680 \\ 
\tastoans\tastopiu\tasto{50}\tastouguale& 1730 \\
\tastoans\tastodiv\tasto{3600}\tastouguale& \num[round-precision=\lungarrotandamento,round-mode=places]{0.480555555} \\
\tastoans\tastopiu\tasto{38}\tastouguale&\num[round-precision=\lungarrotandamento,round-mode=places]{38.480555555} \\
\end{tabular}
\end{center} 

Infine

 \tastosin\tastoans\tastouguale e ottenere
\[\sin\ang{38;28;50}=\num[round-precision=\lungarrotandamento,round-mode=places]{0.622249007}\] 

\begin{esempiot}{Trovare valore funzione}{}
 Calcolare \[\tan\ang{120;30;40}\] 
\end{esempiot}
Controllare che la calcolatrice sia impostata in gradi sessagesimali\index{Grado!Sessagesimale}.
Basta verificare che \testgradi. In caso contrario modificare le impostazioni. 

Si inizia convertendo i gradi in forma sessadecimale\index{Grado!Sessagesimale}\index{Grado!Sessadecimale}\index{Tangente}

\begin{align*}
&\phantom{=}\ang{120}+\left(\dfrac{30}{60}\right)^{\si{\degree}}+\left(\dfrac{40}{3600}\right)^{\si{\degree} }\\
=&\ang{120}+\left(\dfrac{30\cdot60+40}{3600}\right)^{\si{\degree}}\\
=&\ang{120}+\left(\dfrac{1800+40}{3600}\right)^{\si{\degree}}\\
=&\ang{120}+\left(\dfrac{1730}{3600}\right)^{\si{\degree}}\\
=&\ang[round-precision=\lungarrotandamento,round-mode=places]{120.5111111}
\end{align*}

usando la calcolatrice

\begin{center}
 \begin{tabular}{ll}
 \tasto{30}\tastoper\tasto{60}\tastouguale & 1800 \\ 
 \tastoans\tastopiu\tasto{40}\tastouguale & 1840 \\
 \tastoans\tastodiv\tasto{3600}\tastouguale & \num[round-precision=\lungarrotandamento,round-mode=places]{0.511111111} \\
 \tastoans\tastopiu\tasto{120}\tastouguale&\num[round-precision=\lungarrotandamento,round-mode=places]{120.511111111} \\
 \end{tabular}
\end{center} 

Infine \tastotan \tastoans\tastouguale e ottenere
\[\tan\ang{120;30;40}=\num[round-precision=\lungarrotandamento,round-mode=places]{-1.69610537}\] 
\subsection{Angolo in radianti}
\begin{esempiot}{Trovare valore funzione}{}
 Calcolare \[\cos\dfrac{\pi}{4}\] 
\end{esempiot}
Controllare che la calcolatrice sia impostata in radianti\index{Radianti}\index{Coseno}.
Basta verificare che 
\testradianti
 In caso contrario modificare le impostazioni.

Non resta che procedere con il calcolo
 
\begin{center}
\begin{tabular}{ll}
 \tastopgreco\tastodiv\tasto{4}\tastouguale& \num[round-precision=\lungarrotandamento,round-mode=places]{0.785398163} \\ 
\tastocos\tastoans\tastouguale &\num[round-precision=\lungarrotandamento,round-mode=places]{0.707106781} \\ 
\end{tabular} 
\end{center}
\[\cos\dfrac{\pi}{4}=\num[round-precision=\lungarrotandamento,round-mode=places]{0.707106781}\] 
\begin{esempiot}{Trovare valore funzione}{}
 Calcolare \[\tan\SI[round-precision=4,round-mode=places]{1.4589}{\radian}\] 
\end{esempiot}
Controllare che la calcolatrice sia impostata in radianti\index{Radianti}\index{Tangente}.
Basta verificare che 
\testradianti
In caso contrario modificare le impostazioni.

Non resta che procedere con il calcolo

\begin{center}
 \begin{tabular}{ll}
 \tastotan\tasto{\num[round-precision=4,round-mode=places]{1.4589}}\tastouguale& \num[round-precision=\lungarrotandamento,round-mode=places]{8.899513904}\\ 
 \end{tabular} 
\end{center}
otteniamo
 \[\tan\SI[round-precision=4,round-mode=places]{1.4589}{\radian}=\num[round-precision=\lungarrotandamento,round-mode=places]{8.899513904}\] 
 \section{Trovare l'angolo nota la funzione}
 \subsection{Angolo in gradi}
 \begin{esempiot}{Trovare valore funzione}{}
 Trovare l'angolo per cui \[\cos x=\num[round-precision=\lungarrotandamento,round-mode=places]{0.778934}\]
 \end{esempiot}
Controllare che la calcolatrice sia impostata in gradi sessagesimali\index{Grado!Sessagesimale}.
Basta verificare che \testgradi 

In caso contrario modificare le impostazioni.

Le soluzioni sono 
\[\begin{cases}
 x_1=+\alpha+k\ang{360}\\
 x_2=-\alpha+k\ang{360}\\
\end{cases}\]
Calcolo $\alpha$

\begin{center}
 \begin{tabular}{ll}
 \tastoicos\tasto{\num[round-precision=\lungarrotandamento,round-mode=places]{0.778934}}\tastouguale&\SI[round-precision=\lungarrotandamento,round-mode=places]{38.8369232}{\si{\degree}}\\
 \end{tabular}
\end{center}

Convertiamo in gradi sessagesimali

\begin{center} 
 \begin{tabular}{ll}
 \tastoans\tastomeno\tasto{38}\tastouguale&\SI[round-precision=\lungarrotandamento,round-mode=places]{0.810314895}{\si{\degree}}\\
 \tastoans\tastoper\tasto{60}\tastouguale&\SI[round-precision=\lungarrotandamento,round-mode=places]{50.21539175}{\arcminute}\\
 \tastoans\tastomeno\tasto{50}\tastouguale&\SI[round-precision=\lungarrotandamento,round-mode=places]{0.215391754}{\arcminute}\\
 \tastoans\tastoper\tasto{60}\tastouguale&\SI[round-precision=\lungarrotandamento,round-mode=places]{12.929350526}{\arcsecond}\\
 \end{tabular} 
\end{center}
\[\alpha=\ang{38;50;12}\]
le soluzioni sono quindi
\[\begin{cases}
x_1=+\ang{38;50;12}+k\ang{360}\\
x_2=-\ang{38;50;12}+k\ang{360}\\
\end{cases}\]
 \subsection{Angolo in radianti}
 \begin{esempiot}{Trovare valore funzione}{}
 Calcolare \[\sin\alpha=\num[round-precision=\lungarrotandamento,round-mode=places]{-0.783942}\] 
 \end{esempiot}
 Controllare che la calcolatrice sia impostata in radianti\index{Radianti}\index{Seno}.
 Basta verificare che 
 \testradianti
 In caso contrario modificare le impostazioni.
 
 Non resta che procedere con il calcolo
 
 \begin{center}
 \begin{tabular}{ll}
 \tastoisin\tasto{\num[round-precision=\lungarrotandamento,round-mode=places]{-0.783942}}\tastouguale&\num[round-precision=\lungarrotandamento,round-mode=places]{-0.900990129}\\ \tasto{2}\tastoper\tastopgreco\tastopiu\tastoans\tastouguale&\num[round-precision=\lungarrotandamento,round-mode=places]{5.382195178}\\
 \end{tabular} 
 \end{center}
 \[\rho= \SI[round-precision=\lungarrotandamento,round-mode=places]{-0.900990129}{\radian}\]
 \[\rho= \SI[round-precision=\lungarrotandamento,round-mode=places]{5.382195178}{\radian}\] 
\tcbstartrecording
\chapter{Equazioni goniometriche}
\label{cha:EquazioniGoniometriche}
 \section{Esercizi equazioni elementari}
% \tcbstartrecording
 \begin{exercise}
Trovare l'angolo in gradi per cui $\cos x=\num[round-precision=\lungarrotandamento,round-mode=places]{-0.7548329}$
\tcblower
$\cos x=\num[round-precision=\lungarrotandamento,round-mode=places]{-0.7548329}$

 Controllare che la calcolatrice sia impostata in gradi\index{Grado!test}.
 Basta verificare che \testgradi 
 
 In caso contrario modificare le impostazioni.
 
 Le soluzioni sono 
 \[\begin{cases}
 x_1=+\alpha+k\ang{360}\\
 x_2=-\alpha+k\ang{360}\\
 \end{cases}\]
 Calcolo $\alpha$
 
 \begin{center}
 \begin{tabular}{ll}
 \tastoicos\tasto{\num[round-precision=\lungarrotandamento,round-mode=places]{-0.7548329}}\tastouguale&\SI[round-precision=\lungarrotandamento,round-mode=places]{139.0107711}{\si{\degree}}
 \end{tabular}
 \end{center}
 
 Converto in gradi sessagesimali\index{Grado!Sessagesimale}
 
 \begin{center}
 \begin{tabular}{ll}
 \tastoans\tastomeno\tasto{139}\tastouguale&\SI[round-precision=\lungarrotandamento,round-mode=places]{0.010771083}{\si{\degree}}\\
 \tastoans\tastoper\tasto{60}\tastouguale&\SI[round-precision=\lungarrotandamento,round-mode=places]{0.646265034}{\si{\arcminute}}\\
 \tastoans\tastomeno\tasto{0}\tastouguale&\SI[round-precision=\lungarrotandamento,round-mode=places]{0.646265034}{\si{\arcminute}}\\
 \tastoans\tastoper\tasto{60}\tastouguale&\SI[round-precision=\lungarrotandamento,round-mode=places]{38.77590204}{\si{\arcsecond}}\\
 \end{tabular} 
 \end{center}
 \[\alpha=\ang{139;0;38}\]
 le soluzioni sono quindi
 \[\begin{cases}
 x_1=+\ang{139;0;38}+k\ang{360}\\
 x_2=-\ang{139;0;38}+k\ang{360}\\
 \end{cases}\]
 \end{exercise}
 \begin{exercise}
 Trovare l'angolo in gradi per cui $\sin x=\num[round-precision=\lungarrotandamento,round-mode=places]{0.666666666}$
\tcblower
 $\sin x=\num[round-precision=\lungarrotandamento,round-mode=places]{0.666666666}$
 
Controllare che la calcolatrice sia impostata in gradi\index{Grado!test} sessagesimali\index{Grado!Sessagesimale}\index{Seno}.
 
 Basta verificare che 
 
\testgradi 
 
 In caso contrario modificare le impostazioni.
 
 Le soluzioni sono 
 \[\begin{cases}
 x_1=\alpha+k\ang{360}\\
 x_2=\ang{180}-\alpha+k\ang{360}\\
 \end{cases}\]
 Calcolo $\alpha$
 \begin{center}
 \begin{tabular}{ll}
 \tastoisin\tasto{\num[round-precision=\lungarrotandamento,round-mode=places]{0.666666666}}\tastouguale&\SI[round-precision=\lungarrotandamento,round-mode=places]{41.8103149}{\si{\degree}}\\
 \end{tabular}
 \end{center}
 \[\begin{cases}
 x_1=\alpha+k\ang{360}=\SI[round-precision=\lungarrotandamento,round-mode=places]{41.8103149}{\si{\degree}}+k\ang{360}\\
 x_2=\ang{180}-\alpha+k\ang{360}=\SI[round-precision=\lungarrotandamento,round-mode=places]{138.1896851}{\si{\degree}}+k\ang{360}\\
 \end{cases}\]
 
 Converto in gradi sessagesimali\index{Grado!Sessagesimale} $x_1$
 
 \begin{center} 
 \begin{tabular}{ll}
 \tastoans\tastomeno\tasto{41}\tastouguale&\SI[round-precision=\lungarrotandamento,round-mode=places]{0.810314895}{\si{\degree}}\\
 \tastoans\tastoper\tasto{60}\tastouguale&\SI[round-precision=\lungarrotandamento,round-mode=places]{48.61889374}{\si{\arcminute}}\\
 \tastoans\tastomeno\tasto{48}\tastouguale&\SI[round-precision=\lungarrotandamento,round-mode=places]{0.68893743}{\si{\arcminute}}\\
 \tastoans\tastoper\tasto{60}\tastouguale&\SI[round-precision=\lungarrotandamento,round-mode=places]{37.13362463}{\si{\arcsecond}}\\
 \end{tabular} 
 \end{center}
 \[x_1=\ang{41;48;37}\]
 
 Converto in gradi sessagesimali\index{Grado!Sessagesimale} $x_2$
 
 \begin{center} 
 \begin{tabular}{ll}
 \tastoans\tastomeno\tasto{138}\tastouguale&\SI[round-precision=\lungarrotandamento,round-mode=places]{0.189685104}{\si{\degree}}\\
 \tastoans\tastoper\tasto{60}\tastouguale&\SI[round-precision=\lungarrotandamento,round-mode=places]{11.38110625}{\si{\arcminute}}\\
 \tastoans\tastomeno\tasto{48}\tastouguale&\SI[round-precision=\lungarrotandamento,round-mode=places]{0.381106252}{\si{\arcminute}}\\ \tastoans\tastoper\tasto{60}\tastouguale&\SI[round-precision=\lungarrotandamento,round-mode=places]{22.866375512}{\si{\arcsecond}}\\
 \end{tabular} 
 \end{center}
 \[x_2=\ang{138;11;22}\]
 
 le soluzioni sono quindi
 \[\begin{cases}
 x_1=\ang{41;48;37}+k\ang{360}\\
 x_2=\ang{138;11;22}+k\ang{360}\\
 \end{cases}\]
 \end{exercise}
 \begin{exercise}
 Trovare l'angolo in gradi per cui $\sin x=\num[round-precision=2,round-mode=places]{-0.75}$
\tcblower
 $\sin x=\num[round-precision=2,round-mode=places]{-0.75}$

 Controllare che la calcolatrice sia impostata in gradi\index{Grado!test} sessagesimali\index{Grado!Sessagesimale}\index{Seno}.
 
 Basta verificare che 
 \testgradi 
 
 In caso contrario modificare le impostazioni.
 
 Le soluzioni sono 
 \[\begin{cases}
 x_1=\alpha+k\ang{360}\\
 x_2=\ang{180}-\alpha+k\ang{360}\\
 \end{cases}\]
 Calcolo $\alpha$
 
 \begin{center}
 \begin{tabular}{ll}
 \tastoisin\tasto{\num[round-precision=2,round-mode=places]{-0.75}}\tastouguale&\SI[round-precision=\lungarrotandamento,round-mode=places]{-48.59037789}{\si{\degree}}
 \end{tabular}
 \end{center}
 
 \[\begin{cases}
 x_1=\alpha+k\ang{360}=\SI[round-precision=\lungarrotandamento,round-mode=places]{-48.59037789}{\si{\degree}}+k\ang{360}=\SI[round-precision=\lungarrotandamento,round-mode=places]{311.4096221}{\si{\degree}}+k\ang{360}\\
 x_2=-\ang{180}+\alpha+k\ang{360}=\SI[round-precision=\lungarrotandamento,round-mode=places]{228.5903789}{\si{\degree}}+k\ang{360}\\
 \end{cases}\]
 
 Converto in gradi sessagesimali\index{Grado!Sessagesimale} $x_1$
 
 \begin{center} 
 \begin{tabular}{ll}
 \tastoans\tastomeno\tasto{311}\tastouguale&\SI[round-precision=\lungarrotandamento,round-mode=places]{0.409622109}{\si{\degree}}\\
 \tastoans\tastoper\tasto{60}\tastouguale&\SI[round-precision=\lungarrotandamento,round-mode=places]{24.57732655}{\si{\arcminute}}\\
 \tastoans\tastomeno\tasto{48}\tastouguale&\SI[round-precision=\lungarrotandamento,round-mode=places]{0.577326552}{\si{\arcminute}}\\
 \tastoans\tastoper\tasto{60}\tastouguale&\SI[round-precision=\lungarrotandamento,round-mode=places]{34.63959312}{\si{\arcsecond}}\\
 \end{tabular} 
 \end{center}
 \[x_1=\ang{311;24;34}\]
 
 Converto in gradi sessagesimali\index{Grado!Sessagesimale} $x_2$
 
 \begin{center} 
 \begin{tabular}{ll}
 \tastoans\tastomeno\tasto{228}\tastouguale&\SI[round-precision=\lungarrotandamento,round-mode=places]{0.59037789}{\si{\degree}}\\
 \tastoans\tastoper\tasto{60}\tastouguale&\SI[round-precision=\lungarrotandamento,round-mode=places]{35.42267345}{\si{\arcminute}}\\
 \tastoans\tastomeno\tasto{48}\tastouguale&\SI[round-precision=\lungarrotandamento,round-mode=places]{0.422673448}{\si{\arcminute}}\\
 \tastoans\tastoper\tasto{60}\tastouguale&\SI[round-precision=\lungarrotandamento,round-mode=places]{25.36040688}{\si{\arcsecond}}\\
 \end{tabular} 
 \end{center}
 \[x_2=\ang{228;35;25}\]
 
 le soluzioni sono quindi
 \[\begin{cases}
 x_1=\ang{311;24;34}+k\ang{360}\\
 x_2=\ang{228;35;25}+k\ang{360}\\
 \end{cases}\]
 \end{exercise}
 \begin{exercise}
Trovare l'angolo in gradi per cui $\tan x=\num[round-precision=\lungarrotandamento,round-mode=places]{1.414213562}$
\tcblower
$\tan x=\num[round-precision=\lungarrotandamento,round-mode=places]{1.414213562}$

 Controllare che la calcolatrice sia impostata in gradi.
 Basta verificare che 
 
\testgradi 
 
In caso contrario modificare le impostazioni.

Le soluzioni sono \[x_1=\alpha+k\ang{180}\]

Converto in gradi sessagesimali\index{Grado!Sessagesimale} $x_1$
 \begin{center}
 \begin{tabular}{ll}
 \tastoitan\tasto{\num[round-precision=\lungarrotandamento,round-mode=places]{1.414213562}}
 \tastouguale&\SI[round-precision=\lungarrotandamento,round-mode=places]{54.73561032}{\si{\degree}}\\
 \end{tabular}
\end{center} 

 Converto in gradi sessagesimali\index{Grado!Sessagesimale} $x_1$

 \begin{center}
 \begin{tabular}{ll}
 \tastoans\tastomeno\tasto{54}\tastouguale&\SI[round-precision=\lungarrotandamento,round-mode=places]{0.735610317}{\si{\degree}}\\
 \tastoans\tastoper\tasto{60}\tastouguale&\SI[round-precision=\lungarrotandamento,round-mode=places]{44.13661903}{\si{\arcminute}}\\
 \tastoans\tastomeno\tasto{44}\tastouguale&\SI[round-precision=\lungarrotandamento,round-mode=places]{0.136619034}{\si{\arcminute}}\\
 \tastoans\tastoper\tasto{60}\tastouguale&\SI[round-precision=\lungarrotandamento,round-mode=places]{8.197142083}{\si{\arcsecond}}\\
 \end{tabular} 
 \end{center}
Le soluzioni sono \[x_1=\ang{54;44;8}+k\ang{180}\]
 \end{exercise}
 \begin{exercise}
 Trovare l'angolo in gradi per cui $\tan x=\num[round-precision=\lungarrotandamento,round-mode=places]{-3.464101615}$
 \tcblower

 $\tan x=\num[round-precision=\lungarrotandamento,round-mode=places]{-3.464101615}$
 
 Controllare che la calcolatrice sia impostata in gradi.
 
 Basta verificare che 
 \testgradi 
 
 In caso contrario modificare le impostazioni.
 
 Le soluzioni sono \[x_1=\alpha+k\ang{180}\]
 
 Converto in gradi sessagesimali\index{Grado!Sessagesimale} $x_1$
 \begin{center}
 \begin{tabular}{ll}
 \tastoitan\tasto{\num[round-precision=\lungarrotandamento,round-mode=places]{-3.464101615}}\tastouguale&\SI[round-precision=\lungarrotandamento,round-mode=places]{-73.89788625}{\si{\degree}}\\
 \end{tabular}
 \end{center} 
 
 $x_1=\ang{180}\SI[round-precision=\lungarrotandamento,round-mode=places]{-73.89788625}{\si{\degree}}=\SI[round-precision=\lungarrotandamento,round-mode=places]{106.1021138}{\si{\degree}}$
 
 Converto in gradi sessagesimali\index{Grado!Sessagesimale} $x_1$
 \begin{center}
 \begin{tabular}{ll}
 \tastoans\tastomeno\tasto{106}\tastouguale&\SI[round-precision=\lungarrotandamento,round-mode=places]{0.10211375}{\si{\degree}}\\
 \tastoans\tastoper\tasto{60}\tastouguale&\SI[round-precision=\lungarrotandamento,round-mode=places]{6.126825}{\si{\arcminute}}\\
 \tastoans\tastomeno\tasto{6}\tastouguale&\SI[round-precision=\lungarrotandamento,round-mode=places]{0.126825}{\si{\arcminute}}\\
 \tastoans\tastoper\tasto{60}\tastouguale&\SI[round-precision=\lungarrotandamento,round-mode=places]{7.6095}{\si{\arcsecond}}\\
 \end{tabular} 
 \end{center}
 Le soluzioni sono \[x_1=\ang{106;6;7}+k\ang{180}\]
 \end{exercise}
 \begin{exercise}[no solution]
 Trovare l'angolo in gradi per cui $\tan x=\dfrac{\sqrt{3}}{2}$
\end{exercise}
 \begin{exercise}[no solution]
 Trovare l'angolo in gradi per cui $\cos x=\dfrac{3}{5}$
 \end{exercise}
 \begin{exercise}[no solution]
 Trovare l'angolo in gradi per cui $\sin x=-\dfrac{\sqrt{3}}{2}$
 \end{exercise}
 \begin{exercise}
 Trovare l'angolo in radianti per cui $\cos x=\num[round-precision=\lungarrotandamento,round-mode=places]{-0.478973}$
 \tcblower
 $\cos x=\num[round-precision=\lungarrotandamento,round-mode=places]{-0.478973}$
 
 Controllare che la calcolatrice sia impostata in radianti\index{Radianti!test}\index{Radianti}\index{Coseno}.
 
 Basta verificare che 
 
 \testradianti
 
 In caso contrario modificare le impostazioni.
 
 Non resta che procedere con il calcolo.
 
 Le soluzioni sono 
 \[\begin{cases}
 x_1=+\alpha+2k\pi\\
 x_2=-\alpha+2k\pi\\
 \end{cases}\]
 Calcolo $\alpha$
 \begin{center}
 \begin{tabular}{ll}
 \tastoicos\tasto{\num[round-precision=\lungarrotandamento,round-mode=places]{-0.4788973}}\tastouguale&\SI[round-precision=\lungarrotandamento,round-mode=places]{2.070280734}{\radian}\\ 
 \end{tabular} 
 \end{center}
 \[\alpha= \SI[round-precision=\lungarrotandamento,round-mode=places]{2.070280734}{\radian}\]
 \[\begin{cases}
 x_1=+\SI[round-precision=\lungarrotandamento,round-mode=places]{2.070280734}+2k\pi \si{\radian}\\
 x_2=-\SI[round-precision=\lungarrotandamento,round-mode=places]{2.070280734}+2k\pi \si{\radian}\\
 \end{cases}\]
 \end{exercise}
% \tcbstoprecording
% \newpage
% \section{Soluzioni equazioni goniometriche elementari}
% \tcbinputrecords
% \newpage
\section{Esercizi equazioni goniometriche}
% \tcbstartrecording
 \begin{exercise}
 Trovare l'angolo in radianti per cui $\sin 3x=\num[round-precision=2,round-mode=places]{0.48}$
 \tcblower
$\sin 3x=\num[round-precision=2,round-mode=places]{0.48}$ 
 
 Controllare che la calcolatrice sia impostata in radianti\index{Radianti!test}\index{Radianti}\index{Seno}.
 
 Basta verificare che 
 \testradianti
 
 In caso contrario modificare le impostazioni.
 
 Non resta che procedere con il calcolo.
 
 Le soluzioni sono 
 \[\begin{cases}
 x_1=\alpha+2k\pi\\
 x_2=\pi-\alpha+2k\pi\\
 \end{cases}\]
 Calcolo $\alpha$
 
 \begin{center}
 \begin{tabular}{ll}
 \tastoisin\tasto{\num[round-precision=2,round-mode=places]{0.48}}
 \tastouguale&\num[round-precision=\lungarrotandamento,round-mode=places]{0.500654712}\\ 
 \end{tabular} 
 \end{center}
 \[\alpha= \SI[round-precision=\lungarrotandamento,round-mode=places]{0.500654712}{\radian}\]
 \begin{align*}
 3x_1&=\SI[round-precision=\lungarrotandamento,round-mode=places]{0.500654712}+2k\pi \si{\radian}\\
 x_1&=\SI[round-precision=\lungarrotandamento,round-mode=places]{0.166884904}+\dfrac{2}{3}k\pi \si{\radian}\\
 \end{align*}
 \begin{align*}
 3x_2&=\pi-\SI[round-precision=\lungarrotandamento,round-mode=places]{0.500654712}+2k\pi \si{\radian}\\
 3x_2&=\SI[round-precision=\lungarrotandamento,round-mode=places]{2.640937182}+2k\pi \si{\radian}\\
 x_2&=\SI[round-precision=\lungarrotandamento,round-mode=places]{0.880312394}+\dfrac{2}{3}k\pi \si{\radian}\\
 \end{align*}
 Le soluzioni sono
 
\[\begin{cases}
x_1=\SI[round-precision=\lungarrotandamento,round-mode=places]{0.166884904}+\dfrac{2}{3}k\pi \si{\radian}\\
\\
x_2=\SI[round-precision=\lungarrotandamento,round-mode=places]{0.880312394}+\dfrac{2}{3}k\pi \si{\radian}\\
 \end{cases}\]
 \end{exercise}
 \begin{exercise}
 Trovare l'angolo in radianti per cui $\sin 2x=-\num[round-precision=3,round-mode=places]{0.128}$
 \tcblower
 $\sin 2x=-\num[round-precision=3,round-mode=places]{0.128}$ 
 
 Controllare che la calcolatrice sia impostata in radianti\index{Radianti!test}\index{Radianti}\index{Seno}.
 
 Basta verificare che 
 \testradianti
 
 In caso contrario modificare le impostazioni.
 
 Non resta che procedere con il calcolo.
 
 Le soluzioni sono 
 \[\begin{cases}
 x_1=\alpha+2k\pi\\
 x_2=\pi-\alpha+2k\pi\\
 \end{cases}\]
 Calcolo $\alpha$
 
 \begin{center}
 \begin{tabular}{ll}
 \tastoisin\tasto{\num[round-precision=3,round-mode=places]{-0.128}}
 \tastouguale&\num[round-precision=\lungarrotandamento,round-mode=places]{-0.128352127} 
 \end{tabular} 
 \end{center}
 Soluzioni per valori negativi dell'angolo
 \begin{align*}
 	2x_1&=\SI[round-precision=\lungarrotandamento,round-mode=places]{-0.128352127}+2k\pi \si{\radian}\\
 	x_1&=\SI[round-precision=\lungarrotandamento,round-mode=places]{-0.064176063}+k\pi \si{\radian}\\
 \end{align*}
 \begin{align*}
 	2x_2&=-\pi-(\SI[round-precision=\lungarrotandamento,round-mode=places]{-0.128352127})+2k\pi \si{\radian}\\
 	2x_2&=-\SI[round-precision=\lungarrotandamento,round-mode=places]{3.013240526}+2k\pi \si{\radian}\\
 	x_2&=-\SI[round-precision=\lungarrotandamento,round-mode=places]{1.506620263}+k\pi \si{\radian}\\
 \end{align*}
 Le soluzioni sono
 
 \[\begin{cases}
 x_1&=\SI[round-precision=\lungarrotandamento,round-mode=places]{-0.064176063}+k\pi \si{\radian}\\
 x_2&=-\SI[round-precision=\lungarrotandamento,round-mode=places]{1.506620263}+k\pi \si{\radian}\\
 \end{cases}\]
 Soluzioni per valori positivi dell'angolo
 \[\alpha=2\pi+ \SI[round-precision=\lungarrotandamento,round-mode=places]{-0.128352127}{\radian}\]
 \begin{align*}
 2x_1&=\SI[round-precision=\lungarrotandamento,round-mode=places]{6.154833179}+2k\pi \si{\radian}\\
 x_1&=\SI[round-precision=\lungarrotandamento,round-mode=places]{3,0774165895}+k\pi \si{\radian}\\
 \end{align*}
 \begin{align*}
 2x_2&=\pi+\SI[round-precision=\lungarrotandamento,round-mode=places]{0.128352127}+2k\pi \si{\radian}\\
 2x_2&=\SI[round-precision=\lungarrotandamento,round-mode=places]{3.2699447805897}+2k \si{\radian}\\
 x_2&=\SI[round-precision=\lungarrotandamento,round-mode=places]{1.63497239}+k\pi \si{\radian}\\
 \end{align*}
 Le soluzioni sono
 
 \[\begin{cases}
 x_1&=\SI[round-precision=\lungarrotandamento,round-mode=places]{3.205768717}+k\pi \si{\radian}\\
 x_2&=\SI[round-precision=\lungarrotandamento,round-mode=places]{1.63497239}+k\pi \si{\radian}\\
 \end{cases}\]
 
 \end{exercise}
 
 \begin{exercise}
 	Trovare l'angolo in radianti per cui $\sin 6x=-\dfrac{5}{8}$
 	\tcblower
 	$\sin 6x=-\num[round-precision=3,round-mode=places]{0.625}$ 
 	
 	Controllare che la calcolatrice sia impostata in gradi \index{Seno}.
 	
 	Basta verificare che 
 	\testgradi
 	
 	In caso contrario modificare le impostazioni.
 	
 	Non resta che procedere con il calcolo.
 	
 	Le soluzioni sono 
 	\[\begin{cases}
 	x_1=\alpha+k\ang{360;;}\\
 	x_2=\pi-\alpha+k\ang{360;;}\\
 	\end{cases}\]
 	Calcolo $\alpha$
 	
 	\begin{center}
 		\begin{tabular}{ll}
 			\tastoisin\tasto{-\num[round-precision=3,round-mode=places]{0.625}}
 			\tastouguale&\num[round-precision=\lungarrotandamento,round-mode=places]{-38.68218745} 
 		\end{tabular} 
 	\end{center}
 	Soluzioni per valori negativi dell'angolo
 	\begin{align*}
 	6x_1&=-\SI[round-precision=\lungarrotandamento,round-mode=places]{38.68218745}{\si{\degree}}+k\ang{360;;}\\
 	x_1&=-\SI[round-precision=\lungarrotandamento,round-mode=places]{6.447031242}{\si{\degree}}+k\ang{60;;}\\
 	\end{align*}
 	\begin{align*}
 	6x_2&=-\ang{180;;}-(-\SI[round-precision=\lungarrotandamento,round-mode=places]{38.68218745}{\si{\degree}})+k\ang{360}\\
 	6x_2&=-\SI[round-precision=\lungarrotandamento,round-mode=places]{141.3178125}{\si{\degree}}+k\ang{360}\\
 	x_2&=-\SI[round-precision=\lungarrotandamento,round-mode=places]{23.55296876}{\si{\degree}}+k\ang{60}\\
 	\end{align*}
 	Le soluzioni sono
 	
 	\[\begin{cases}
 	x_1&=-\SI[round-precision=\lungarrotandamento,round-mode=places]{6.447031242}{\si{\degree}}+k\ang{60;;}\\
 	x_2&=-\SI[round-precision=\lungarrotandamento,round-mode=places]{23.55296876}{\si{\degree}}+k\ang{60}\\
 	\end{cases}\]
 	Soluzioni per valori positivi dell'angolo
 	\[\alpha=\ang{360} -\SI[round-precision=\lungarrotandamento,round-mode=places]{38.682187745}{\si{\degree}}=\SI[round-precision=\lungarrotandamento,round-mode=places]{141.3178125}{\si{\degree}}\]
 	\begin{align*}
 	6x_1&=\SI[round-precision=\lungarrotandamento,round-mode=places]{141.3178125}{\si{\degree}} +k\ang{360}\\
 	x_1&=\SI[round-precision=\lungarrotandamento,round-mode=places]{23.552966875}{\si{\degree}} +k\ang{60}\\
 	\end{align*}
 	\[\alpha=\ang{360} -\ang{180;;}-(-\SI[round-precision=\lungarrotandamento,round-mode=places]{38.68218745}{\si{\degree}})=\SI[round-precision=\lungarrotandamento,round-mode=places]{218.68221875}{\si{\degree}}\]
 	\begin{align*}
 	6x_2&=\SI[round-precision=\lungarrotandamento,round-mode=places]{218.68221875}{\si{\degree}} +k\ang{60}\\
 	x_2&=\SI[round-precision=\lungarrotandamento,round-mode=places]{36.44703646}{\si{\degree}} +k\ang{60}\\
 	\end{align*}
 	Le soluzioni sono
 	
 	\[\begin{cases}
 	x_1&=\SI[round-precision=\lungarrotandamento,round-mode=places]{23.552966875}{\si{\degree}} +k\ang{60}\\
 	x_2&=\SI[round-precision=\lungarrotandamento,round-mode=places]{36.44703646}{\si{\degree}} +k\ang{60}\\
 	\end{cases}\]
% 	
\end{exercise}
 \begin{exercise}
 Trovare l'angolo in radianti per cui $\cos 2x=\num[round-precision=3,round-mode=places]{0.128}$
 \tcblower
 $\cos 2x=\num[round-precision=3,round-mode=places]{0.128}$ 
 
 Controllare che la calcolatrice sia impostata in radianti\index{Radianti!test}\index{Radianti}\index{Coseno}.
 
 Basta verificare che 
 \testradianti
 
 In caso contrario modificare le impostazioni.
 
 Non resta che procedere con il calcolo.
 
 Le soluzioni sono 
 \[\begin{cases}
 x_1=+\alpha+2k\pi\\
 x_2=-\alpha+2k\pi\\
 \end{cases}\]
 Calcolo $\alpha$
 
 \begin{center}
 \begin{tabular}{ll}
 \tastoicos\tasto{\num[round-precision=3,round-mode=places]{0.128}}
 \tastouguale&\num[round-precision=\lungarrotandamento,round-mode=places]{1.442444199} 
 \end{tabular} 
 \end{center}
 \[\alpha=\SI[round-precision=\lungarrotandamento,round-mode=places]{1.442444199}{\radian} +2k\pi\]
 \begin{align*}
 2x_1&=\SI[round-precision=\lungarrotandamento,round-mode=places]{1.442444199}+2k\pi \si{\radian}\\
 x_1&=\SI[round-precision=\lungarrotandamento,round-mode=places]{0.721222099}+k\pi \si{\radian}\\
 \end{align*}
 Le soluzioni sono
 
 \[\begin{cases}
 x_1&=+\SI[round-precision=\lungarrotandamento,round-mode=places]{0.721222099}+k\pi \si{\radian}\\\
 
 x_2&=-\SI[round-precision=\lungarrotandamento,round-mode=places]{0.721222099}+k\pi \si{\radian}\\ 
 \end{cases}\]
 \end{exercise}
 \begin{exercise}
 Trovare l'angolo in gradi per cui $\tan 3x=\dfrac{3}{5}$
 \tcblower
 $\tan 3x=\dfrac{3}{5}$
 
 Controllare che la calcolatrice sia impostata in gradi\index{Grado!test}\index{Grado!test}\index{Tangente}.
 
 Basta verificare che 
 \testgradi
 
 In caso contrario modificare le impostazioni.
 
 Non resta che procedere con il calcolo.
 
 Esistenza 
 \begin{align*}
 \alpha\neq&\ang{90;;}+k\ang{180;;}\\
 3x=&\ang{90;;}+k\ang{180;;}\\
 x=&\ang{30;;}+k\ang{60;;}\\
 x\neq&\ang{30;;}+k\ang{60;;}\\
 \end{align*}
 Soluzione
 \begin{align*}
 \tan 3x=&\frac{3}{5}\\
 \tan y=&\frac{3}{5}\\
 \intertext{Calcoliamo $y$}
 \tastoitan\tasto{\num[round-precision=1,round-mode=places]{0.6}}\tastouguale&\quad\SI[round-precision=\lungarrotandamento,round-mode=places]{30.96375653}{\si{\degree}} +k\ang{180}\\
 \intertext{Calcoliamo $x$}
 3x=&\SI[round-precision=\lungarrotandamento,round-mode=places]{30.96375653}{\si{\degree}} +k\ang{180}\\
 x=&\SI[round-precision=\lungarrotandamento,round-mode=places]{10.32125218}{\si{\degree}} +k\ang{60}\\
 \end{align*}
 \end{exercise}
 \begin{exercise}[no solution]
 Trovare l'angolo in radianti per cui $\tan 10x=-\dfrac{7}{2}$
 \end{exercise}
 \begin{exercise}[no solution]
 Trovare l'angolo in gradi per cui $\cos 4x=\dfrac{\sqrt{3}}{2}$
 \end{exercise}
 \begin{exercise}[no solution]
 Trovare l'angolo in gradi per cui $\sin 4x=-\dfrac{1}{2}$
 \end{exercise}

 \begin{exercise}
 Trovare l'angolo in gradi per cui $\cos (3x+\ang{30;;})=\dfrac{1}{4}$
 \tcblower
 $\cos (3x+\ang{30;;})=\dfrac{1}{4}$
 
 Controllare che la calcolatrice sia impostata in gradi\index{Grado!test}\index{Coseno}.
 
 Basta verificare che 
 \testgradi
 
 In caso contrario modificare le impostazioni.
 
 Non resta che procedere con il calcolo.
 
 Le soluzioni sono 
 \[\begin{cases}
 x_1=+\alpha+k\ang{360;;}\\
 x_2=-\alpha+k\ang{360;;}\\
 \end{cases}\]
 Calcolo $\alpha$
 \begin{center}
 \begin{tabular}{ll}
 \tastoicos\tasto{\num[round-precision=2,round-mode=places]{0.25}}
 \tastouguale&\num[round-precision=\lungarrotandamento,round-mode=places]{75.52248781} 
 \end{tabular} 
 \end{center}
 \[\alpha=\SI[round-precision=\lungarrotandamento,round-mode=places]{75.52248781}{\si{\degree}}\]
 \begin{align*}
 3x_1+\ang{30;;}&=\SI[round-precision=\lungarrotandamento,round-mode=places]{75.52248781}{\si{\degree}}+k\ang{360;;}\\
 3x_1&=\SI[round-precision=\lungarrotandamento,round-mode=places]{75.52248781}{\si{\degree}}-\ang{30;;}+k\ang{360;;}\\
 3x_1&=\SI[round-precision=\lungarrotandamento,round-mode=places]{45.52248781}{\si{\degree}}+k\ang{360;;}\\
 x_1&=\SI[round-precision=\lungarrotandamento,round-mode=places]{15.1741626}{\si{\degree}}+k\ang{120;;}\\
 \end{align*}
 \begin{align*}
 3x_2+\ang{30;;}&=-\SI[round-precision=\lungarrotandamento,round-mode=places]{75.52248781}{\si{\degree}}+k\ang{360;;}\\
 3x_2&=-\SI[round-precision=\lungarrotandamento,round-mode=places]{75.52248781}{\si{\degree}}-\ang{30;;}+k\ang{360;;}\\
 3x_1&=\SI[round-precision=\lungarrotandamento,round-mode=places]{-105.5224878}{\si{\degree}}+k\ang{360;;}\\
 x_2&=\SI[round-precision=\lungarrotandamento,round-mode=places]{-35.1741626}{\si{\degree}}+k\ang{180;;}\\
 \end{align*}
 
 Le soluzioni sono
 
 \[\begin{cases}
x_1=\SI[round-precision=\lungarrotandamento,round-mode=places]{15.1741626}{\si{\degree}}+k\ang{120;;}\\
x_2=\SI[round-precision=\lungarrotandamento,round-mode=places]{-35.1741626}{\si{\degree}}+k\ang{120;;}\\
 \end{cases}\]
 \end{exercise}
 \begin{exercise}
 	Trovare l'angolo in gradi per cui $\sin (3x+\ang{45;;})=\dfrac{3}{5}$
 	\tcblower
 	$\sin (3x+\ang{45;;})=\dfrac{3}{5}$
 	
 	Controllare che la calcolatrice sia impostata in gradi\index{Grado!test}\index{Grado!test}\index{Seno}.
 	
 	Basta verificare che 
 	\testgradi
 	
 	In caso contrario modificare le impostazioni.
 	
 	Non resta che procedere con il calcolo.
 	
 	Le soluzioni sono 
 	\[\begin{cases}
 	x_1=+\alpha+k\ang{360;;}\\
 	x_2=\ang{180;;}-\alpha+k\ang{360;;}\\
 	\end{cases}\]
 	Calcolo $\alpha$
 	\begin{center}
 		\begin{tabular}{ll}
 			\tastoisin\tasto{\num[round-precision=1,round-mode=places]{0.6}}
 			\tastouguale&\num[round-precision=\lungarrotandamento,round-mode=places]{36.86989765} 
 		\end{tabular} 
 	\end{center}
 	\[\alpha=\SI[round-precision=\lungarrotandamento,round-mode=places]{36.86989765}{\si{\degree}}\]
 	\begin{align*}
 		3x_1+\ang{45;;}&=\SI[round-precision=\lungarrotandamento,round-mode=places]{36.86989765}{\si{\degree}}+k\ang{360;;}\\
 		3x_1&=\SI[round-precision=\lungarrotandamento,round-mode=places]{36.86989765}{\si{\degree}}-\ang{45;;}+k\ang{360;;}\\
 		3x_1&=-\SI[round-precision=\lungarrotandamento,round-mode=places]{8.130102354}{\si{\degree}}+k\ang{360;;}\\
 		x_1&=-\SI[round-precision=\lungarrotandamento,round-mode=places]{2.710034118}{\si{\degree}}+k\ang{120;;}\\
 	\end{align*}
 	\begin{align*}
 		3x_2+\ang{45;;}&=\ang{180;;}-\SI[round-precision=\lungarrotandamento,round-mode=places]{36.86989765}{\si{\degree}}+k\ang{360;;}\\
 		3x_2+\ang{45;;}&=\SI[round-precision=\lungarrotandamento,round-mode=places]{143.1301024}{\si{\degree}}+k\ang{360;;}\\
 		3x_2&=\SI[round-precision=\lungarrotandamento,round-mode=places]{143.1301024}{\si{\degree}}-\ang{45;;}+k\ang{360;;}\\
 		3x_2&=\SI[round-precision=\lungarrotandamento,round-mode=places]{98.13010235}{\si{\degree}}+k\ang{360;;}\\
 		x_2&=\SI[round-precision=\lungarrotandamento,round-mode=places]{32.71003412}{\si{\degree}}+k\ang{120;;}\\
 	\end{align*}
 	
 	Le soluzioni sono
 	
 	\[\begin{cases}
x_1=-\SI[round-precision=\lungarrotandamento,round-mode=places]{2.710034118}{\si{\degree}}+k\ang{120;;}\\
x_2=\SI[round-precision=\lungarrotandamento,round-mode=places]{32.71003412}{\si{\degree}}+k\ang{120;;}\\
 	\end{cases}\]
 \end{exercise}
 \begin{exercise}
 	Trovare l'angolo in gradi per cui $\tan (5x-\ang{70;;})=5$
 	\tcblower
 	$\tan (5x-\ang{70;;})=5$
 	
 	Controllare che la calcolatrice sia impostata in gradi\index{Grado!test}\index{Grado!test}\index{Tangente}.
 	
 	Basta verificare che 
 	\testgradi
 	
 	In caso contrario modificare le impostazioni.
 	
 	Non resta che procedere con il calcolo.
 	
 	Le soluzioni sono 
 	\[x_1=+\alpha+k\ang{180;;}\]
 	Calcolo $\alpha$
 	\begin{center}
 		\begin{tabular}{ll}
 			\tastoitan\tasto{\num[round-precision=1,round-mode=places]{5}}
 			\tastouguale&\num[round-precision=\lungarrotandamento,round-mode=places]{76.69006753} 
 		\end{tabular} 
 	\end{center}
 	\[\alpha=\SI[round-precision=\lungarrotandamento,round-mode=places]{78.69006753}{\si{\degree}}\]
 	\begin{align*}
 	5x_1-\ang{70;;}&=\SI[round-precision=\lungarrotandamento,round-mode=places]{78.69006753}{\si{\degree}}+k\ang{180;;}\\
 	5x_1&=\SI[round-precision=\lungarrotandamento,round-mode=places]{78.69006753}{\si{\degree}}+\ang{70;;}+k\ang{180;;}\\
 	5x_1&=\SI[round-precision=\lungarrotandamento,round-mode=places]{148.6900675}{\si{\degree}}+k\ang{180;;}\\
 	x_1&=\SI[round-precision=\lungarrotandamento,round-mode=places]{29.73801351}{\si{\degree}}+k\ang{36;;}
 	\end{align*}
 	
 	La soluzione è
 \[x_1=\SI[round-precision=\lungarrotandamento,round-mode=places]{29.73801351}{\si{\degree}}+k\ang{36;;}\]
 \end{exercise}
 \begin{exercise}
 	Trovare l'angolo in radianti per cui $\cos (4x+\dfrac{4}{9}\pi)=-\dfrac{15}{16}$
 	\tcblower
 $\cos (4x-\dfrac{4}{9}\pi)=-\dfrac{15}{16}=-\num[round-precision=4,round-mode=places]{0.9375}$
 	
 	Controllare che la calcolatrice sia impostata in radianti\index{Radianti!test}\index{Coseno}.
 	
 	Basta verificare che 
 	\testradianti
 	
 	In caso contrario modificare le impostazioni.
 	
 	Non resta che procedere con il calcolo.
 	
 	Le soluzioni sono 
 	\[\begin{cases}
 	x_1=+\alpha+2k\pi\\
 	x_2=-\alpha+2k\pi\\
 	\end{cases}\]
 	Calcolo $\alpha$
 	\begin{center}
 		\begin{tabular}{ll}
 			\tastoicos\tasto{-\num[round-precision=4,round-mode=places]{0.9375}}
 			\tastouguale&\num[round-precision=\lungarrotandamento,round-mode=places]{2.786171452} 
 		\end{tabular} 
 	\end{center}
 	\[\alpha=\SI[round-precision=\lungarrotandamento,round-mode=places]{2.786171452}{\radian}\]
 	\begin{align*}
 	4x_1-\dfrac{4}{9}\pi&=\SI[round-precision=\lungarrotandamento,round-mode=places]{2.786171452}{\radian}+2k\pi\\
 	4x_1&=\SI[round-precision=\lungarrotandamento,round-mode=places]{2.786171452}{\radian}+\dfrac{4}{9}\pi+2k\pi\\
 	4x_1&=\SI[round-precision=\lungarrotandamento,round-mode=places]{2.786171452}{\radian}+\SI[round-precision=\lungarrotandamento,round-mode=places]{1.396263402}{\radian}+2k\pi\\
 	4x_1&=\SI[round-precision=\lungarrotandamento,round-mode=places]{4.182408602}{\radian}+2k\pi\\
 	x_1&=\SI[round-precision=\lungarrotandamento,round-mode=places]{1.045602151}{\radian}+\dfrac{1}{2}k\pi\\
 	\end{align*}
\begin{align*}
4x_2-\dfrac{4}{9}\pi&=-\SI[round-precision=\lungarrotandamento,round-mode=places]{2.786171452}{\radian}+2k\pi\\
4x_1&=-\SI[round-precision=\lungarrotandamento,round-mode=places]{2.786171452}{\radian}+\dfrac{4}{9}\pi+2k\pi\\
4x_2&=-\SI[round-precision=\lungarrotandamento,round-mode=places]{2.786171452}{\radian}+\SI[round-precision=\lungarrotandamento,round-mode=places]{1.396263402}{\radian}+2k\pi\\
4x_2&=-\SI[round-precision=\lungarrotandamento,round-mode=places]{1.38990805}{\radian}+2k\pi\\
x_2&=-\SI[round-precision=\lungarrotandamento,round-mode=places]{0.347477012}{\radian}+\dfrac{1}{2}k\pi\\
\end{align*}
 	
 	Le soluzioni sono
 	
 	\[\begin{cases}
x_1=\SI[round-precision=\lungarrotandamento,round-mode=places]{1.045602151}{\radian}+\dfrac{1}{2}k\pi\\
x_2=-\SI[round-precision=\lungarrotandamento,round-mode=places]{0.347477012}{\radian}+\dfrac{1}{2}k\pi\\
 	\end{cases}\]
 \end{exercise}
 \begin{exercise}[no solution]
 	Trovare l'angolo in gradi per cui $\tan (3x+\ang{50;;})=\dfrac{\sqrt{3}}{2}$
 \end{exercise}
 \begin{exercise}[no solution]
 	Trovare l'angolo in gradi per cui $\cos (7x+\ang{40;;})-=\dfrac{3}{5}$
 \end{exercise}
 \begin{exercise}[no solution]
 	Trovare l'angolo in radianti per cui $\sin (3x+\dfrac{3}{5}\pi)=-\dfrac{\sqrt{3}}{2}$
 \end{exercise}
\begin{exercise}
	Trovare l'angolo in gradi per cui $\tan (5x-\ang{70;;})=5$
	\tcblower
	$\tan (5x-\ang{70;;})=-5$
	
	Controllare che la calcolatrice sia impostata in gradi\index{Grado!test}\index{Grado!test}\index{Tangente}.
	
	Basta verificare che 
	\testgradi
	
	In caso contrario modificare le impostazioni.
	
	Non resta che procedere con il calcolo.
	
	Le soluzioni sono 
	\[x_1=+\alpha+k\ang{180;;}\]
	Calcolo $\alpha$
	\begin{center}
		\begin{tabular}{ll}
			\tastoitan\tasto{\num[round-precision=1,round-mode=places]{5}}
			\tastouguale&\num[round-precision=\lungarrotandamento,round-mode=places]{-78.69006753} 
		\end{tabular} 
	\end{center}
	\[\alpha=-\SI[round-precision=\lungarrotandamento,round-mode=places]{78.69006753}{\si{\degree}}\]
	\begin{align*}
	5x_1-\ang{70;;}&=-\SI[round-precision=\lungarrotandamento,round-mode=places]{76.69006753}{\si{\degree}}+k\ang{180;;}\\
	5x_1&=-\SI[round-precision=\lungarrotandamento,round-mode=places]{78.69006753}{\si{\degree}}+\ang{70;;}+k\ang{180;;}\\
	5x_1&=-\SI[round-precision=\lungarrotandamento,round-mode=places]{8.690067526}{\si{\degree}}+k\ang{180;;}\\
	x_1&=-\SI[round-precision=\lungarrotandamento,round-mode=places]{1.738013505}{\si{\degree}}+k\ang{36;;}
	\end{align*}
	
	La soluzione è
	\[x_1=-\SI[round-precision=\lungarrotandamento,round-mode=places]{1.738013505}{\si{\degree}}+k\ang{36;;}\]
\end{exercise}
\begin{exercise}
	Trovare i valori dell'incognita per cui $\cos(3x+\ang{50;;})=\cos(2x-\ang{30;;})$
	\tcblower
	$\cos3x+\ang{50;;})=\cos(2x-\ang{30;;})$
	
	Le soluzioni sono 
	\[\begin{cases}
	x_1=+\alpha+k\ang{360;;}\\
	x_2=-\alpha+k\ang{360;;}\\
	\end{cases}\]
	Troviamo la prima
	
	\begin{align*}
	3x+\ang{50;;}&=2x-\ang{30;;}+k\ang{360;;}\\
	3x-2x&=-\ang{50;;}-\ang{30;;}+k\ang{360;;}\\
	x&=-\ang{80;;}+k\ang{360;;}
	\end{align*}
 Troviamo la seconda
	
	\begin{align*}
	3x+\ang{50;;}&=-2x+\ang{30;;}+k\ang{360;;}\\
	3x+2x&=-\ang{50;;}+\ang{30;;}+k\ang{360;;}\\
	5x&=-\ang{20;;}+k\ang{360;;}\\
 x&=-\ang{4;;}+k\ang{72;;}
	\end{align*}
	
	Le soluzioni sono
	
	\[\begin{cases}
	x_1=-\ang{80;;}+k\ang{360;;}\\
	x_2=-\ang{4;;}+k\ang{72;;}\\
	\end{cases}\]
\end{exercise}
\begin{exercise}
	Trovare i valori dell'incognita per cui $\sin(6x-\dfrac{\pi}{3})=\sin(4x +\dfrac{2}{3}\pi)$
	\tcblower
$\sin(6x-\dfrac{\pi}{3})=\sin(4x +\dfrac{2}{3}\pi)$
	
	Le soluzioni sono 
	\[\begin{cases}
	x_1=+\alpha+2k\pi\\
	x_2=\pi-\alpha+2k\pi\\
	\end{cases}\]
	Troviamo la prima
	
	\begin{align*}
	6x-\dfrac{\pi}{3}&=4x +\dfrac{2}{3}\pi+2k\pi\\
6x-4x&=\dfrac{\pi}{3} +\dfrac{2}{3}\pi+2k\pi\\
	2x&=\pi+2k\pi\\
	x&=\dfrac{\pi}{2}+k\pi
	\end{align*}
	Troviamo la seconda
	
	\begin{align*}
		6x-\dfrac{\pi}{3}&=\pi-4x-\dfrac{2}{3}\pi+2k\pi\\
		6x+4x&=\pi+\dfrac{\pi}{3}-\dfrac{2}{3}\pi+2k\pi\\
		10x&=\dfrac{2}{3}\pi+2k\pi\\
		x&=\dfrac{2}{30}\pi+k\dfrac{1}{5}\pi
	\end{align*}
	
	Le soluzioni sono
	
	\[\begin{cases}
		x=\dfrac{\pi}{2}+k\pi\\
		\\
	x=\dfrac{2}{30}\pi+k\dfrac{1}{5}\pi
	\end{cases}\]
\end{exercise}
\begin{exercise}
	Trovare i valori dell'incognita per cui $\tan(5x-\ang{20;;})=\tan(2x+\ang{30;;})$
	\tcblower
	$\tan(5x-\ang{20;;})=\tan(2x+\ang{30;;})$
	
	Esistenza lato sinistro
	
		\begin{align*}
		5x-\ang{20;;}&=\ang{90;;}+k\ang{180;;}\\
		5x&=\ang{90;;}+\ang{20;;}+k\ang{180;;}\\
		5x&=\ang{110;;}+k\ang{180;;}\\
	 x&=\ang{22;;}+k\ang{36;;}
		\end{align*}
	
	Esistenza lato destro
	
	\begin{align*}
	2x+\ang{30;;}&=\ang{90;;}+k\ang{180;;}\\
	2x&=\ang{90;;}-\ang{30;;}+k\ang{180;;}\\
	2x&=\ang{60;;}+k\ang{180;;}\\
	x&=\ang{30;;}+k\ang{90;;}
	\end{align*}
	
	
	\begin{align*}
	5x-\ang{20;;}&=2x+\ang{30;;}+k\ang{180;;}\\
	3x&=\ang{20;;}+\ang{30;;}+k\ang{180;;}\\
	3x&=\ang{50;;}+k\ang{180;;}\\
	x =\dfrac{\ang{50;;}}{3}+k\ang{60;;}
	\end{align*}
	
	La soluzione è
	
\[x =\dfrac{\ang{50;;}}{3}+k\ang{60;;}\]
\end{exercise}
\begin{exercise}
	Trovare i valori dell'incognita per cui $\cos(3x+\ang{50;;})=-\cos(2x-\ang{40;;})$
	\tcblower
$\cos(3x+\ang{50;;})=-\cos(2x-\ang{40;;})=\cos(\ang{180;;}-2x+\ang{40;;})$
	
	Le soluzioni sono 
	\[\begin{cases}
	x_1=+\alpha+k\ang{360;;}\\
	x_2=-\alpha+k\ang{360;;}\\
	\end{cases}\]
	Troviamo la prima
	
	\begin{align*}
	3x+\ang{50;;}&=\ang{180;;}-2x+\ang{40;;}+k\ang{360;;}\\
	3x+2x&=\ang{180;;}-\ang{50;;}+\ang{40;;}+k\ang{360;;}\\
	5x&=\ang{170;;}+k\ang{360;;}\\
	x&=\ang{34;;}+k\ang{72;;}
	\end{align*}
	Troviamo la seconda
	
	\begin{align*}
		3x+\ang{50;;}&=-\ang{180;;}+2x-\ang{40;;}+k\ang{360;;}\\
		3x-2x&=-\ang{180;;}-\ang{50;;}-\ang{40;;}+k\ang{360;;}\\
		x&=-\ang{270;;}+k\ang{360;;}
	\end{align*}
	
	Le soluzioni sono
	
	\[\begin{cases}
	x_1=\ang{34;;}+k\ang{72;;}\\
	x_2-\ang{270;;}+k\ang{360;;}
	\end{cases}\]
\end{exercise}

\begin{exercise}
	Trovare i valori dell'incognita per cui $\sin(5x-\ang{70;;})=-\sin(7x +\ang{80;;})$
	\tcblower
$\sin(5x-\ang{70;;})=-\sin(7x +\ang{80;;})=\sin(-7x-\ang{80;;})$
	
	Le soluzioni sono 
	\[\begin{cases}
	x_1=\alpha+k\ang{360;;}\\
	x_2=\ang{180;;}-\alpha+k\ang{360;;}\\
	\end{cases}\]
	Troviamo la prima
	
	\begin{align*}
		5x-\ang{70;;}&=-7x-\ang{80;;}+k\ang{360;;}\\
		5x+7x&=\ang{70;;}-\ang{80;;}+k\ang{360;;}\\
		12x&=-\ang{10;;}+k\ang{360;;}\\
		x&=-\dfrac{\ang{10;;}}{12}+k\ang{30;;}
	\end{align*}
	Troviamo la seconda
	
	\begin{align*}
		5x-\ang{70;;}&=\ang{180;;}+7x+\ang{80;;}+k\ang{360;;}\\
		5x-7x&=\ang{180;;}+\ang{70;;}+\ang{80;;}+k\ang{360;;}\\
		-2x&=\ang{330;;}+k\ang{360;;}\\
		x&=-\ang{165;;}+k\ang{180;;}
	\end{align*}
	
	Le soluzioni sono
	
	\[\begin{cases}
	x_1=-\dfrac{\ang{10;;}}{12}+k\ang{30;;}\\
	x_2=-\ang{165;;}+k\ang{180;;}
	\end{cases}\]
\end{exercise}
\begin{exercise}
	Trovare i valori dell'incognita per cui $\tan(6x-\ang{30;;})=-\tan(2x-\ang{40;;})$
	\tcblower
	 $\tan(6x-\ang{30;;})=-\tan(2x-\ang{40;;})=\tan(-2x+\ang{40;;})$
	
	Esistenza lato sinistro
	
	\begin{align*}
		6x-\ang{30;;}&=\ang{90;;}+k\ang{180;;}\\
		6x&=\ang{90;;}+\ang{30;;}+k\ang{180;;}\\
		6x&=\ang{120;;}+k\ang{180;;}\\
		x&=\ang{20;;}+k\ang{36;;}
		\end{align*}
	
	Esistenza lato destro
	
	\begin{align*}
		-2x+\ang{40;;}&=\ang{90;;}+k\ang{180;;}\\
		-2x&=\ang{90;;}-\ang{40;;}+k\ang{180;;}\\
		-2x&=\ang{50;;}+k\ang{180;;}\\
		x&=-\ang{25;;}+k\ang{90;;}
		\end{align*}
	Troviamo la soluzione
	
		\begin{align*}
		6x-\ang{30;;}&=-2x+\ang{40;;}+k\ang{180;;}\\
		6x+2x&=\ang{40;;}+\ang{30;;}+k\ang{180;;}\\
		8x&=\ang{70;;}+k\ang{180;;}\\
		x&=\dfrac{\ang{70;;}}{8}+k\dfrac{\ang{180;;}}{8}
		\end{align*}
	
	La soluzione è
	\[x=\dfrac{\ang{70;;}}{8}+k\dfrac{\ang{180;;}}{8}\]
\end{exercise}
\begin{exercise}
	Trovare i valori dell'incognita per cui $\sin(3x+\ang{20})=-\sin(x+\ang{10;;})$
	\tcblower
	$\sin(3x+\ang{20})=-\sin(x+\ang{10;;})$
	
Modifichiamo il lato destro. Angoli associati
\begin{align*}
-\sin(\alpha)=&\sin(-\alpha)\\
\intertext{Diviene}
\sin(3x+\ang{20})=&\sin(-x-\ang{10})
\end{align*}	

	Troviamo la soluzione
	
	\begin{align*}
3x+\ang{20}=&-x-\ang{10}+k\ang{360}\\
3x+x=&-\ang{20}-\ang{10}+k\ang{360}\\
4x=&-\ang{30}+k\ang{360}\\
x=&-\frac{\ang{30}}{4}+k\frac{\ang{360}}{4}\\
x=&\ang{7.5}+k\ang{90}\\
	\end{align*}
	
	La soluzione è
	\[x=\ang{7.5}+k\ang{90}\]
	
La soluzione associata è:
\begin{align*}
3x+\ang{20}=&\ang{180}-(-x-\ang{10})+k\ang{360}\\
3x-x=&- \ang{20}+\ang{10}+\ang{180}+k\ang{360}\\
2x=&\ang{170}+k\ang{360}\\
x=&\frac{\ang{170}}{2}+k\frac{\ang{360}}{2}\\
x=&-\ang{85}+k\ang{180}\\
\end{align*}

La soluzione è
\[x=-\ang{85}+k\ang{180}\]
\end{exercise}
\begin{exercise}
	Trovare i valori dell'incognita per cui $\sin(3x+\ang{20})=\cos(x+\ang{10})$
	\tcblower
	$\sin(3x+\ang{20})=\cos(x+\ang{10})$
	
	Modifichiamo il lato destro. 
	
	Angoli complementari. 
	\begin{align*}
	\cos\alpha=&\sin(\ang{90}-\alpha)\\
	\intertext{Diviene}
	\sin(3x+\ang{20})=&\sin(\ang{90}-(x+\ang{10}))
	\end{align*}	
	
	Troviamo la soluzione
	
	\begin{align*}
	3x+\ang{20}=&\ang{90}-x-\ang{10}+k\ang{360}\\
	3x+x=&\ang{90}-\ang{20}-\ang{10}+k\ang{360}\\
	4x=&\ang{60}+k\ang{360}\\
	x=&\frac{\ang{60}}{4}+k\frac{\ang{360}}{4}\\
	x=&\ang{15}+k\ang{90}\\
	\end{align*}
	
	La soluzione è
	\[x=\ang{15}+k\ang{90}\]
	
	La soluzione associata è:
	\begin{align*}
	3x+\ang{20}=&\ang{180}-(\ang{90}-(x+\ang{10}))+k\ang{360}\\
	3x+\ang{20}=&\ang{180}-\ang{90}+x+\ang{10}+k\ang{360}\\
	3x-x=&\ang{80}+k\ang{360}\\
	2x=&\ang{80}+k\ang{360}\\
	x=&\frac{\ang{80}}{2}+k\frac{\ang{360}}{2}\\
	x=&\ang{40}+k\ang{180}\\
	\end{align*}
	
	La soluzione è
	\[x=\ang{40}+k\ang{180}\]
\end{exercise}
\begin{exercise}
	Trovare i valori dell'incognita per cui $\sin(4x+\ang{60})=-\cos(2x+\ang{30})$
	\tcblower
$\sin(4x+\ang{60})=-\cos(2x+\ang{30})$
	
	Modifichiamo il lato destro.
	
	Angoli associati 
	\begin{align*}
	\cos(\ang{180}-\alpha)=&-\cos\alpha\\
	\intertext{Diviene}
	\sin(4x+\ang{60})=&\cos(\ang{180}-(2x+\ang{30}))
	\end{align*}	
	
	Angoli complementari
	
	\begin{align*}
	\cos\alpha=&\sin(\ang{90}-\alpha)\\
	\intertext{Diviene}
	\sin(4x+\ang{60})=&\sin(\ang{90}-(\ang{180}-(2x+\ang{30})))
	\end{align*}	
	
	Troviamo la soluzione
	
	\begin{align*}
	4x+\ang{60}=&\ang{90}-(\ang{180}-(2x+\ang{30}))+k\ang{360}\\
	4x+\ang{60}=&\ang{90}-\ang{180}+2x+\ang{30}+k\ang{360}\\
	4x-2x=&\ang{90}-\ang{180}+\ang{30}+k\ang{360}\\
	2x=&-\ang{120}+k\ang{360}\\
	x=&-\frac{\ang{120}}{2}+k\frac{\ang{360}}{2}\\
	x=&-\ang{60}+k\ang{180}\\
	\end{align*}
	
	La soluzione è
	\[x=-\ang{60}+k\ang{180}\]
	
	La soluzione associata è:
	\begin{align*}
	4x+\ang{60}=&\ang{180}-(\ang{90}-(\ang{180}-(2x+\ang{30})))+k\ang{360}\\
	4x+\ang{60}=&\ang{180}-\ang{90}+(\ang{180}-(2x+\ang{30}))+k\ang{360}\\
	4x+\ang{60}=&\ang{180}-\ang{90}+\ang{180}-2x-\ang{30}+k\ang{360}\\
	4x+2x=&\ang{180}-\ang{90}+\ang{180}-\ang{60}-\ang{30}+k\ang{360}\\
	6x=&\ang{180}-\ang{90}+\ang{180}-\ang{60}-\ang{30}+k\ang{360}\\
	6x=&\ang{180}+k\ang{360}\\
	x=&\frac{\ang{180}}{6}+k\frac{\ang{360}}{6}\\
	x=&\ang{30}+k\ang{90}\\
	\end{align*}
	
	La soluzione è
	\[x=\ang{30}+k\ang{90}\]
\end{exercise}
 \tcbstoprecording
 \newpage
 \section{Soluzioni equazioni goniometriche}
 \tcbinputrecords
\tcbstartrecording
\chapter{Triangoli}
\label{cha:trigonometriatriangoli}
 \section{Triangoli rettangoli}
 
\begin{figure}
	\centering
	\includestandalone[width=.6\textwidth]{terzo/grafici/triangolopitagorico1}
	\caption{Triangolo rettangolo}
	\label{fig:triangolorettangolo}
\end{figure}

Vista la figura\nobs\vref{fig:triangolorettangolo} risolvere i seguenti esercizi
% \tcbstartrecording
 \begin{exercise}
 	Dato un triangolo rettangolo con
 	\begin{align*}
 	a=&5\\
 	\beta=&\ang{30}
 	\end{align*}
\tcblower
Controllare che la calcolatrice sia impostata in gradi sessagesimali\index{Grado!Sessagesimale}.
Basta verificare che \testgradi 
\begin{align*}
a=&5\\
\beta=&\ang{30}\\
\gamma=&\ang{90}-\ang{30}=\ang{60}\\
c=&a\sin\gamma\\
=& 5\sin\ang{60}=\num[round-precision=\lungarrotandamento,round-mode=places]{4.330}\\
b=&a\sin\beta\\
=& 5\sin\ang{30}=\num[round-precision=1,round-mode=places]{2.5}
\end{align*}
 \end{exercise}
 
  \begin{exercise}
  	Dato un triangolo rettangolo con
  	\begin{align*}
  	a=&5\\
  	\gamma=&\ang{35}
  	\end{align*}
  	\tcblower
  	Controllare che la calcolatrice sia impostata in gradi sessagesimali\index{Grado!Sessagesimale}.
  	Basta verificare che \testgradi 
  	\begin{align*}
  	a=& 7\\
  	\gamma=&\ang{35}\\
  	\beta=&\ang{90}-\ang{35}=\ang{55}\\
  	b=&a\cos\gamma\\
  	=& 7\cos\ang{55}=\num[round-precision=\lungarrotandamento,round-mode=places]{5.734064}\\
  	b=&a\sin\beta\\
  	=& 7\sin\ang{35}=\num[round-precision=\lungarrotandamento,round-mode=places]{4.015035054}
  	\end{align*}
  \end{exercise}
\begin{exercise}
	Dato un triangolo rettangolo con
	\begin{align*}
	b=&5\\
	c=&8
	\end{align*}
	\tcblower
	Controllare che la calcolatrice sia impostata in gradi sessagesimali\index{Grado!Sessagesimale}.
	Basta verificare che \testgradi 
	\begin{align*}
	a=&\sqrt{8^2+5^2}\\
	=\sqrt{89}\\
	b=&a\sin\beta\\
	\sqrt{89}\sin\beta=&5\\
	\sin\beta=&\frac{5}{\sqrt{89}}\\
	\beta=&\arcsin\frac{5}{\sqrt{89}}\\
	&\tasto{5}\tastodiv\tastoradicequadrata\tasto{89}\tastouguale\\
	&\tastoisin\tastoans\tastouguale\\
	=&\num[round-precision=\lungarrotandamento,round-mode=places]{32.00538321}\\
	c=&a\sin\gamma\\
\sqrt{89}\sin\gamma=&8\\
\sin\gamma=&\frac{8}{\sqrt{89}}\\
\gamma=&\arcsin\frac{8}{\sqrt{89}}\\
&\tasto{8}\tastodiv\tastoradicequadrata\tasto{89}\tastouguale\\
&\tastoisin\tastoans\tastouguale\\
&=\num[round-precision=\lungarrotandamento,round-mode=places]{57.999461679}\\
	\end{align*}
\end{exercise}
\begin{exercise}
	Dato un triangolo rettangolo con
	\begin{align*}
	b=&7\\
	\beta=&\ang{30}
	\end{align*}
	\tcblower
	Controllare che la calcolatrice sia impostata in gradi sessagesimali\index{Grado!Sessagesimale}.
	Basta verificare che \testgradi 
	\begin{align*}
	\gamma=&\ang{90}-\ang{30}\\
	c=&b\tan\gamma\\
	c=&7\tan\ang{60}\\
	&\tasto{7}\tastodiv\tastoradicequadrata\tasto{89}\tastouguale\\
	&\tastoisin\tastoans\tastouguale\\
	=&\num[round-precision=\lungarrotandamento,round-mode=places]{32.00538321}\\
	c=&a\sin\gamma\\
	\sqrt{89}\sin\gamma=&8\\
	\sin\gamma=&\frac{8}{\sqrt{89}}\\
	\gamma=&\arcsin\frac{8}{\sqrt{89}}\\
	&\tasto{8}\tastodiv\tastoradicequadrata\tasto{89}\tastouguale\\
	&\tastoisin\tastoans\tastouguale\\
	&=\num[round-precision=\lungarrotandamento,round-mode=places]{57.999461679}\\
	\end{align*}
\end{exercise}
\tcbstoprecording
 \newpage
 \section{Soluzioni triangoli rettangoli}
 \tcbinputrecords
\chapter{Soluzioni esercizi}
\tcbstoprecording
% \newpage
%\section{Soluzioni esercizi}
\tcbinputrecords
\backmatter
\begin{appendices}
	\input{../Mod_base/MezziUsati}
\end{appendices}
\addcontentsline{toc}{chapter}{\indexname}
\printindex
\end{document}
