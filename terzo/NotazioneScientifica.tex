\chapter{Notazione scientifica}
\label{cha:Notazionescietifica}
 
\section{Conversioni}


\begin{esempiot}{Da decimale a esponenziale}{}
	Convertire il numero \num{0.0035} in notazione scientifica
\end{esempiot}
La virgola va' spostata di tre posti verso destra per cui
\[\num{0.0035}=\num[scientific-notation=true]{0.0035}\]
\begin{esempiot}{Da decimale a esponenziale}{}
	Convertire il numero \num{7255.45} in notazione scientifica
\end{esempiot}
La virgola va' spostata di tre posti verso sinistra per cui
\[\num{7255.45}=\num[scientific-notation=true]{7255.45}\]
\tcbstartrecording
\begin{exercise}[no solution]
	Disegnare nel piano complesso il numero $z=1+2\uimm$
\end{exercise}
\begin{exercise}
	Convertire il numero \num{0.372} in notazione scientifica\index{Notazione!scientifica}
	\tcblower
	La virgola va' spostata di un posto verso destra per cui
	\[\num{0.372}=\num[scientific-notation=true]{0.372}\]
\end{exercise}
\tcbstoprecording
% \newpage
\section{Soluzioni esercizi Notazione scientifica}
\tcbinputrecords