\chapter{Funzioni goniometriche usando la calcolatrice}
\label{cha:ValFunzGonioCalc}
Quando si parla di calcolatrice ci si riferisce a una calcolatrice scientifica. Devono essere presenti i tasti:
\begin{center}
 \begin{tabular}{ccc}
\tastosin&\tastocos&\tastotan \\ 
\end{tabular} 
\end{center}
In genere per le funzioni inverse si usa una combinazione di tasti \tastoshift 
\begin{center}
 \begin{tabular}{ccc}
 \tastoisin&\tastoicos&\tastoitan \\ 
 \end{tabular} 
\end{center}
La calcolatrice deve permettere di gestire i radianti e gradi. La calcolatrice deve avere  il numero $\pi$ \tastopgreco presente. Un tasto molto utile è il tasto \tastoans\ (dall'inglese answer risposta) che richiama l'ultimo risultato ottenuto. In una calcolatrice è facile trovare tre unità di misura per gli angoli i radianti $RAD$, i gradi centesimali $GRAD$ e i gradi sessagesimali $DEG$. Dobbiamo sempre essere in grado di verificare come è impostata la calcolatrice in modo da ottenere risultati corretti.
\begin{table}
	\centering
	\begin{tabular}{lll}
\toprule
Unità di misura		& Sigla &Inglese\\ 
\midrule
Radianti		&RAD &Radians \\ 
Gradi centesimali		&GRAD &Gradian \\ 
Gradi Sessagesimali		&DEG &Degree \\ 
\bottomrule
	\end{tabular} 
	\caption{Calcolatrice angoli}\label{tab:calcolatrice_angoli}
\end{table}
\section{Trovare valore funzione}
\subsection{Angolo in gradi}
\begin{esempiot}{Trovare valore funzione}{}
Calcolare \[\sin\ang{38;28;50}\] 
\end{esempiot}
Controllare che la calcolatrice sia impostata in gradi sessagesimali\index{Grado!Sessagesimale}.
Basta verificare che \testgradi In caso contrario modificare le impostazioni. 

Si inizia convertendo i gradi in forma sessadecimale\index{Grado!Sessagesimale}\index{Grado!Sessadecimale}\index{Seno}

\begin{align*}
&\phantom{=}\ang{38}+\left(\dfrac{28}{60}\right)^{\si{\degree}}+\left(\dfrac{50}{3600}\right)^{\si{\degree} }=\\
=&\ang{38}+\left(\dfrac{28\cdot60+50}{3600}\right)^{\si{\degree}}=\\
=&\ang{38}+\left(\dfrac{1680+50}{3600}\right)^{\si{\degree}}=\\
=&\ang{38}+\left(\dfrac{1730}{3600}\right)^{\si{\degree}}=\\
=&\ang[round-precision=\lungarrotandamento,round-mode=places]{38.4805556}
\end{align*}
usando la calcolatrice

\begin{center}
\begin{tabular}{ll}
\tasto{28}\tastoper\tasto{60}\tastouguale& 1680 \\ 
\tastoans\tastopiu\tasto{50}\tastouguale& 1730 \\
\tastoans\tastodiv\tasto{3600}\tastouguale& \num[round-precision=\lungarrotandamento,round-mode=places]{0.480555555} \\
\tastoans\tastopiu\tasto{38}\tastouguale&\num[round-precision=\lungarrotandamento,round-mode=places]{38.480555555} \\
\end{tabular}
\end{center} 

Infine

 \tastosin\tastoans\tastouguale e ottenere
\[\sin\ang{38;28;50}=\num[round-precision=\lungarrotandamento,round-mode=places]{0.622249007}\] 

\begin{esempiot}{Trovare valore funzione}{}
 Calcolare \[\tan\ang{120;30;40}\] 
\end{esempiot}
Controllare che la calcolatrice sia impostata in gradi sessagesimali\index{Grado!Sessagesimale}.
Basta verificare che \testgradi. In caso contrario modificare le impostazioni. 

Si inizia convertendo i gradi in forma sessadecimale\index{Grado!Sessagesimale}\index{Grado!Sessadecimale}\index{Tangente}

\begin{align*}
&\phantom{=}\ang{120}+\left(\dfrac{30}{60}\right)^{\si{\degree}}+\left(\dfrac{40}{3600}\right)^{\si{\degree} }\\
=&\ang{120}+\left(\dfrac{30\cdot60+40}{3600}\right)^{\si{\degree}}\\
=&\ang{120}+\left(\dfrac{1800+40}{3600}\right)^{\si{\degree}}\\
=&\ang{120}+\left(\dfrac{1730}{3600}\right)^{\si{\degree}}\\
=&\ang[round-precision=\lungarrotandamento,round-mode=places]{120.5111111}
\end{align*}

usando la calcolatrice

\begin{center}
 \begin{tabular}{ll}
 \tasto{30}\tastoper\tasto{60}\tastouguale & 1800 \\ 
 \tastoans\tastopiu\tasto{40}\tastouguale & 1840 \\
 \tastoans\tastodiv\tasto{3600}\tastouguale & \num[round-precision=\lungarrotandamento,round-mode=places]{0.511111111} \\
 \tastoans\tastopiu\tasto{120}\tastouguale&\num[round-precision=\lungarrotandamento,round-mode=places]{120.511111111} \\
 \end{tabular}
\end{center} 

Infine \tastotan \tastoans\tastouguale e ottenere
\[\tan\ang{120;30;40}=\num[round-precision=\lungarrotandamento,round-mode=places]{-1.69610537}\] 
\subsection{Angolo in radianti}
\begin{esempiot}{Trovare valore funzione}{}
 Calcolare \[\cos\dfrac{\pi}{4}\] 
\end{esempiot}
Controllare che la calcolatrice sia impostata in radianti\index{Radianti}\index{Coseno}.
Basta verificare che 
\testradianti
 In caso contrario modificare le impostazioni.

Non resta che procedere con il calcolo
 
\begin{center}
\begin{tabular}{ll}
 \tastopgreco\tastodiv\tasto{4}\tastouguale& \num[round-precision=\lungarrotandamento,round-mode=places]{0.785398163} \\ 
\tastocos\tastoans\tastouguale &\num[round-precision=\lungarrotandamento,round-mode=places]{0.707106781} \\ 
\end{tabular} 
\end{center}
\[\cos\dfrac{\pi}{4}=\num[round-precision=\lungarrotandamento,round-mode=places]{0.707106781}\] 
\begin{esempiot}{Trovare valore funzione}{}
 Calcolare \[\tan\SI[round-precision=4,round-mode=places]{1.4589}{\radian}\] 
\end{esempiot}
Controllare che la calcolatrice sia impostata in radianti\index{Radianti}\index{Tangente}.
Basta verificare che 
\testradianti
In caso contrario modificare le impostazioni.

Non resta che procedere con il calcolo

\begin{center}
 \begin{tabular}{ll}
 \tastotan\tasto{\num[round-precision=4,round-mode=places]{1.4589}}\tastouguale& \num[round-precision=\lungarrotandamento,round-mode=places]{8.899513904}\\ 
 \end{tabular} 
\end{center}
otteniamo
 \[\tan\SI[round-precision=4,round-mode=places]{1.4589}{\radian}=\num[round-precision=\lungarrotandamento,round-mode=places]{8.899513904}\] 
 \section{Trovare l'angolo nota la funzione}
 \subsection{Angolo in gradi}
 \begin{esempiot}{Trovare valore funzione}{}
 Trovare l'angolo per cui \[\cos x=\num[round-precision=\lungarrotandamento,round-mode=places]{0.778934}\]
 \end{esempiot}
Controllare che la calcolatrice sia impostata in gradi sessagesimali\index{Grado!Sessagesimale}.
Basta verificare che \testgradi 

In caso contrario modificare le impostazioni.

Le soluzioni sono 
\[\begin{cases}
 x_1=+\alpha+k\ang{360}\\
 x_2=-\alpha+k\ang{360}\\
\end{cases}\]
Calcolo $\alpha$

\begin{center}
 \begin{tabular}{ll}
 \tastoicos\tasto{\num[round-precision=\lungarrotandamento,round-mode=places]{0.778934}}\tastouguale&\SI[round-precision=\lungarrotandamento,round-mode=places]{38.8369232}{\si{\degree}}\\
 \end{tabular}
\end{center}

Convertiamo in gradi sessagesimali

\begin{center} 
 \begin{tabular}{ll}
 \tastoans\tastomeno\tasto{38}\tastouguale&\SI[round-precision=\lungarrotandamento,round-mode=places]{0.810314895}{\si{\degree}}\\
 \tastoans\tastoper\tasto{60}\tastouguale&\SI[round-precision=\lungarrotandamento,round-mode=places]{50.21539175}{\arcminute}\\
 \tastoans\tastomeno\tasto{50}\tastouguale&\SI[round-precision=\lungarrotandamento,round-mode=places]{0.215391754}{\arcminute}\\
 \tastoans\tastoper\tasto{60}\tastouguale&\SI[round-precision=\lungarrotandamento,round-mode=places]{12.929350526}{\arcsecond}\\
 \end{tabular} 
\end{center}
\[\alpha=\ang{38;50;12}\]
le soluzioni sono quindi
\[\begin{cases}
x_1=+\ang{38;50;12}+k\ang{360}\\
x_2=-\ang{38;50;12}+k\ang{360}\\
\end{cases}\]
 \subsection{Angolo in radianti}
 \begin{esempiot}{Trovare valore funzione}{}
 Calcolare \[\sin\alpha=\num[round-precision=\lungarrotandamento,round-mode=places]{-0.783942}\] 
 \end{esempiot}
 Controllare che la calcolatrice sia impostata in radianti\index{Radianti}\index{Seno}.
 Basta verificare che 
 \testradianti
 In caso contrario modificare le impostazioni.
 
 Non resta che procedere con il calcolo
 
 \begin{center}
 \begin{tabular}{ll}
 \tastoisin\tasto{\num[round-precision=\lungarrotandamento,round-mode=places]{-0.783942}}\tastouguale&\num[round-precision=\lungarrotandamento,round-mode=places]{-0.900990129}\\ \tasto{2}\tastoper\tastopgreco\tastopiu\tastoans\tastouguale&\num[round-precision=\lungarrotandamento,round-mode=places]{5.382195178}\\
 \end{tabular} 
 \end{center}
 \[\rho= \SI[round-precision=\lungarrotandamento,round-mode=places]{-0.900990129}{\radian}\]
 \[\rho= \SI[round-precision=\lungarrotandamento,round-mode=places]{5.382195178}{\radian}\]