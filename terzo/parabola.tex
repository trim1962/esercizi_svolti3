\chapter{Parabola}
\label{cha:parabola}
\section{Disegnare una parabola}
\begin{esempiot}{Disegnare una parabola nota}{parabola1}
	Disegnare il grafico della parabola $y=4x^2-18x+18$
	\end{esempiot}
	La parabola ha $a>0$ quindi ha la concavità rivolta verso l'alto. La parabola ha l'asse parallelo all'asse $y$. L'asse ha quindi equazione\[x=-\dfrac{b}{2a}=-\dfrac{-18}{2\cdot 4}=\dfrac{18}{8}=\dfrac{9}{4}\]
	Per trovare l'ordinata del vertice $V$ sostituisco l'ascissa dell'asse nell'equazione della parabola
	\begin{align*}
y=&4x^2-18x+18\\
=&4\left( \dfrac{9}{4}\right)^2-18\dfrac{9}{4}+18\\
=&4\dfrac{81}{16}-18\dfrac{9}{4}+18\\
=&\dfrac{81}{4}-\dfrac{162}{4}+18\\
=&\dfrac{81-162+72}{4}\\
=&-\dfrac{9}{4}
	\end{align*}
	Il vertice ha quindi coordinate $V\coord{\dfrac{9}{4}}{-\dfrac{9}{4}}$
	
Passaggio per punti
\begin{align*}
x&=1\\
y=&4x^2-18x+18\\
=&4(1)^2-18\cdot1 +18\\
=&4
\end{align*}
\begin{align*}
x&=2\\
y=&4x^2-18x+18\\
=&4(2)^2-18\cdot2 +18\\
=&-2
\end{align*}

Ricapitolando
		\begin{tabular}{c|c}
			x & y\\
			\hline 
			1& 4 \\ 
			2&-2  \\ 
		\end{tabular}
			
$A\coord{1}{4}$ $B\coord{2}{-2}$ Per simmetria rispetto all'asse ottengo altri due punti.
\begin{center}
	\includestandalone[width=.5\textwidth]{terzo/grafici/parabola1}
	\captionof{figure}{Grafico parabola}\label{fig:disegnoparabola1}
\end{center}
\begin{esempiot}{Disegnare una parabola nota}{parabola2}
	Disegnare il grafico della parabola $y=-2x^2+8x-6$
\end{esempiot}

La parabola $a<0$ quindi ha la concavità rivolta verso il basso. Ha l'asse parallelo all'asse delle $y$. L'asse ha quindi equazione \[x=-\dfrac{b}{2a}=-\dfrac{8}{-4}=+2\]
Per trovare l'ordinata del vertice $V$ sostituisco l'ascissa dell'asse nell'equazione della parabola\[y=-2\cdot4+8\cdot2-6=2\]
Il vertice ha quindi coordinate $V\coord{2}{2}$
\begin{align*}
	x&=0\\
	y=&-2x^2+8x-6\\
	=&-2(0)^2+8\cdot0 -6\\
	=&-6
\end{align*}
\begin{align*}
	x&=1\\
	y=&-2x^2+8x-6\\
	=&-2(1)^2+8\cdot1 -6\\
	=&0
\end{align*}
Ricapitolando
\begin{tabular}{c|c}
	x & y\\
	\hline 
	0& -6 \\ 
	1&0  \\ 
\end{tabular}

$A\coord{0}{-6}$ $B\coord{1}{0}$ Per simmetria rispetto all'asse ottengo altri due punti.
\begin{center}
	\includestandalone[width=.5\textwidth]{terzo/grafici/parabola2}
	\captionof{figure}{Grafico parabola}\label{fig:disegnoparabola2}
\end{center}
\section{Elementi della parabola}	
\begin{esempiot}{Elementi di una parabola}{parabola3}
	Trovare asse, fuoco, vertice e direttrice della parabola $y=\dfrac{1}{4}x^2-x-2$
\end{esempiot}
$a>0$ la parabola ha concavità rivolta verso l'alto. I coefficienti sono\[\begin{cases}
a=\dfrac{1}{4}\\
b=-1\\
c=-2
\end{cases} \]
L'asse della parabola è\[x=-\dfrac{-b}{2a}=\dfrac{-1}{2\dfrac{-1}{4}}=2 \]
Per trovare le coordinate del fuoco $F\coord{-\dfrac{b}{2a}}{\dfrac{1-\Delta}{4a}}$, del vertice $V\coord{-\dfrac{b}{2a}}{-\dfrac{\Delta}{4a}}$ e l'equazione della direttrice $y=-\dfrac{1+\Delta}{4a}$, bisogna conoscere il valore di $\Delta$ e di $4a$. Inizio con il calcolare \[\Delta=b^2-4a=1-4\cdot\dfrac{1}{4}\cdot(-2)=3\]
\[4a=4\cdot\dfrac{1}{4}=1\]
Quindi \[\begin{cases}
\dfrac{1-\Delta}{4a}=\dfrac{1-3}{1}=-2\\
\\
-\dfrac{\Delta}{4a}=-\dfrac{3}{1}=-3\\
\\
-\dfrac{1+\Delta}{4a}=-\dfrac{1+3}{1}=-4
\end{cases} \]
Ricapitolando \[F\coord{2}{-2} \]\[V\coord{2}{-3} \]\[y=-4 \]
\begin{center}
	\includestandalone[width=.5\textwidth]{terzo/grafici/parabola3}
	\captionof{figure}{Elementi della parabola}\label{fig:disegnoparabola3}
\end{center}
\begin{esempiot}{Elementi di una parabola}{parabola3a}
	Trovare asse, fuoco, vertice e direttrice della parabola $y=-x^2+4$
\end{esempiot}
$a<0$ la parabola ha concavità rivolta verso il basso. I coefficienti sono\[\begin{cases}
a=-1\\
b=0\\
c=4
\end{cases} \]
L'asse della parabola è\[x=-\dfrac{-b}{2a}=\dfrac{0}{-2}=0 \]
Per trovare le coordinate del fuoco $F\coord{-\dfrac{b}{2a}}{\dfrac{1-\Delta}{4a}}$, del vertice $V\coord{-\dfrac{b}{2a}}{-\dfrac{\Delta}{4a}}$ e l'equazione della direttrice $y=-\dfrac{1+\Delta}{4a}$, bisogna conoscere il valore di $\Delta$ e di $4a$. Inizio con il calcolare \[\Delta=b^2-4a=0-4\cdot(-1)\cdot4=16\]
\[4a=4\cdot(-1)=-4\]
Quindi \[\begin{cases}
\dfrac{1-\Delta}{4a}=\dfrac{1-16}{-4}=\dfrac{-15}{-4}=\dfrac{15}{4}\\
\\
-\dfrac{\Delta}{4a}=-\dfrac{16}{-4}=\dfrac{16}{4}=4\\
\\
-\dfrac{1+\Delta}{4a}=-\dfrac{1+16}{-4}=\dfrac{17}{4}
\end{cases} \]
Ricapitolando \[F\coord{0}{\dfrac{15}{4}} \]\[V\coord{0}{3} \]\[y=\dfrac{17}{4} \]
\begin{center}
	\includestandalone[width=.5\textwidth]{terzo/grafici/parabola3a}
	\captionof{figure}{Elementi della parabola}\label{fig:disegnoparabola3a}
\end{center}
\begin{esempiot}{Elementi di una parabola}{parabola3b}
	Trovare asse, fuoco, vertice e direttrice della parabola $y=-\dfrac{3}{2}x^2+\dfrac{9}{2}$
\end{esempiot}
$a<0$ la parabola ha concavità rivolta verso il basso. I coefficienti sono\[\begin{cases}
a=-\dfrac{3}{2}\\
b=\dfrac{9}{2}\\
c=0
\end{cases} \]
L'asse della parabola è\[x=-\dfrac{-b}{2a}=-\dfrac{9}{2}\cdot\dfrac{1}{2(-\dfrac{3}{2})}=\dfrac{9}{2}\cdot\dfrac{1}{3}=\dfrac{3}{2} \]
Per trovare le coordinate del fuoco $F\coord{-\dfrac{b}{2a}}{\dfrac{1-\Delta}{4a}}$, del vertice $V\coord{-\dfrac{b}{2a}}{-\dfrac{\Delta}{4a}}$ e l'equazione della direttrice $y=-\dfrac{1+\Delta}{4a}$, bisogna conoscere il valore di $\Delta$ e di $4a$. Inizio con il calcolare \[\Delta=b^2-4a=\dfrac{81}{4}-4\cdot(-\dfrac{3}{2})\cdot0=\dfrac{81}{4}\]
\[4a=4\cdot\left(-\dfrac{3}{2}\right)=-6\]
Quindi \[\begin{cases}
\dfrac{1-\Delta}{4a}=\left(1-\dfrac{81}{4}\right)\left(-\dfrac{1}{6}\right)=\dfrac{4-81}{4}\left(-\dfrac{1}{6}\right)=\dfrac{77}{24}\\
\\
-\dfrac{\Delta}{4a}=-\dfrac{81}{4}\left(-\dfrac{1}{6}\right)=\dfrac{81}{24}=\dfrac{27}{8}\\
\\
-\dfrac{1+\Delta}{4a}=\left(1+\dfrac{81}{4}\right)\left(-\dfrac{1}{6}\right)=\dfrac{4+81}{4}\left(-\dfrac{1}{6}\right)=\dfrac{85}{24}\\
\end{cases} \]
Ricapitolando \[F\coord{0}{\dfrac{77}{4}} \]\[V\coord{0}{\dfrac{27}{8}} \]\[y=\dfrac{81}{24} \]
\begin{center}
	\includestandalone[width=.5\textwidth]{terzo/grafici/parabola3b}
	\captionof{figure}{Elementi della parabola}\label{fig:disegnoparabola3b}
\end{center}
\begin{esempiot}{Elementi di una parabola}{parabola3c}
	Trovare asse, fuoco, vertice e direttrice della parabola $y=\dfrac{1}{4}x^2$
\end{esempiot}
$a<0$ la parabola ha concavità rivolta verso il basso. I coefficienti sono\[\begin{cases}
a=\dfrac{1}{4}\\
b=0\\
c=0
\end{cases} \]
L'asse della parabola è\[x=-\dfrac{-b}{2a}=-\dfrac{0}{2\cdot\dfrac{1}{4}}=0 \]
Per trovare le coordinate del fuoco $F\coord{-\dfrac{b}{2a}}{\dfrac{1-\Delta}{4a}}$, del vertice $V\coord{-\dfrac{b}{2a}}{-\dfrac{\Delta}{4a}}$ e l'equazione della direttrice $y=-\dfrac{1+\Delta}{4a}$, bisogna conoscere il valore di $\Delta$ e di $4a$. Inizio con il calcolare \[\Delta=b^2-4a=0-0=0\]
\[4a=4\cdot\left(\dfrac{1}{4}\right)=1\]
Quindi \[\begin{cases}
\dfrac{1-\Delta}{4a}=\dfrac{1-0}{1}=1\\
\\
-\dfrac{\Delta}{4a}=-\dfrac{0}{1}=0\\
\\
-\dfrac{1+\Delta}{4a}=\dfrac{1+0}{1}=-1\\
\end{cases} \]
Ricapitolando \[F\coord{0}{1} \]\[V\coord{0}{0} \]\[y=-1 \]
\begin{center}
	\includestandalone[width=.5\textwidth]{terzo/grafici/parabola3c}
	\captionof{figure}{Elementi della parabola}\label{fig:disegnoparabola3c}
\end{center}
\section{Intersezioni}
\begin{esempiot}{Intersezione della parabola con gli assi}{parabola4}
	Trovare i punti di intersezione della parabola $y=3x^2+2x+5$ con gli assi.
\end{esempiot}
Intersezione asse $x$
\begin{align*}
	&\begin{cases}
		y=0\\y=3x^2+2x-5
	\end{cases}\\
	&3x^2+2x-5=0\\
	&x_{1.2}=\dfrac{-2\pm\sqrt{4+60}}{6}=\\
	&=\dfrac{-2\pm\sqrt{64}}{6}=\\
	&=\dfrac{-2\pm 8}{6}=\begin{cases}
		x_1=-\dfrac{10}{6}=-\dfrac{5}{3}\\
		x_2=\dfrac{6}{6}=1
	\end{cases}
\end{align*}
Intersezione asse $y$
\begin{align*}
&\begin{cases}
	x=0\\y=3x^2+2x-5
\end{cases}&\begin{cases}
x=0\\y=3\cdot0+2\cdot 0-5
\end{cases}\\&\begin{cases}
x=0\\y=-5
\end{cases}
\end{align*}
Quindi \[A\coord{-\dfrac{5}{3}}{0} \] \[B\coord{1}{0} \] \[C\coord{0}{-5} \]
\begin{center}
	\includestandalone[width=.5\textwidth]{terzo/grafici/parabola4}
	\captionof{figure}{Intersezioni con gli assi}\label{fig:disegnoparabola4}
\end{center}
\begin{esempiot}{Intersezione della parabola con gli assi}{parabola5}
	Trovare i punti di intersezione della parabola $y=x^2+2x+1$ con gli assi.
\end{esempiot}
Intersezione asse $x$
\begin{align*}
	&\begin{cases}
		y=0\\y=x^2+2x+1
	\end{cases}\\
	&x^2+2x+1=0\\
	&x_{1.2}=\dfrac{-2\pm\sqrt{4-4}}{2}=\\
	&=\dfrac{-2\pm\sqrt{0}}{2}=\\
	&=-1
\end{align*}
Intersezione asse $y$
\begin{align*}
	&\begin{cases}
		x=0\\y=x^2+2x+1
	\end{cases}&\begin{cases}
	x=0\\y=0+2\cdot 0+1
\end{cases}\\&\begin{cases}
x=0\\y=1
\end{cases}
\end{align*}
Quindi \[A\coord{-1}{0} \]  \[B\coord{0}{1} \]
\begin{center}
	\includestandalone[width=.5\textwidth]{terzo/grafici/parabola5}
	\captionof{figure}{Intersezioni con gli assi}\label{fig:disegnoparabola5}
\end{center}
\begin{esempiot}{Intersezione della parabola con gli assi}{parabola6}
	Trovare i punti di intersezione della parabola $y=-2x^2+3x-4$ con gli assi.
\end{esempiot}
Intersezione asse $x$
\begin{align*}
	&\begin{cases}
		y=0\\y=-2x^2+3x-4
	\end{cases}\\
	&-2x^2+3x-4=0\\
	&x_{1.2}=\dfrac{-3\pm\sqrt{9-32}}{-4}=\\
\end{align*}
La parabola non interseca l'asse.\par
Intersezione asse $y$
\begin{align*}
	&\begin{cases}
		x=0\\y=-2x^2+3x-4
	\end{cases}&\begin{cases}
	x=0\\y=-2\cdot 0+3\cdot 0-4
\end{cases}\\&\begin{cases}
x=0\\y=-4
\end{cases}
\end{align*}
Quindi \[A\coord{0}{-4} \] 
\begin{center}
	\includestandalone[width=.5\textwidth]{terzo/grafici/parabola6}
	\captionof{figure}{Intersezioni con gli assi}\label{fig:disegnoparabola6}
\end{center}
\section{Parabola per tre punti}
\begin{esempiot}{Parabola per tre punti}{parabola7}
	Trovare la parabola che passa per i punti $A\coord{3}{5}$, $B\coord{2}{3}$ e $C\coord{-1}{5}$
\end{esempiot}
Consideriamo le parabola $y=ax^2+bx+c$

Con il passaggio per i tre punti otteniamo
\begin{align*}
	&\begin{cases}
		9a+3b+c=5\\4a+2b+c=3\\a-b+c=5
	\end{cases}
		&&\begin{cases}
			c=5-9a-3b\\4a+2b+5-9a-3b=3\\a-b+5-9a-3b=5
		\end{cases}\\
		&\begin{cases}
			c=5-9a-3b\\-5a-b=-2\\-8a-4b=0
		\end{cases}
		&&\begin{cases}
			c=5-9a-3b\\2a+b=0\\5a+b=2
		\end{cases}\\
		&\begin{cases}
			c=5-9a+6a\\b=-2a\\3a=2
		\end{cases}
		&&\begin{cases}
			c=5-3a\\b=-2a\\3a=2
		\end{cases}\\
			&\begin{cases}
		a=\dfrac{2}{3}\\b=-\dfrac{4}{3}\\c=5-3\cdot\dfrac{2}{3}
			\end{cases}
			&&\begin{cases}
			a=\dfrac{2}{3}\\b=-\dfrac{4}{3}\\c=3
			\end{cases}\\
\end{align*}
otteniamo\[y=\dfrac{2}{3}x^2-\dfrac{4}{3}x+3 \]
\begin{center}
	\includestandalone[width=.5\textwidth]{terzo/grafici/parabola7}
	\captionof{figure}{Parabola per tre punti}\label{fig:disegnoparabola7}
\end{center}
\begin{esempiot}{Parabola per tre punti}{parabola8}
	Trovare la parabola che passa per i punti $A\coord{1}{0}$, $B\coord{3}{0}$ e $C\coord{0}{4}$
\end{esempiot}
Consideriamo le parabola $y=ax^2+bx+c$

Con il passaggio per i tre punti otteniamo
\begin{align*}
	&\begin{cases}
		a+b+c=0\\9a+3b+c=0\\c=4
	\end{cases}
	&&\begin{cases}
		a+b+4=0\\9a+3b+4=0\\c=4
	\end{cases}\\
	&\begin{cases}
		c=4\\a=-b-4\\9(-b-4)+3b+4=0
	\end{cases}
	&&\begin{cases}
			c=4\\a=-b-4\\-9b-36+3b+4=0
	\end{cases}\\
	&\begin{cases}
			c=4\\a=-b-4\\-6b-32=0
	\end{cases}
	&&\begin{cases}
		c=4\\a=-b-4\\6b=-32
	\end{cases}\\
	&\begin{cases}
		c=4\\a=+\dfrac{62}{6}-4\\b=-\dfrac{62}{6}
	\end{cases}
	&&\begin{cases}
		a=\dfrac{32-24}{6}=\dfrac{8}{6}=\dfrac{4}{3}\\b=-\dfrac{32}{6}=-\dfrac{16}{3}\\c=4
	\end{cases}\\
\end{align*}
Quindi\[y=\dfrac{4}{3}x^2-\dfrac{16}{3}x+4 \]
\begin{center}
	\includestandalone[width=.5\textwidth]{terzo/grafici/parabola8}
	\captionof{figure}{Parabola per tre punti}\label{fig:disegnoparabola8}
\end{center}
\begin{esempiot}{Parabola per tre punti}{parabola9}
	Trovare la parabola che passa per i punti $A\coord{1}{3}$, $B\coord{3}{0}$ e $C\coord{0}{0}$
\end{esempiot}
Consideriamo le parabola $y=ax^2+bx+c$

Con il passaggio per i tre punti otteniamo
\begin{align*}
&\begin{cases}
a+b+c=3\\9a+3b+c=0\\c=0
\end{cases}
&&\begin{cases}
a+b=3\\9a+3b=0\\c=0
\end{cases}\\
&\begin{cases}
c=0\\b=-3a\\a-3a=3
\end{cases}
&&\begin{cases}
c=0\\b=-3a\\a-3a=3
\end{cases}\\
&\begin{cases}
c=0\\-2a=3\\b=3a
\end{cases}
&&\begin{cases}
c=0\\a=-\dfrac{3}{2}\\b=\dfrac{9}{2}
\end{cases}
\end{align*}
\begin{center}
	\includestandalone[width=.5\textwidth]{terzo/grafici/parabola9}
	\captionof{figure}{Parabola per tre punti}\label{fig:disegnoparabola9}
\end{center}
Quindi \[y=-\dfrac{3}{2}x^2+ \dfrac{9}{2}x\]
\section{Intersezioni retta parabola}
\begin{esempiot}{Intersezioni retta parabola}{parabola10}
	Trovare i punti di intersezione fra la retta $y=3x-5$ e la parabola di equazione $y=3x^2+2x+5$
\end{esempiot}
Imposto il sistema
\begin{align*}
&\begin{cases}
y=3x^2+2x-5\\
y=3x-5
\end{cases}\\
&3x^2+2x-5=3x-5\\
&3x^2+2x-5-3x+5=0\\
&3x^2-x=0\\
&x(3x-1)=0\\
&x_1=0\\
&3x-1=0\\
&x_2=\dfrac{1}{3}\\
&\begin{cases}
x_1=0\\
y=3x-5
\end{cases}
&&\begin{cases}
x_1=0\\
y=-5
\end{cases}\\
&\begin{cases}
x_2=\dfrac{1}{3}\\
y=3x-5
\end{cases}
&\begin{cases}
x_2=\dfrac{1}{3}\\
\\
y=3\dfrac{1}{3}-5
\end{cases}
&\begin{cases}
x_2=\dfrac{1}{3}\\
\\
y=-4
\end{cases}
\end{align*}
%\begin{center}
%	\includestandalone[width=\textwidth]{terzo/grafici/parabola10}
%	\captionof{figure}{Intersezione retta parabola}\label{fig:disegnoparabola10}
%\end{center}
\begin{esempiot}{Intersezioni retta parabola}{parabola11}
	Trovare i punti di intersezione fra la retta $y=2x+7$ e la parabola di equazione $y=x^2+2x+6$
\end{esempiot}
Imposto il sistema
\begin{align*}
&\begin{cases}
y=x^2+2x+6\\
y=2x+7
\end{cases}\\
&x^2+2x+6=2x+7\\
&x^2+2x+6-2x-7=0\\
&x^2-1=0\\
&x_1=+1\\
&x_2=-1\\
&\begin{cases}
x_1=+1\\
y=2\cdot 1+7
\end{cases}
&&\begin{cases}
x_1=1\\
y=9
\end{cases}\\
&\begin{cases}
x_2=-1\\
y=2\cdot(-1)+7
\end{cases}
&&\begin{cases}
x_2=-1\\
y=5
\end{cases}
\end{align*}
\begin{center}
	\includestandalone[width=.5\textwidth]{terzo/grafici/parabola11}
	\captionof{figure}{Intersezioni retta parabola}\label{fig:disegnoparabola11}
\end{center}
\begin{esempiot}{Intersezioni retta parabola}{parabola12}
	Trovare i punti di intersezione fra la retta $y=x+3$ e la parabola di equazione $y=x^2+3x+5$
\end{esempiot}
Imposto il sistema
\begin{align*}
&\begin{cases}
y=x^2+3x+5\\
y=x+3
\end{cases}\\
&x^2+3x+5=x+3\\
&x^2+3x+5-x-3=0\\
&x^2+2x+2=0\\
&x_{1,2}=\dfrac{-2\pm\sqrt{4-4\cdot 1\cdot 2}}{2}
\end{align*}
Non ha soluzione.
\begin{center}
	\includestandalone[width=.5\textwidth]{terzo/grafici/parabola12}
	\captionof{figure}{Intersezioni retta parabola}\label{fig:disegnoparabola12}
\end{center}
\begin{esempiot}{Intersezioni retta parabola}{parabola13}
	Trovare i punti di intersezione fra la retta $y=x+1$ e la parabola di equazione $y=9x^2-5x+2$
\end{esempiot}
Imposto il sistema
\begin{align*}
&\begin{cases}
y=9x^2-5x+2\\
y=x+1
\end{cases}\\
&9x^2-5x+2=x+1\\
&9x^2-5x+2-x-1=0\\
&9x^2-6x+1=0\\
&x_{1,2}=\dfrac{+6\pm\sqrt{36-4\cdot 1\cdot 9}}{18}\\
&x_{1,2}=\dfrac{+6\pm\sqrt{36-36}}{18}\\
&=\dfrac{6}{18}\\
&=\dfrac{1}{3}
&\begin{cases}
x_1=\dfrac{1}{3}\\
y=\dfrac{1}{3}+1
\end{cases}
&\begin{cases}
x_1=\dfrac{1}{3}\\
y=\dfrac{1+3}{3}
\end{cases}
&\begin{cases}
x_1=\dfrac{1}{3}\\
y=\dfrac{4}{3}
\end{cases}
\end{align*}

\begin{center}
	\includestandalone[width=.5\textwidth]{terzo/grafici/parabola13}
	\captionof{figure}{Intersezioni retta parabola}\label{fig:disegnoparabola13}
\end{center}
