\tcbstartrecording
\chapter{Triangoli}
\label{cha:trigonometriatriangoli}
 \section{Triangoli rettangoli}
 \begin{figure}
	\centering
	\includestandalone[width=.6\textwidth]{terzo/grafici/triangolopitagorico1}
	\caption{Triangolo rettangolo}
	\label{fig:triangolorettangolo}
\end{figure}

Vista la figura\nobs\vref{fig:triangolorettangolo} risolvere i seguenti esercizi
 \tcbstartrecording
 \begin{exercise}
 	Dato un triangolo rettangolo con\index{Triangolo!rettangolo}\index{Triangolo!rettangolo!ipotenusa}\index{Triangolo!rettangolo!angolo acuto}
 	\begin{align*}
 	a=&5\\
 	\beta=&\ang{30}
 	\end{align*}
\tcblower
	Dato un triangolo rettangolo con
\begin{align*}
a=&5\\
\beta=&\ang{30}
\end{align*}
Controllare che la calcolatrice sia impostata in gradi sessagesimali\index{Grado!Sessagesimale}.
Basta verificare che \testgradi 
\begin{align*}
a=&5\\
\beta=&\ang{30}\\
\gamma=&\ang{90}-\ang{30}=\ang{60}\\
c=&a\sin\gamma\\
=& 5\sin\ang{60}=\num[round-precision=\lungarrotandamento,round-mode=places]{4.330}\\
b=&a\sin\beta\\
=& 5\sin\ang{30}=\num[round-precision=1,round-mode=places]{2.5}
\end{align*}
 \end{exercise}
 \begin{exercise}[no solution]
 	 	Dato un triangolo rettangolo con\index{Triangolo!rettangolo}\index{Triangolo!rettangolo!ipotenusa}\index{Triangolo!rettangolo!angolo acuto}
 	\begin{align*}
 	a=&5\\
 	\gamma=&\ang{38}
 	\end{align*}
 \end{exercise}
\begin{exercise}[no solution]
	Dato un triangolo rettangolo con\index{Triangolo!rettangolo}\index{Triangolo!rettangolo!ipotenusa}\index{Triangolo!rettangolo!angolo acuto}
	\begin{align*}
	a=&7\\
	\beta=&\ang{42.5}
	\end{align*}
\end{exercise}
  \begin{exercise}
  	Dato un triangolo rettangolo con\index{Triangolo!rettangolo}\index{Triangolo!rettangolo!ipotenusa}\index{Triangolo!rettangolo!angolo acuto}
  	\begin{align*}
  	a=&5\\
  	\gamma=&\ang{35}
  	\end{align*}
  	\tcblower
  		Dato un triangolo rettangolo con
  	\begin{align*}
  	a=&5\\
  	\gamma=&\ang{35}
  	\end{align*}
  	Controllare che la calcolatrice sia impostata in gradi sessagesimali\index{Grado!Sessagesimale}.
  	Basta verificare che \testgradi 
  	\begin{align*}
  	a=& 7\\
  	\gamma=&\ang{35}\\
  	\beta=&\ang{90}-\ang{35}=\ang{55}\\
  	b=&a\cos\gamma\\
  	=& 7\cos\ang{55}=\num[round-precision=\lungarrotandamento,round-mode=places]{5.734064}\\
  	b=&a\sin\beta\\
  	=& 7\sin\ang{35}=\num[round-precision=\lungarrotandamento,round-mode=places]{4.015035054}
  	\end{align*}
  \end{exercise}
 \begin{exercise}[no solution]
	Dato un triangolo rettangolo con\index{Triangolo!rettangolo}\index{Triangolo!rettangolo!cateto}
	\begin{align*}
	b=&7\\
	c=&15\\
	\end{align*}
\end{exercise}
\begin{exercise}
	Dato un triangolo rettangolo con\index{Triangolo!rettangolo}\index{Triangolo!rettangolo!cateto}
	\begin{align*}
	b=&5\\
	c=&8
	\end{align*}
	\tcblower
	Dato un triangolo rettangolo con
	\begin{align*}
	b=&5\\
	c=&8
	\end{align*}
	Controllare che la calcolatrice sia impostata in gradi sessagesimali\index{Grado!Sessagesimale}.
	Basta verificare che \testgradi 
	\begin{align*}
	a=&\sqrt{8^2+5^2}\\
	=&\sqrt{89}\\
	b=&a\sin\beta\\
	\sqrt{89}\sin\beta=&5\\
	\sin\beta=&\frac{5}{\sqrt{89}}\\
	\beta=&\arcsin\frac{5}{\sqrt{89}}\\
	&\tasto{5}\tastodiv\tastoradicequadrata\tasto{89}\tastouguale\\
	&\tastoisin\tastoans\tastouguale\\
	\approx&\num[round-precision=\lungarrotandamento,round-mode=places]{32.00538321}\\
	c=&a\sin\gamma\\
\sqrt{89}\sin\gamma=&8\\
\sin\gamma=&\frac{8}{\sqrt{89}}\\
\gamma=&\arcsin\frac{8}{\sqrt{89}}\\
&\tasto{8}\tastodiv\tastoradicequadrata\tasto{89}\tastouguale\\
&\tastoisin\tastoans\tastouguale\\
&\approx\num[round-precision=\lungarrotandamento,round-mode=places]{57.999461679}\\
	\end{align*}
\end{exercise}

\begin{exercise}
	Dato un triangolo rettangolo con\index{Triangolo!rettangolo}\index{Triangolo!rettangolo!cateto}\index{Triangolo!rettangolo!angolo acuto}
	\begin{align*}
	b=&7\\
	\beta=&\ang{30}
	\end{align*}
	\tcblower
		Dato un triangolo rettangolo con
	\begin{align*}
	b=&7\\
	\beta=&\ang{30}
	\end{align*}
	Controllare che la calcolatrice sia impostata in gradi sessagesimali\index{Grado!Sessagesimale}.
	Basta verificare che \testgradi 
	\begin{align*}
	\gamma=&\ang{90}-\ang{30}\\
	c=&b\tan\gamma\\
	c=&7\tan\ang{60}\\
	&\tasto{7}\tastodiv\tastoradicequadrata\tasto{89}\tastouguale\\
	&\tastoisin\tastoans\tastouguale\\
	=&\num[round-precision=\lungarrotandamento,round-mode=places]{32.00538321}\\
	c=&a\sin\gamma\\
	\sqrt{89}\sin\gamma=&8\\
	\sin\gamma=&\frac{8}{\sqrt{89}}\\
	\gamma=&\arcsin\frac{8}{\sqrt{89}}\\
	&\tasto{8}\tastodiv\tastoradicequadrata\tasto{89}\tastouguale\\
	&\tastoisin\tastoans\tastouguale\\
	&=\num[round-precision=\lungarrotandamento,round-mode=places]{57.999461679}\\
	\end{align*}
\end{exercise}
\begin{exercise}
	Dato un triangolo rettangolo con\index{Triangolo!rettangolo}\index{Triangolo!rettangolo!ipotenusa}\index{Triangolo!rettangolo!cateto}
	\begin{align*}
	a=&7\\
	b=&5
	\end{align*}
	\tcblower
		Dato un triangolo rettangolo con
	\begin{align*}
	a=&7\\
	b=&5
	\end{align*}
	Controllare che la calcolatrice sia impostata in gradi sessagesimali\index{Grado!Sessagesimale}.
	Basta verificare che \testgradi 
	\begin{align*}
	b=&a\sin\beta\\
	5=&7\sin\beta\\
	\sin\beta=&\dfrac{5}{7}\\
	\beta=&\arcsin\left(\dfrac{5}{7}\right)\\
	&\tastoisin\tastoparentesisin\tasto{5}\tastodiv\tasto{7}\tastoparentesides\tastouguale\\
	\approx&\num[round-precision=\lungarrotandamento,round-mode=places]{45.5846914}\\
	\gamma\approx&\ang{90}-\beta=\num[round-precision=\lungarrotandamento,round-mode=places]{44.4153086}\\
	&\tasto{90}\tastomeno\tastoans\tastouguale\\
	&\num[round-precision=\lungarrotandamento,round-mode=places]{44.4153086}\\
	c=&a\sin\gamma\\
	c=&7\sin\gamma\\
	&\tastosin\tastoans\tastoper\tasto{7}\tastouguale\\
	\approx&\num[round-precision=\lungarrotandamento,round-mode=places]{4.898979486}\\
	\end{align*}
\end{exercise}
\tcbstoprecording
 \newpage
 \section{Soluzioni triangoli rettangoli}
 \tcbinputrecords