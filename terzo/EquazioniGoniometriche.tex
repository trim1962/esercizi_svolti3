\tcbstartrecording
\chapter{Equazioni goniometriche}
\label{cha:EquazioniGoniometriche}
 \section{Esercizi equazioni elementari}
% \tcbstartrecording
 \begin{exercise}
Trovare l'angolo in gradi per cui $\cos x=\num[round-precision=\lungarrotandamento,round-mode=places]{-0.7548329}$
\tcblower
$\cos x=\num[round-precision=\lungarrotandamento,round-mode=places]{-0.7548329}$

 Controllare che la calcolatrice sia impostata in gradi\index{Grado!test}.
 Basta verificare che \testgradi 
 
 In caso contrario modificare le impostazioni.
 
 Le soluzioni sono 
 \[\begin{cases}
 x_1=+\alpha+k\ang{360}\\
 x_2=-\alpha+k\ang{360}\\
 \end{cases}\]
 Calcolo $\alpha$
 
 \begin{center}
 \begin{tabular}{ll}
 \tastoicos\tasto{\num[round-precision=\lungarrotandamento,round-mode=places]{-0.7548329}}\tastouguale&\SI[round-precision=\lungarrotandamento,round-mode=places]{139.0107711}{\si{\degree}}
 \end{tabular}
 \end{center}
 
 Converto in gradi sessagesimali\index{Grado!Sessagesimale}
 
 \begin{center}
 \begin{tabular}{ll}
 \tastoans\tastomeno\tasto{139}\tastouguale&\SI[round-precision=\lungarrotandamento,round-mode=places]{0.010771083}{\si{\degree}}\\
 \tastoans\tastoper\tasto{60}\tastouguale&\SI[round-precision=\lungarrotandamento,round-mode=places]{0.646265034}{\si{\arcminute}}\\
 \tastoans\tastomeno\tasto{0}\tastouguale&\SI[round-precision=\lungarrotandamento,round-mode=places]{0.646265034}{\si{\arcminute}}\\
 \tastoans\tastoper\tasto{60}\tastouguale&\SI[round-precision=\lungarrotandamento,round-mode=places]{38.77590204}{\si{\arcsecond}}\\
 \end{tabular} 
 \end{center}
 \[\alpha=\ang{139;0;38}\]
 le soluzioni sono quindi
 \[\begin{cases}
 x_1=+\ang{139;0;38}+k\ang{360}\\
 x_2=-\ang{139;0;38}+k\ang{360}\\
 \end{cases}\]
 \end{exercise}
 \begin{exercise}
 Trovare l'angolo in gradi per cui $\sin x=\num[round-precision=\lungarrotandamento,round-mode=places]{0.666666666}$
\tcblower
 $\sin x=\num[round-precision=\lungarrotandamento,round-mode=places]{0.666666666}$
 
Controllare che la calcolatrice sia impostata in gradi\index{Grado!test} sessagesimali\index{Grado!Sessagesimale}\index{Seno}.
 
 Basta verificare che 
 
\testgradi 
 
 In caso contrario modificare le impostazioni.
 
 Le soluzioni sono 
 \[\begin{cases}
 x_1=\alpha+k\ang{360}\\
 x_2=\ang{180}-\alpha+k\ang{360}\\
 \end{cases}\]
 Calcolo $\alpha$
 \begin{center}
 \begin{tabular}{ll}
 \tastoisin\tasto{\num[round-precision=\lungarrotandamento,round-mode=places]{0.666666666}}\tastouguale&\SI[round-precision=\lungarrotandamento,round-mode=places]{41.8103149}{\si{\degree}}\\
 \end{tabular}
 \end{center}
 \[\begin{cases}
 x_1=\alpha+k\ang{360}=\SI[round-precision=\lungarrotandamento,round-mode=places]{41.8103149}{\si{\degree}}+k\ang{360}\\
 x_2=\ang{180}-\alpha+k\ang{360}=\SI[round-precision=\lungarrotandamento,round-mode=places]{138.1896851}{\si{\degree}}+k\ang{360}\\
 \end{cases}\]
 
 Converto in gradi sessagesimali\index{Grado!Sessagesimale} $x_1$
 
 \begin{center} 
 \begin{tabular}{ll}
 \tastoans\tastomeno\tasto{41}\tastouguale&\SI[round-precision=\lungarrotandamento,round-mode=places]{0.810314895}{\si{\degree}}\\
 \tastoans\tastoper\tasto{60}\tastouguale&\SI[round-precision=\lungarrotandamento,round-mode=places]{48.61889374}{\si{\arcminute}}\\
 \tastoans\tastomeno\tasto{48}\tastouguale&\SI[round-precision=\lungarrotandamento,round-mode=places]{0.68893743}{\si{\arcminute}}\\
 \tastoans\tastoper\tasto{60}\tastouguale&\SI[round-precision=\lungarrotandamento,round-mode=places]{37.13362463}{\si{\arcsecond}}\\
 \end{tabular} 
 \end{center}
 \[x_1=\ang{41;48;37}\]
 
 Converto in gradi sessagesimali\index{Grado!Sessagesimale} $x_2$
 
 \begin{center} 
 \begin{tabular}{ll}
 \tastoans\tastomeno\tasto{138}\tastouguale&\SI[round-precision=\lungarrotandamento,round-mode=places]{0.189685104}{\si{\degree}}\\
 \tastoans\tastoper\tasto{60}\tastouguale&\SI[round-precision=\lungarrotandamento,round-mode=places]{11.38110625}{\si{\arcminute}}\\
 \tastoans\tastomeno\tasto{48}\tastouguale&\SI[round-precision=\lungarrotandamento,round-mode=places]{0.381106252}{\si{\arcminute}}\\ \tastoans\tastoper\tasto{60}\tastouguale&\SI[round-precision=\lungarrotandamento,round-mode=places]{22.866375512}{\si{\arcsecond}}\\
 \end{tabular} 
 \end{center}
 \[x_2=\ang{138;11;22}\]
 
 le soluzioni sono quindi
 \[\begin{cases}
 x_1=\ang{41;48;37}+k\ang{360}\\
 x_2=\ang{138;11;22}+k\ang{360}\\
 \end{cases}\]
 \end{exercise}
 \begin{exercise}
 Trovare l'angolo in gradi per cui $\sin x=\num[round-precision=2,round-mode=places]{-0.75}$
\tcblower
 $\sin x=\num[round-precision=2,round-mode=places]{-0.75}$

 Controllare che la calcolatrice sia impostata in gradi\index{Grado!test} sessagesimali\index{Grado!Sessagesimale}\index{Seno}.
 
 Basta verificare che 
 \testgradi 
 
 In caso contrario modificare le impostazioni.
 
 Le soluzioni sono 
 \[\begin{cases}
 x_1=\alpha+k\ang{360}\\
 x_2=\ang{180}-\alpha+k\ang{360}\\
 \end{cases}\]
 Calcolo $\alpha$
 
 \begin{center}
 \begin{tabular}{ll}
 \tastoisin\tasto{\num[round-precision=2,round-mode=places]{-0.75}}\tastouguale&\SI[round-precision=\lungarrotandamento,round-mode=places]{-48.59037789}{\si{\degree}}
 \end{tabular}
 \end{center}
 
 \[\begin{cases}
 x_1=\alpha+k\ang{360}=\SI[round-precision=\lungarrotandamento,round-mode=places]{-48.59037789}{\si{\degree}}+k\ang{360}=\SI[round-precision=\lungarrotandamento,round-mode=places]{311.4096221}{\si{\degree}}+k\ang{360}\\
 x_2=-\ang{180}+\alpha+k\ang{360}=\SI[round-precision=\lungarrotandamento,round-mode=places]{228.5903789}{\si{\degree}}+k\ang{360}\\
 \end{cases}\]
 
 Converto in gradi sessagesimali\index{Grado!Sessagesimale} $x_1$
 
 \begin{center} 
 \begin{tabular}{ll}
 \tastoans\tastomeno\tasto{311}\tastouguale&\SI[round-precision=\lungarrotandamento,round-mode=places]{0.409622109}{\si{\degree}}\\
 \tastoans\tastoper\tasto{60}\tastouguale&\SI[round-precision=\lungarrotandamento,round-mode=places]{24.57732655}{\si{\arcminute}}\\
 \tastoans\tastomeno\tasto{48}\tastouguale&\SI[round-precision=\lungarrotandamento,round-mode=places]{0.577326552}{\si{\arcminute}}\\
 \tastoans\tastoper\tasto{60}\tastouguale&\SI[round-precision=\lungarrotandamento,round-mode=places]{34.63959312}{\si{\arcsecond}}\\
 \end{tabular} 
 \end{center}
 \[x_1=\ang{311;24;34}\]
 
 Converto in gradi sessagesimali\index{Grado!Sessagesimale} $x_2$
 
 \begin{center} 
 \begin{tabular}{ll}
 \tastoans\tastomeno\tasto{228}\tastouguale&\SI[round-precision=\lungarrotandamento,round-mode=places]{0.59037789}{\si{\degree}}\\
 \tastoans\tastoper\tasto{60}\tastouguale&\SI[round-precision=\lungarrotandamento,round-mode=places]{35.42267345}{\si{\arcminute}}\\
 \tastoans\tastomeno\tasto{48}\tastouguale&\SI[round-precision=\lungarrotandamento,round-mode=places]{0.422673448}{\si{\arcminute}}\\
 \tastoans\tastoper\tasto{60}\tastouguale&\SI[round-precision=\lungarrotandamento,round-mode=places]{25.36040688}{\si{\arcsecond}}\\
 \end{tabular} 
 \end{center}
 \[x_2=\ang{228;35;25}\]
 
 le soluzioni sono quindi
 \[\begin{cases}
 x_1=\ang{311;24;34}+k\ang{360}\\
 x_2=\ang{228;35;25}+k\ang{360}\\
 \end{cases}\]
 \end{exercise}
 \begin{exercise}
Trovare l'angolo in gradi per cui $\tan x=\num[round-precision=\lungarrotandamento,round-mode=places]{1.414213562}$
\tcblower
$\tan x=\num[round-precision=\lungarrotandamento,round-mode=places]{1.414213562}$

 Controllare che la calcolatrice sia impostata in gradi.
 Basta verificare che 
 
\testgradi 
 
In caso contrario modificare le impostazioni.

Le soluzioni sono \[x_1=\alpha+k\ang{180}\]

Converto in gradi sessagesimali\index{Grado!Sessagesimale} $x_1$
 \begin{center}
 \begin{tabular}{ll}
 \tastoitan\tasto{\num[round-precision=\lungarrotandamento,round-mode=places]{1.414213562}}
 \tastouguale&\SI[round-precision=\lungarrotandamento,round-mode=places]{54.73561032}{\si{\degree}}\\
 \end{tabular}
\end{center} 

 Converto in gradi sessagesimali\index{Grado!Sessagesimale} $x_1$

 \begin{center}
 \begin{tabular}{ll}
 \tastoans\tastomeno\tasto{54}\tastouguale&\SI[round-precision=\lungarrotandamento,round-mode=places]{0.735610317}{\si{\degree}}\\
 \tastoans\tastoper\tasto{60}\tastouguale&\SI[round-precision=\lungarrotandamento,round-mode=places]{44.13661903}{\si{\arcminute}}\\
 \tastoans\tastomeno\tasto{44}\tastouguale&\SI[round-precision=\lungarrotandamento,round-mode=places]{0.136619034}{\si{\arcminute}}\\
 \tastoans\tastoper\tasto{60}\tastouguale&\SI[round-precision=\lungarrotandamento,round-mode=places]{8.197142083}{\si{\arcsecond}}\\
 \end{tabular} 
 \end{center}
Le soluzioni sono \[x_1=\ang{54;44;8}+k\ang{180}\]
 \end{exercise}
 \begin{exercise}
 Trovare l'angolo in gradi per cui $\tan x=\num[round-precision=\lungarrotandamento,round-mode=places]{-3.464101615}$
 \tcblower

 $\tan x=\num[round-precision=\lungarrotandamento,round-mode=places]{-3.464101615}$
 
 Controllare che la calcolatrice sia impostata in gradi.
 
 Basta verificare che 
 \testgradi 
 
 In caso contrario modificare le impostazioni.
 
 Le soluzioni sono \[x_1=\alpha+k\ang{180}\]
 
 Converto in gradi sessagesimali\index{Grado!Sessagesimale} $x_1$
 \begin{center}
 \begin{tabular}{ll}
 \tastoitan\tasto{\num[round-precision=\lungarrotandamento,round-mode=places]{-3.464101615}}\tastouguale&\SI[round-precision=\lungarrotandamento,round-mode=places]{-73.89788625}{\si{\degree}}\\
 \end{tabular}
 \end{center} 
 
 $x_1=\ang{180}\SI[round-precision=\lungarrotandamento,round-mode=places]{-73.89788625}{\si{\degree}}=\SI[round-precision=\lungarrotandamento,round-mode=places]{106.1021138}{\si{\degree}}$
 
 Converto in gradi sessagesimali\index{Grado!Sessagesimale} $x_1$
 \begin{center}
 \begin{tabular}{ll}
 \tastoans\tastomeno\tasto{106}\tastouguale&\SI[round-precision=\lungarrotandamento,round-mode=places]{0.10211375}{\si{\degree}}\\
 \tastoans\tastoper\tasto{60}\tastouguale&\SI[round-precision=\lungarrotandamento,round-mode=places]{6.126825}{\si{\arcminute}}\\
 \tastoans\tastomeno\tasto{6}\tastouguale&\SI[round-precision=\lungarrotandamento,round-mode=places]{0.126825}{\si{\arcminute}}\\
 \tastoans\tastoper\tasto{60}\tastouguale&\SI[round-precision=\lungarrotandamento,round-mode=places]{7.6095}{\si{\arcsecond}}\\
 \end{tabular} 
 \end{center}
 Le soluzioni sono \[x_1=\ang{106;6;7}+k\ang{180}\]
 \end{exercise}
 \begin{exercise}[no solution]
 Trovare l'angolo in gradi per cui $\tan x=\dfrac{\sqrt{3}}{2}$
\end{exercise}
 \begin{exercise}[no solution]
 Trovare l'angolo in gradi per cui $\cos x=\dfrac{3}{5}$
 \end{exercise}
 \begin{exercise}[no solution]
 Trovare l'angolo in gradi per cui $\sin x=-\dfrac{\sqrt{3}}{2}$
 \end{exercise}
 \begin{exercise}
 Trovare l'angolo in radianti per cui $\cos x=\num[round-precision=\lungarrotandamento,round-mode=places]{-0.478973}$
 \tcblower
 $\cos x=\num[round-precision=\lungarrotandamento,round-mode=places]{-0.478973}$
 
 Controllare che la calcolatrice sia impostata in radianti\index{Radianti!test}\index{Radianti}\index{Coseno}.
 
 Basta verificare che 
 
 \testradianti
 
 In caso contrario modificare le impostazioni.
 
 Non resta che procedere con il calcolo.
 
 Le soluzioni sono 
 \[\begin{cases}
 x_1=+\alpha+2k\pi\\
 x_2=-\alpha+2k\pi\\
 \end{cases}\]
 Calcolo $\alpha$
 \begin{center}
 \begin{tabular}{ll}
 \tastoicos\tasto{\num[round-precision=\lungarrotandamento,round-mode=places]{-0.4788973}}\tastouguale&\SI[round-precision=\lungarrotandamento,round-mode=places]{2.070280734}{\radian}\\ 
 \end{tabular} 
 \end{center}
 \[\alpha= \SI[round-precision=\lungarrotandamento,round-mode=places]{2.070280734}{\radian}\]
 \[\begin{cases}
 x_1=+\SI[round-precision=\lungarrotandamento,round-mode=places]{2.070280734}+2k\pi \si{\radian}\\
 x_2=-\SI[round-precision=\lungarrotandamento,round-mode=places]{2.070280734}+2k\pi \si{\radian}\\
 \end{cases}\]
 \end{exercise}
% \tcbstoprecording
% \newpage
% \section{Soluzioni equazioni goniometriche elementari}
% \tcbinputrecords
% \newpage
\section{Esercizi equazioni goniometriche}
% \tcbstartrecording
 \begin{exercise}
 Trovare l'angolo in radianti per cui $\sin 3x=\num[round-precision=2,round-mode=places]{0.48}$
 \tcblower
$\sin 3x=\num[round-precision=2,round-mode=places]{0.48}$ 
 
 Controllare che la calcolatrice sia impostata in radianti\index{Radianti!test}\index{Radianti}\index{Seno}.
 
 Basta verificare che 
 \testradianti
 
 In caso contrario modificare le impostazioni.
 
 Non resta che procedere con il calcolo.
 
 Le soluzioni sono 
 \[\begin{cases}
 x_1=\alpha+2k\pi\\
 x_2=\pi-\alpha+2k\pi\\
 \end{cases}\]
 Calcolo $\alpha$
 
 \begin{center}
 \begin{tabular}{ll}
 \tastoisin\tasto{\num[round-precision=2,round-mode=places]{0.48}}
 \tastouguale&\num[round-precision=\lungarrotandamento,round-mode=places]{0.500654712}\\ 
 \end{tabular} 
 \end{center}
 \[\alpha= \SI[round-precision=\lungarrotandamento,round-mode=places]{0.500654712}{\radian}\]
 \begin{align*}
 3x_1&=\SI[round-precision=\lungarrotandamento,round-mode=places]{0.500654712}+2k\pi \si{\radian}\\
 x_1&=\SI[round-precision=\lungarrotandamento,round-mode=places]{0.166884904}+\dfrac{2}{3}k\pi \si{\radian}\\
 \end{align*}
 \begin{align*}
 3x_2&=\pi-\SI[round-precision=\lungarrotandamento,round-mode=places]{0.500654712}+2k\pi \si{\radian}\\
 3x_2&=\SI[round-precision=\lungarrotandamento,round-mode=places]{2.640937182}+2k\pi \si{\radian}\\
 x_2&=\SI[round-precision=\lungarrotandamento,round-mode=places]{0.880312394}+\dfrac{2}{3}k\pi \si{\radian}\\
 \end{align*}
 Le soluzioni sono
 
\[\begin{cases}
x_1=\SI[round-precision=\lungarrotandamento,round-mode=places]{0.166884904}+\dfrac{2}{3}k\pi \si{\radian}\\
\\
x_2=\SI[round-precision=\lungarrotandamento,round-mode=places]{0.880312394}+\dfrac{2}{3}k\pi \si{\radian}\\
 \end{cases}\]
 \end{exercise}
 \begin{exercise}
 Trovare l'angolo in radianti per cui $\sin 2x=-\num[round-precision=3,round-mode=places]{0.128}$
 \tcblower
 $\sin 2x=-\num[round-precision=3,round-mode=places]{0.128}$ 
 
 Controllare che la calcolatrice sia impostata in radianti\index{Radianti!test}\index{Radianti}\index{Seno}.
 
 Basta verificare che 
 \testradianti
 
 In caso contrario modificare le impostazioni.
 
 Non resta che procedere con il calcolo.
 
 Le soluzioni sono 
 \[\begin{cases}
 x_1=\alpha+2k\pi\\
 x_2=\pi-\alpha+2k\pi\\
 \end{cases}\]
 Calcolo $\alpha$
 
 \begin{center}
 \begin{tabular}{ll}
 \tastoisin\tasto{\num[round-precision=3,round-mode=places]{-0.128}}
 \tastouguale&\num[round-precision=\lungarrotandamento,round-mode=places]{-0.128352127} 
 \end{tabular} 
 \end{center}
 Soluzioni per valori negativi dell'angolo
 \begin{align*}
 	2x_1&=\SI[round-precision=\lungarrotandamento,round-mode=places]{-0.128352127}+2k\pi \si{\radian}\\
 	x_1&=\SI[round-precision=\lungarrotandamento,round-mode=places]{-0.064176063}+k\pi \si{\radian}\\
 \end{align*}
 \begin{align*}
 	2x_2&=-\pi-(\SI[round-precision=\lungarrotandamento,round-mode=places]{-0.128352127})+2k\pi \si{\radian}\\
 	2x_2&=-\SI[round-precision=\lungarrotandamento,round-mode=places]{3.013240526}+2k\pi \si{\radian}\\
 	x_2&=-\SI[round-precision=\lungarrotandamento,round-mode=places]{1.506620263}+k\pi \si{\radian}\\
 \end{align*}
 Le soluzioni sono
 
 \[\begin{cases}
 x_1&=\SI[round-precision=\lungarrotandamento,round-mode=places]{-0.064176063}+k\pi \si{\radian}\\
 x_2&=-\SI[round-precision=\lungarrotandamento,round-mode=places]{1.506620263}+k\pi \si{\radian}\\
 \end{cases}\]
 Soluzioni per valori positivi dell'angolo
 \[\alpha=2\pi+ \SI[round-precision=\lungarrotandamento,round-mode=places]{-0.128352127}{\radian}\]
 \begin{align*}
 2x_1&=\SI[round-precision=\lungarrotandamento,round-mode=places]{6.154833179}+2k\pi \si{\radian}\\
 x_1&=\SI[round-precision=\lungarrotandamento,round-mode=places]{3,0774165895}+k\pi \si{\radian}\\
 \end{align*}
 \begin{align*}
 2x_2&=\pi+\SI[round-precision=\lungarrotandamento,round-mode=places]{0.128352127}+2k\pi \si{\radian}\\
 2x_2&=\SI[round-precision=\lungarrotandamento,round-mode=places]{3.2699447805897}+2k \si{\radian}\\
 x_2&=\SI[round-precision=\lungarrotandamento,round-mode=places]{1.63497239}+k\pi \si{\radian}\\
 \end{align*}
 Le soluzioni sono
 
 \[\begin{cases}
 x_1&=\SI[round-precision=\lungarrotandamento,round-mode=places]{3.205768717}+k\pi \si{\radian}\\
 x_2&=\SI[round-precision=\lungarrotandamento,round-mode=places]{1.63497239}+k\pi \si{\radian}\\
 \end{cases}\]
 
 \end{exercise}
 
 \begin{exercise}
 	Trovare l'angolo in radianti per cui $\sin 6x=-\dfrac{5}{8}$
 	\tcblower
 	$\sin 6x=-\num[round-precision=3,round-mode=places]{0.625}$ 
 	
 	Controllare che la calcolatrice sia impostata in gradi \index{Seno}.
 	
 	Basta verificare che 
 	\testgradi
 	
 	In caso contrario modificare le impostazioni.
 	
 	Non resta che procedere con il calcolo.
 	
 	Le soluzioni sono 
 	\[\begin{cases}
 	x_1=\alpha+k\ang{360;;}\\
 	x_2=\pi-\alpha+k\ang{360;;}\\
 	\end{cases}\]
 	Calcolo $\alpha$
 	
 	\begin{center}
 		\begin{tabular}{ll}
 			\tastoisin\tasto{-\num[round-precision=3,round-mode=places]{0.625}}
 			\tastouguale&\num[round-precision=\lungarrotandamento,round-mode=places]{-38.68218745} 
 		\end{tabular} 
 	\end{center}
 	Soluzioni per valori negativi dell'angolo
 	\begin{align*}
 	6x_1&=-\SI[round-precision=\lungarrotandamento,round-mode=places]{38.68218745}{\si{\degree}}+k\ang{360;;}\\
 	x_1&=-\SI[round-precision=\lungarrotandamento,round-mode=places]{6.447031242}{\si{\degree}}+k\ang{60;;}\\
 	\end{align*}
 	\begin{align*}
 	6x_2&=-\ang{180;;}-(-\SI[round-precision=\lungarrotandamento,round-mode=places]{38.68218745}{\si{\degree}})+k\ang{360}\\
 	6x_2&=-\SI[round-precision=\lungarrotandamento,round-mode=places]{141.3178125}{\si{\degree}}+k\ang{360}\\
 	x_2&=-\SI[round-precision=\lungarrotandamento,round-mode=places]{23.55296876}{\si{\degree}}+k\ang{60}\\
 	\end{align*}
 	Le soluzioni sono
 	
 	\[\begin{cases}
 	x_1&=-\SI[round-precision=\lungarrotandamento,round-mode=places]{6.447031242}{\si{\degree}}+k\ang{60;;}\\
 	x_2&=-\SI[round-precision=\lungarrotandamento,round-mode=places]{23.55296876}{\si{\degree}}+k\ang{60}\\
 	\end{cases}\]
 	Soluzioni per valori positivi dell'angolo
 	\[\alpha=\ang{360} -\SI[round-precision=\lungarrotandamento,round-mode=places]{38.682187745}{\si{\degree}}=\SI[round-precision=\lungarrotandamento,round-mode=places]{141.3178125}{\si{\degree}}\]
 	\begin{align*}
 	6x_1&=\SI[round-precision=\lungarrotandamento,round-mode=places]{141.3178125}{\si{\degree}} +k\ang{360}\\
 	x_1&=\SI[round-precision=\lungarrotandamento,round-mode=places]{23.552966875}{\si{\degree}} +k\ang{60}\\
 	\end{align*}
 	\[\alpha=\ang{360} -\ang{180;;}-(-\SI[round-precision=\lungarrotandamento,round-mode=places]{38.68218745}{\si{\degree}})=\SI[round-precision=\lungarrotandamento,round-mode=places]{218.68221875}{\si{\degree}}\]
 	\begin{align*}
 	6x_2&=\SI[round-precision=\lungarrotandamento,round-mode=places]{218.68221875}{\si{\degree}} +k\ang{60}\\
 	x_2&=\SI[round-precision=\lungarrotandamento,round-mode=places]{36.44703646}{\si{\degree}} +k\ang{60}\\
 	\end{align*}
 	Le soluzioni sono
 	
 	\[\begin{cases}
 	x_1&=\SI[round-precision=\lungarrotandamento,round-mode=places]{23.552966875}{\si{\degree}} +k\ang{60}\\
 	x_2&=\SI[round-precision=\lungarrotandamento,round-mode=places]{36.44703646}{\si{\degree}} +k\ang{60}\\
 	\end{cases}\]
% 	
\end{exercise}
 \begin{exercise}
 Trovare l'angolo in radianti per cui $\cos 2x=\num[round-precision=3,round-mode=places]{0.128}$
 \tcblower
 $\cos 2x=\num[round-precision=3,round-mode=places]{0.128}$ 
 
 Controllare che la calcolatrice sia impostata in radianti\index{Radianti!test}\index{Radianti}\index{Coseno}.
 
 Basta verificare che 
 \testradianti
 
 In caso contrario modificare le impostazioni.
 
 Non resta che procedere con il calcolo.
 
 Le soluzioni sono 
 \[\begin{cases}
 x_1=+\alpha+2k\pi\\
 x_2=-\alpha+2k\pi\\
 \end{cases}\]
 Calcolo $\alpha$
 
 \begin{center}
 \begin{tabular}{ll}
 \tastoicos\tasto{\num[round-precision=3,round-mode=places]{0.128}}
 \tastouguale&\num[round-precision=\lungarrotandamento,round-mode=places]{1.442444199} 
 \end{tabular} 
 \end{center}
 \[\alpha=\SI[round-precision=\lungarrotandamento,round-mode=places]{1.442444199}{\radian} +2k\pi\]
 \begin{align*}
 2x_1&=\SI[round-precision=\lungarrotandamento,round-mode=places]{1.442444199}+2k\pi \si{\radian}\\
 x_1&=\SI[round-precision=\lungarrotandamento,round-mode=places]{0.721222099}+k\pi \si{\radian}\\
 \end{align*}
 Le soluzioni sono
 
 \[\begin{cases}
 x_1&=+\SI[round-precision=\lungarrotandamento,round-mode=places]{0.721222099}+k\pi \si{\radian}\\\
 
 x_2&=-\SI[round-precision=\lungarrotandamento,round-mode=places]{0.721222099}+k\pi \si{\radian}\\ 
 \end{cases}\]
 \end{exercise}
 \begin{exercise}[no solution]
 Trovare l'angolo in gradi per cui $\tan 3x=\dfrac{3}{5}$
 \end{exercise}
 \begin{exercise}[no solution]
 Trovare l'angolo in radianti per cui $\tan 10x=-\dfrac{7}{2}$
 \end{exercise}
 \begin{exercise}[no solution]
 Trovare l'angolo in gradi per cui $\cos 4x=\dfrac{\sqrt{3}}{2}$
 \end{exercise}
 \begin{exercise}[no solution]
 Trovare l'angolo in gradi per cui $\sin 4x=-\dfrac{1}{2}$
 \end{exercise}

 \begin{exercise}
 Trovare l'angolo in gradi per cui $\cos (3x+\ang{30;;})=\dfrac{1}{4}$
 \tcblower
 $\cos (3x+\ang{30;;})=\dfrac{1}{4}$
 
 Controllare che la calcolatrice sia impostata in gradi\index{Grado!test}\index{Coseno}.
 
 Basta verificare che 
 \testgradi
 
 In caso contrario modificare le impostazioni.
 
 Non resta che procedere con il calcolo.
 
 Le soluzioni sono 
 \[\begin{cases}
 x_1=+\alpha+k\ang{360;;}\\
 x_2=-\alpha+k\ang{360;;}\\
 \end{cases}\]
 Calcolo $\alpha$
 \begin{center}
 \begin{tabular}{ll}
 \tastoicos\tasto{\num[round-precision=2,round-mode=places]{0.25}}
 \tastouguale&\num[round-precision=\lungarrotandamento,round-mode=places]{75.52248781} 
 \end{tabular} 
 \end{center}
 \[\alpha=\SI[round-precision=\lungarrotandamento,round-mode=places]{75.52248781}{\si{\degree}}\]
 \begin{align*}
 3x_1+\ang{30;;}&=\SI[round-precision=\lungarrotandamento,round-mode=places]{75.52248781}{\si{\degree}}+k\ang{360;;}\\
 3x_1&=\SI[round-precision=\lungarrotandamento,round-mode=places]{75.52248781}{\si{\degree}}-\ang{30;;}+k\ang{360;;}\\
 3x_1&=\SI[round-precision=\lungarrotandamento,round-mode=places]{45.52248781}{\si{\degree}}+k\ang{360;;}\\
 x_1&=\SI[round-precision=\lungarrotandamento,round-mode=places]{15.1741626}{\si{\degree}}+k\ang{120;;}\\
 \end{align*}
 \begin{align*}
 3x_2+\ang{30;;}&=-\SI[round-precision=\lungarrotandamento,round-mode=places]{75.52248781}{\si{\degree}}+k\ang{360;;}\\
 3x_2&=-\SI[round-precision=\lungarrotandamento,round-mode=places]{75.52248781}{\si{\degree}}-\ang{30;;}+k\ang{360;;}\\
 3x_1&=\SI[round-precision=\lungarrotandamento,round-mode=places]{-105.5224878}{\si{\degree}}+k\ang{360;;}\\
 x_2&=\SI[round-precision=\lungarrotandamento,round-mode=places]{-35.1741626}{\si{\degree}}+k\ang{180;;}\\
 \end{align*}
 
 Le soluzioni sono
 
 \[\begin{cases}
x_1=\SI[round-precision=\lungarrotandamento,round-mode=places]{15.1741626}{\si{\degree}}+k\ang{120;;}\\
x_2=\SI[round-precision=\lungarrotandamento,round-mode=places]{-35.1741626}{\si{\degree}}+k\ang{120;;}\\
 \end{cases}\]
 \end{exercise}
 \begin{exercise}
 	Trovare l'angolo in gradi per cui $\sin (3x+\ang{45;;})=\dfrac{3}{5}$
 	\tcblower
 	$\sin (3x+\ang{45;;})=\dfrac{3}{5}$
 	
 	Controllare che la calcolatrice sia impostata in gradi\index{Grado!test}\index{Grado!test}\index{Seno}.
 	
 	Basta verificare che 
 	\testgradi
 	
 	In caso contrario modificare le impostazioni.
 	
 	Non resta che procedere con il calcolo.
 	
 	Le soluzioni sono 
 	\[\begin{cases}
 	x_1=+\alpha+k\ang{360;;}\\
 	x_2=\ang{180;;}-\alpha+k\ang{360;;}\\
 	\end{cases}\]
 	Calcolo $\alpha$
 	\begin{center}
 		\begin{tabular}{ll}
 			\tastoisin\tasto{\num[round-precision=1,round-mode=places]{0.6}}
 			\tastouguale&\num[round-precision=\lungarrotandamento,round-mode=places]{36.86989765} 
 		\end{tabular} 
 	\end{center}
 	\[\alpha=\SI[round-precision=\lungarrotandamento,round-mode=places]{36.86989765}{\si{\degree}}\]
 	\begin{align*}
 		3x_1+\ang{45;;}&=\SI[round-precision=\lungarrotandamento,round-mode=places]{36.86989765}{\si{\degree}}+k\ang{360;;}\\
 		3x_1&=\SI[round-precision=\lungarrotandamento,round-mode=places]{36.86989765}{\si{\degree}}-\ang{45;;}+k\ang{360;;}\\
 		3x_1&=-\SI[round-precision=\lungarrotandamento,round-mode=places]{8.130102354}{\si{\degree}}+k\ang{360;;}\\
 		x_1&=-\SI[round-precision=\lungarrotandamento,round-mode=places]{2.710034118}{\si{\degree}}+k\ang{120;;}\\
 	\end{align*}
 	\begin{align*}
 		3x_2+\ang{45;;}&=\ang{180;;}-\SI[round-precision=\lungarrotandamento,round-mode=places]{36.86989765}{\si{\degree}}+k\ang{360;;}\\
 		3x_2+\ang{45;;}&=\SI[round-precision=\lungarrotandamento,round-mode=places]{143.1301024}{\si{\degree}}+k\ang{360;;}\\
 		3x_2&=\SI[round-precision=\lungarrotandamento,round-mode=places]{143.1301024}{\si{\degree}}-\ang{45;;}+k\ang{360;;}\\
 		3x_2&=\SI[round-precision=\lungarrotandamento,round-mode=places]{98.13010235}{\si{\degree}}+k\ang{360;;}\\
 		x_2&=\SI[round-precision=\lungarrotandamento,round-mode=places]{32.71003412}{\si{\degree}}+k\ang{120;;}\\
 	\end{align*}
 	
 	Le soluzioni sono
 	
 	\[\begin{cases}
x_1=-\SI[round-precision=\lungarrotandamento,round-mode=places]{2.710034118}{\si{\degree}}+k\ang{120;;}\\
x_2=\SI[round-precision=\lungarrotandamento,round-mode=places]{32.71003412}{\si{\degree}}+k\ang{120;;}\\
 	\end{cases}\]
 \end{exercise}
 \begin{exercise}
 	Trovare l'angolo in gradi per cui $\tan (5x-\ang{70;;})=5$
 	\tcblower
 	$\tan (5x-\ang{70;;})=5$
 	
 	Controllare che la calcolatrice sia impostata in gradi\index{Grado!test}\index{Grado!test}\index{Tangente}.
 	
 	Basta verificare che 
 	\testgradi
 	
 	In caso contrario modificare le impostazioni.
 	
 	Non resta che procedere con il calcolo.
 	
 	Le soluzioni sono 
 	\[x_1=+\alpha+k\ang{180;;}\]
 	Calcolo $\alpha$
 	\begin{center}
 		\begin{tabular}{ll}
 			\tastoitan\tasto{\num[round-precision=1,round-mode=places]{5}}
 			\tastouguale&\num[round-precision=\lungarrotandamento,round-mode=places]{76.69006753} 
 		\end{tabular} 
 	\end{center}
 	\[\alpha=\SI[round-precision=\lungarrotandamento,round-mode=places]{78.69006753}{\si{\degree}}\]
 	\begin{align*}
 	5x_1-\ang{70;;}&=\SI[round-precision=\lungarrotandamento,round-mode=places]{78.69006753}{\si{\degree}}+k\ang{180;;}\\
 	5x_1&=\SI[round-precision=\lungarrotandamento,round-mode=places]{78.69006753}{\si{\degree}}+\ang{70;;}+k\ang{180;;}\\
 	5x_1&=\SI[round-precision=\lungarrotandamento,round-mode=places]{148.6900675}{\si{\degree}}+k\ang{180;;}\\
 	x_1&=\SI[round-precision=\lungarrotandamento,round-mode=places]{29.73801351}{\si{\degree}}+k\ang{36;;}
 	\end{align*}
 	
 	La soluzione è
 \[x_1=\SI[round-precision=\lungarrotandamento,round-mode=places]{29.73801351}{\si{\degree}}+k\ang{36;;}\]
 \end{exercise}
 \begin{exercise}
 	Trovare l'angolo in radianti per cui $\cos (4x+\dfrac{4}{9}\pi)=-\dfrac{15}{16}$
 	\tcblower
 $\cos (4x-\dfrac{4}{9}\pi)=-\dfrac{15}{16}=-\num[round-precision=4,round-mode=places]{0.9375}$
 	
 	Controllare che la calcolatrice sia impostata in radianti\index{Radianti!test}\index{Coseno}.
 	
 	Basta verificare che 
 	\testradianti
 	
 	In caso contrario modificare le impostazioni.
 	
 	Non resta che procedere con il calcolo.
 	
 	Le soluzioni sono 
 	\[\begin{cases}
 	x_1=+\alpha+2k\pi\\
 	x_2=-\alpha+2k\pi\\
 	\end{cases}\]
 	Calcolo $\alpha$
 	\begin{center}
 		\begin{tabular}{ll}
 			\tastoicos\tasto{-\num[round-precision=4,round-mode=places]{0.9375}}
 			\tastouguale&\num[round-precision=\lungarrotandamento,round-mode=places]{2.786171452} 
 		\end{tabular} 
 	\end{center}
 	\[\alpha=\SI[round-precision=\lungarrotandamento,round-mode=places]{2.786171452}{\radian}\]
 	\begin{align*}
 	4x_1-\dfrac{4}{9}\pi&=\SI[round-precision=\lungarrotandamento,round-mode=places]{2.786171452}{\radian}+2k\pi\\
 	4x_1&=\SI[round-precision=\lungarrotandamento,round-mode=places]{2.786171452}{\radian}+\dfrac{4}{9}\pi+2k\pi\\
 	4x_1&=\SI[round-precision=\lungarrotandamento,round-mode=places]{2.786171452}{\radian}+\SI[round-precision=\lungarrotandamento,round-mode=places]{1.396263402}{\radian}+2k\pi\\
 	4x_1&=\SI[round-precision=\lungarrotandamento,round-mode=places]{4.182408602}{\radian}+2k\pi\\
 	x_1&=\SI[round-precision=\lungarrotandamento,round-mode=places]{1.045602151}{\radian}+\dfrac{1}{2}k\pi\\
 	\end{align*}
\begin{align*}
4x_2-\dfrac{4}{9}\pi&=-\SI[round-precision=\lungarrotandamento,round-mode=places]{2.786171452}{\radian}+2k\pi\\
4x_1&=-\SI[round-precision=\lungarrotandamento,round-mode=places]{2.786171452}{\radian}+\dfrac{4}{9}\pi+2k\pi\\
4x_2&=-\SI[round-precision=\lungarrotandamento,round-mode=places]{2.786171452}{\radian}+\SI[round-precision=\lungarrotandamento,round-mode=places]{1.396263402}{\radian}+2k\pi\\
4x_2&=-\SI[round-precision=\lungarrotandamento,round-mode=places]{1.38990805}{\radian}+2k\pi\\
x_2&=-\SI[round-precision=\lungarrotandamento,round-mode=places]{0.347477012}{\radian}+\dfrac{1}{2}k\pi\\
\end{align*}
 	
 	Le soluzioni sono
 	
 	\[\begin{cases}
x_1=\SI[round-precision=\lungarrotandamento,round-mode=places]{1.045602151}{\radian}+\dfrac{1}{2}k\pi\\
x_2=-\SI[round-precision=\lungarrotandamento,round-mode=places]{0.347477012}{\radian}+\dfrac{1}{2}k\pi\\
 	\end{cases}\]
 \end{exercise}
 \begin{exercise}[no solution]
 	Trovare l'angolo in gradi per cui $\tan (3x+\ang{50;;})=\dfrac{\sqrt{3}}{2}$
 \end{exercise}
 \begin{exercise}[no solution]
 	Trovare l'angolo in gradi per cui $\cos (7x+\ang{40;;})-=\dfrac{3}{5}$
 \end{exercise}
 \begin{exercise}[no solution]
 	Trovare l'angolo in radianti per cui $\sin (3x+\dfrac{3}{5}\pi)=-\dfrac{\sqrt{3}}{2}$
 \end{exercise}
\begin{exercise}
	Trovare l'angolo in gradi per cui $\tan (5x-\ang{70;;})=5$
	\tcblower
	$\tan (5x-\ang{70;;})=-5$
	
	Controllare che la calcolatrice sia impostata in gradi\index{Grado!test}\index{Grado!test}\index{Tangente}.
	
	Basta verificare che 
	\testgradi
	
	In caso contrario modificare le impostazioni.
	
	Non resta che procedere con il calcolo.
	
	Le soluzioni sono 
	\[x_1=+\alpha+k\ang{180;;}\]
	Calcolo $\alpha$
	\begin{center}
		\begin{tabular}{ll}
			\tastoitan\tasto{\num[round-precision=1,round-mode=places]{5}}
			\tastouguale&\num[round-precision=\lungarrotandamento,round-mode=places]{-78.69006753} 
		\end{tabular} 
	\end{center}
	\[\alpha=-\SI[round-precision=\lungarrotandamento,round-mode=places]{78.69006753}{\si{\degree}}\]
	\begin{align*}
	5x_1-\ang{70;;}&=-\SI[round-precision=\lungarrotandamento,round-mode=places]{76.69006753}{\si{\degree}}+k\ang{180;;}\\
	5x_1&=-\SI[round-precision=\lungarrotandamento,round-mode=places]{78.69006753}{\si{\degree}}+\ang{70;;}+k\ang{180;;}\\
	5x_1&=-\SI[round-precision=\lungarrotandamento,round-mode=places]{8.690067526}{\si{\degree}}+k\ang{180;;}\\
	x_1&=-\SI[round-precision=\lungarrotandamento,round-mode=places]{1.738013505}{\si{\degree}}+k\ang{36;;}
	\end{align*}
	
	La soluzione è
	\[x_1=-\SI[round-precision=\lungarrotandamento,round-mode=places]{1.738013505}{\si{\degree}}+k\ang{36;;}\]
\end{exercise}
\begin{exercise}
	Trovare i valori dell'incognita per cui $\cos(3x+\ang{50;;})=\cos(2x-\ang{30;;})$
	\tcblower
	$\cos3x+\ang{50;;})=\cos(2x-\ang{30;;})$
	
	Le soluzioni sono 
	\[\begin{cases}
	x_1=+\alpha+k\ang{360;;}\\
	x_2=-\alpha+k\ang{360;;}\\
	\end{cases}\]
	Troviamo la prima
	
	\begin{align*}
	3x+\ang{50;;}&=2x-\ang{30;;}+k\ang{360;;}\\
	3x-2x&=-\ang{50;;}-\ang{30;;}+k\ang{360;;}\\
	x&=-\ang{80;;}+k\ang{360;;}
	\end{align*}
 Troviamo la seconda
	
	\begin{align*}
	3x+\ang{50;;}&=-2x+\ang{30;;}+k\ang{360;;}\\
	3x+2x&=-\ang{50;;}+\ang{30;;}+k\ang{360;;}\\
	5x&=-\ang{20;;}+k\ang{360;;}\\
 x&=-\ang{4;;}+k\ang{72;;}
	\end{align*}
	
	Le soluzioni sono
	
	\[\begin{cases}
	x_1=-\ang{80;;}+k\ang{360;;}\\
	x_2=-\ang{4;;}+k\ang{72;;}\\
	\end{cases}\]
\end{exercise}
\begin{exercise}
	Trovare i valori dell'incognita per cui $\sin(6x-\dfrac{\pi}{3})=\sin(4x +\dfrac{2}{3}\pi)$
	\tcblower
$\sin(6x-\dfrac{\pi}{3})=\sin(4x +\dfrac{2}{3}\pi)$
	
	Le soluzioni sono 
	\[\begin{cases}
	x_1=+\alpha+2k\pi\\
	x_2=\pi-\alpha+2k\pi\\
	\end{cases}\]
	Troviamo la prima
	
	\begin{align*}
	6x-\dfrac{\pi}{3}&=4x +\dfrac{2}{3}\pi+2k\pi\\
6x-4x&=\dfrac{\pi}{3} +\dfrac{2}{3}\pi+2k\pi\\
	2x&=\pi+2k\pi\\
	x&=\dfrac{\pi}{2}+k\pi
	\end{align*}
	Troviamo la seconda
	
	\begin{align*}
		6x-\dfrac{\pi}{3}&=\pi-4x-\dfrac{2}{3}\pi+2k\pi\\
		6x+4x&=\pi+\dfrac{\pi}{3}-\dfrac{2}{3}\pi+2k\pi\\
		10x&=\dfrac{2}{3}\pi+2k\pi\\
		x&=\dfrac{2}{30}\pi+k\dfrac{1}{5}\pi
	\end{align*}
	
	Le soluzioni sono
	
	\[\begin{cases}
		x=\dfrac{\pi}{2}+k\pi\\
		\\
	x=\dfrac{2}{30}\pi+k\dfrac{1}{5}\pi
	\end{cases}\]
\end{exercise}
\begin{exercise}
	Trovare i valori dell'incognita per cui $\tan(5x-\ang{20;;})=\tan(2x+\ang{30;;})$
	\tcblower
	$\tan(5x-\ang{20;;})=\tan(2x+\ang{30;;})$
	
	Esistenza lato sinistro
	
		\begin{align*}
		5x-\ang{20;;}&=\ang{90;;}+k\ang{180;;}\\
		5x&=\ang{90;;}+\ang{20;;}+k\ang{180;;}\\
		5x&=\ang{110;;}+k\ang{180;;}\\
	 x&=\ang{22;;}+k\ang{36;;}
		\end{align*}
	
	Esistenza lato destro
	
	\begin{align*}
	2x+\ang{30;;}&=\ang{90;;}+k\ang{180;;}\\
	2x&=\ang{90;;}-\ang{30;;}+k\ang{180;;}\\
	2x&=\ang{60;;}+k\ang{180;;}\\
	x&=\ang{30;;}+k\ang{90;;}
	\end{align*}
	
	
	\begin{align*}
	5x-\ang{20;;}&=2x+\ang{30;;}+k\ang{180;;}\\
	3x&=\ang{20;;}+\ang{30;;}+k\ang{180;;}\\
	3x&=\ang{50;;}+k\ang{180;;}\\
	x =\dfrac{\ang{50;;}}{3}+k\ang{60;;}
	\end{align*}
	
	La soluzione è
	
\[x =\dfrac{\ang{50;;}}{3}+k\ang{60;;}\]
\end{exercise}
\begin{exercise}
	Trovare i valori dell'incognita per cui $\cos(3x+\ang{50;;})=-\cos(2x-\ang{40;;})$
	\tcblower
$\cos(3x+\ang{50;;})=-\cos(2x-\ang{40;;})=\cos(\ang{180;;}-2x+\ang{40;;})$
	
	Le soluzioni sono 
	\[\begin{cases}
	x_1=+\alpha+k\ang{360;;}\\
	x_2=-\alpha+k\ang{360;;}\\
	\end{cases}\]
	Troviamo la prima
	
	\begin{align*}
	3x+\ang{50;;}&=\ang{180;;}-2x+\ang{40;;}+k\ang{360;;}\\
	3x+2x&=\ang{180;;}-\ang{50;;}+\ang{40;;}+k\ang{360;;}\\
	5x&=\ang{170;;}+k\ang{360;;}\\
	x&=\ang{34;;}+k\ang{72;;}
	\end{align*}
	Troviamo la seconda
	
	\begin{align*}
		3x+\ang{50;;}&=-\ang{180;;}+2x-\ang{40;;}+k\ang{360;;}\\
		3x-2x&=-\ang{180;;}-\ang{50;;}-\ang{40;;}+k\ang{360;;}\\
		x&=-\ang{270;;}+k\ang{360;;}
	\end{align*}
	
	Le soluzioni sono
	
	\[\begin{cases}
	x_1=\ang{34;;}+k\ang{72;;}\\
	x_2-\ang{270;;}+k\ang{360;;}
	\end{cases}\]
\end{exercise}

\begin{exercise}
	Trovare i valori dell'incognita per cui $\sin(5x-\ang{70;;})=-\sin(7x +\ang{80;;})$
	\tcblower
$\sin(5x-\ang{70;;})=-\sin(7x +\ang{80;;})=\sin(-7x-\ang{80;;})$
	
	Le soluzioni sono 
	\[\begin{cases}
	x_1=\alpha+k\ang{360;;}\\
	x_2=\ang{180;;}-\alpha+k\ang{360;;}\\
	\end{cases}\]
	Troviamo la prima
	
	\begin{align*}
		5x-\ang{70;;}&=-7x-\ang{80;;}+k\ang{360;;}\\
		5x+7x&=\ang{70;;}-\ang{80;;}+k\ang{360;;}\\
		12x&=-\ang{10;;}+k\ang{360;;}\\
		x&=-\dfrac{\ang{10;;}}{12}+k\ang{30;;}
	\end{align*}
	Troviamo la seconda
	
	\begin{align*}
		5x-\ang{70;;}&=\ang{180;;}+7x+\ang{80;;}+k\ang{360;;}\\
		5x-7x&=\ang{180;;}+\ang{70;;}+\ang{80;;}+k\ang{360;;}\\
		-2x&=\ang{330;;}+k\ang{360;;}\\
		x&=-\ang{165;;}+k\ang{180;;}
	\end{align*}
	
	Le soluzioni sono
	
	\[\begin{cases}
	x_1=-\dfrac{\ang{10;;}}{12}+k\ang{30;;}\\
	x_2=-\ang{165;;}+k\ang{180;;}
	\end{cases}\]
\end{exercise}
\begin{exercise}
	Trovare i valori dell'incognita per cui $\tan(6x-\ang{30;;})=-\tan(2x-\ang{40;;})$
	\tcblower
	 $\tan(6x-\ang{30;;})=-\tan(2x-\ang{40;;})=\tan(-2x+\ang{40;;})$
	
	Esistenza lato sinistro
	
	\begin{align*}
		6x-\ang{30;;}&=\ang{90;;}+k\ang{180;;}\\
		6x&=\ang{90;;}+\ang{30;;}+k\ang{180;;}\\
		6x&=\ang{120;;}+k\ang{180;;}\\
		x&=\ang{20;;}+k\ang{36;;}
		\end{align*}
	
	Esistenza lato destro
	
	\begin{align*}
		-2x+\ang{40;;}&=\ang{90;;}+k\ang{180;;}\\
		-2x&=\ang{90;;}-\ang{40;;}+k\ang{180;;}\\
		-2x&=\ang{50;;}+k\ang{180;;}\\
		x&=-\ang{25;;}+k\ang{90;;}
		\end{align*}
	Troviamo la soluzione
	
		\begin{align*}
		6x-\ang{30;;}&=-2x+\ang{40;;}+k\ang{180;;}\\
		6x+2x&=\ang{40;;}+\ang{30;;}+k\ang{180;;}\\
		8x&=\ang{70;;}+k\ang{180;;}\\
		x&=\dfrac{\ang{70;;}}{8}+k\dfrac{\ang{180;;}}{8}
		\end{align*}
	
	La soluzione è
	\[x=\dfrac{\ang{70;;}}{8}+k\dfrac{\ang{180;;}}{8}\]
\end{exercise}

 \tcbstoprecording
 \newpage
 \section{Soluzioni equazioni goniometriche}
 \tcbinputrecords