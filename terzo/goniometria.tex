\chapter{Goniometria}
\label{cha:goniometriaEss}
\section{Trovare seno coseno, tangente e cotangente esempi}\index{Seno}\index{Coseno}\index{Tangente}\index{Cotangente}
\begin{esempiot}{Trovare seno coseno e tangente}{exemplum1}
	Se $\sin\alpha=\dfrac{3}{5}$ con $\dfrac{\pi}{2}\leq\alpha\leq\pi$ determinare coseno, tangente cotangente.
\end{esempiot}
Se $\sin\alpha=\dfrac{3}{5}$ disegno la figura\nobs\vref{fig:esempio1}. Con il vincolo $\dfrac{\pi}{2}\leq\alpha\leq\pi$ il Punto $P$ è nel secondo quadrante, di conseguenza il coseno è negativo, come è negativa la tangente e la cotangente.
\begin{align*}
\sin\alpha=&\dfrac{3}{5}\\
\cos\alpha=&-\sqrt{1-\sin^2\alpha}=\\
=&-\sqrt{1-\dfrac{9}{25}}=\\
=&-\sqrt{\dfrac{25-9}{25}}=\\
=&-\sqrt{\dfrac{16}{25}}=\\
=&-\dfrac{4}{5}
\intertext{La tangente è:}
\tan\alpha=&\dfrac{\sin\alpha}{\cos\alpha}=\dfrac{\dfrac{3}{5}}{-\dfrac{4}{5}}=\dfrac{3}{5}\cdot\left(-\dfrac{5}{4}\right)=-\dfrac{3}{4}
\intertext{La cotangente è}
\cot\alpha=&\frac{1}{\tan\alpha}=-\frac{4}{3}
\end{align*}
\begin{esempiot}{Trovare seno coseno, tangente e cotangente}{}\index{Seno}\index{Coseno}\index{Tangente}\index{Cotangente}
	Se $cos\alpha=\dfrac{7}{9}$ con $\dfrac{3\pi}{2}\leq\alpha\leq 2\pi$ determinare seno, tangente e cotangente.
\end{esempiot}
L'angolo è nel quarto quadrante come lo mostra la figura\nobs\vref{fig:esempio2}, quindi seno e tangente sono negativi
\begin{align*}
\cos\alpha=&\dfrac{7}{9}\\
\sin\alpha=&-\sqrt{1-\cos^2\alpha}=\\
=&-\sqrt{1-\dfrac{49}{81}}=\\
=&-\sqrt{\dfrac{81-49}{81}}=\\
=&-\sqrt{\dfrac{32}{81}}=\\
=&-\dfrac{4\sqrt{2}}{9}
\intertext{La tangente è:}
\tan\alpha=&\dfrac{\sin\alpha}{\cos\alpha}=\dfrac{-\dfrac{4\sqrt{2}}{9}}{\dfrac{7}{9}}=-\dfrac{4\sqrt{2}}{9}\cdot\dfrac{9}{7}=-\dfrac{4\sqrt{2}}{7}
\intertext{La cotangente}
\cot\alpha=&\dfrac{1}{\tan\alpha}=-\dfrac{7}{4\sqrt{2}}=-\dfrac{7}{8}\sqrt{2}
\end{align*}
\begin{esempiot}{Trovare seno coseno, tangente e cotangente}{}\index{Seno}\index{Coseno}\index{Tangente}\index{Cotangente}
	Se $sin\alpha=-\dfrac{9}{10}$ con $\pi\leq\alpha\leq\frac{3\pi}{2}$ determinare coseno, tangente e cotangente.
\end{esempiot}
Se $\sin\alpha=-\dfrac{9}{10}$ disegno la figura\nobs\vref{fig:esempio3}. Con il vincolo $\pi\leq\alpha\leq\dfrac{3\pi}{2}$ il punto $P$ è nel terzo quadrante, di conseguenza il coseno è negativo, mentre  la tangente e la cotangente sono positive.
\begin{align*}
\sin\alpha=&-\dfrac{9}{10}\\
\cos\alpha=&-\sqrt{1-\sin^2\alpha}=\\
=&-\sqrt{1-\dfrac{81}{100}}=\\
=&-\sqrt{\dfrac{100-81}{100}}=\\
=&-\sqrt{\dfrac{19}{100}}=\\
=&-\dfrac{\sqrt{19}}{100}\\
\tan\alpha=&\dfrac{\sin\alpha}{\cos\alpha}=\dfrac{-\dfrac{9}{10}}{-\dfrac{\sqrt{19}}{10}}=\left(-\dfrac{9}{10}\right)\cdot\left(-\dfrac{\sqrt{19}}{10}\right)=\dfrac{9}{\sqrt{19}}=\dfrac{9\sqrt{19}}{19}\\
\cot\alpha=&\dfrac{1}{\tan\alpha}=\dfrac{\sqrt{19}}{9}
\end{align*}
\begin{esempiot}{Trovare seno coseno, tangente e cotangente}{}\index{Seno}\index{Coseno}\index{Tangente}\index{Cotangente}
	Se $cos\alpha=-\dfrac{5}{6}$ con $\dfrac{\pi}{2}\leq\alpha\leq \pi$ determinare seno e tangente.
\end{esempiot}
L'angolo è nel secondo quadrante come  mostra la figura\nobs\vref{fig:esempio4}, quindi seno è positivo e la tangente è negativa
\begin{align*}
\cos\alpha=&-\dfrac{5}{6}\\
\sin\alpha=&\sqrt{1-\cos^2\alpha}=\\
=&\sqrt{1-\dfrac{25}{36}}=\\
=&\sqrt{\dfrac{36-25}{36}}=\\
=&\sqrt{\dfrac{11}{36}}=\\
=&\dfrac{\sqrt{11}}{6}\\
\tan\alpha=&\dfrac{\sin\alpha}{\cos\alpha}=\dfrac{\dfrac{\sqrt{11}}{6}}{-\dfrac{5}{6}}=\dfrac{\sqrt{11}}{6}\cdot\left(-\dfrac{6}{5}\right)=-\dfrac{\sqrt{11}}{5}\\
\cot\alpha=&\dfrac{1}{\tan\alpha}=-\dfrac{5}{\sqrt{11}}=-\dfrac{5}{11}\sqrt{11}
\end{align*}
\begin{figure}
\begin{subfigure}[b]{.5\linewidth}
\centering
\includestandalone[width=.8\textwidth]{terzo/grafici/EquaElementareSeno1}
% \captionsetup{format=esempio}
\caption{Seno noto}\label{fig:esempio1}
\end{subfigure}%
\begin{subfigure}[b]{.5\linewidth}
\centering
\includestandalone[width=.8\textwidth]{terzo/grafici/EquaElementareCoseno1}
%\captionsetup{format=esempio}
\caption{Coseno noto}\label{fig:esempio2}
\end{subfigure}%
\quad
\begin{subfigure}[b]{.5\linewidth}
\centering
\includestandalone[width=.8\textwidth]{terzo/grafici/EquaElementareSeno2}
%\captionsetup{format=esempio}
\caption{Seno noto}\label{fig:esempio3}
\end{subfigure}%
\begin{subfigure}[b]{.5\linewidth}
\centering
\includestandalone[width=.8\textwidth]{terzo/grafici/EquaElementareCoseno2}
%\captionsetup{format=esempio}
\caption{Coseno noto}\label{fig:esempio4}
\end{subfigure}%
\quad
\begin{subfigure}[b]{.5\linewidth}
	\centering
	\includestandalone[width=.8\textwidth]{terzo/grafici/EquaElementareTangente1}
	\caption{Tangente nota}\label{fig:esempio5}
\end{subfigure}%
\begin{subfigure}[b]{.5\linewidth}
	\centering
	\includestandalone[width=.8\textwidth]{terzo/grafici/EquaElementareTangente2}
	\caption{Tangente nota}\label{fig:esempio6}
\end{subfigure}%
\caption{Trovare seno, coseno, tangente e cotangente}
\end{figure}
\begin{esempiot}{Trovare seno coseno e cotangente}{}\index{Seno}\index{Coseno}\index{Tangente}\index{Cotangente}
Se $\tan\alpha=\dfrac{1}{8}$ con $\pi\leq\alpha<\dfrac{3\pi}{2}$ determinare seno, coseno e cotangente.
\end{esempiot}
Come dalla figura\nobs\vref{fig:esempio5} l'angolo è nel terzo quadrante quindi seno e coseno sono negativi e la cotangente è positiva.
\begin{align*}
\cos\alpha=&-\dfrac{1}{\sqrt{1+\tan^2\alpha}}=-\dfrac{1}{\sqrt{1+\dfrac{1}{64}}}\\
=&-\dfrac{1}{\sqrt{\dfrac{64+1}{64}}}=-\dfrac{1}{\sqrt{\dfrac{65}{64}}}\\
=&-\dfrac{1}{\dfrac{\sqrt{65}}{8}}=-\dfrac{8}{\sqrt{65}}=-\dfrac{8\sqrt{65}}{65}\\
\sin\alpha=&\tan\alpha\cdot\cos\alpha\\
=&\dfrac{1}{8}\cdot\left(-\dfrac{8\sqrt{65}}{65}\right)=-\dfrac{\sqrt{65}}{65}\\
\cot\alpha=&\dfrac{1}{\tan\alpha}=8\\
\end{align*}
\begin{esempiot}{Trovare seno coseno e cotangente}{}
	Se $\tan\alpha=\dfrac{3}{5}$ con $\pi\leq\alpha<\dfrac{3\pi}{2}$ determinare seno, coseno e cotangente.
\end{esempiot}
Come dalla figura\nobs\vref{fig:esempio6} l'angolo è nel terzo quadrante quindi seno e coseno sono negativi.
\begin{align*}
\cos\alpha=&-\dfrac{1}{\sqrt{1+\tan^2\alpha}}=-\dfrac{1}{\sqrt{1+\dfrac{9}{25}}}\\
=&-\dfrac{1}{\sqrt{\dfrac{25+9}{25}}}=-\dfrac{1}{\sqrt{\dfrac{34}{25}}}\\
=&-\dfrac{1}{\dfrac{\sqrt{34}}{5}}=-\dfrac{5}{\sqrt{34}}\\
=&-\dfrac{5\sqrt{34}}{34}\\
\sin\alpha=&\tan\alpha\cdot\cos\alpha\\
=&\left(\dfrac{3}{5}\right)\cdot\left(-\dfrac{5\sqrt{34}}{34}\right)\\
=&-\dfrac{3\sqrt{34}}{34}\\
\cot\alpha=&\dfrac{1}{\tan\alpha}=\dfrac{25}{9}
\end{align*}
\begin{figure}
	\begin{subfigure}[b]{.5\linewidth}
		\centering
		\includestandalone[width=.8\textwidth]{terzo/grafici/EquaElementareCotangente1}
		\captionsetup{format=esempio}
		\caption{Cotangente nota}\label{fig:esempio7}
	\end{subfigure}%
	\begin{subfigure}[b]{.5\linewidth}
		\centering
		\includestandalone[width=.8\textwidth]{terzo/grafici/EquaElementareCotangente2}
		\captionsetup{format=esempio}
		\caption{Cotangente nota}\label{fig:esempio8}
	\end{subfigure}%
	\quad
\caption{Trovare seno, coseno, tangente e cotangente}
\end{figure}
\begin{esempiot}{Trovare seno coseno e tangente}{}
	Se $\cot\alpha=\dfrac{7}{4}$ con $\pi\leq\alpha<\dfrac{3\pi}{2}$ determinare seno e coseno e tangente.
\end{esempiot}
Come dalla figura\nobs\vref{fig:esempio7} l'angolo è nel terzo quadrante quindi seno e coseno sono negativi e tangente positiva.
\begin{align*}
\tan\alpha=&\dfrac{1}{\cot\alpha}=\dfrac{4}{7}\\
\cos\alpha=&-\dfrac{1}{\sqrt{1+\tan^2\alpha}}=-\dfrac{1}{\sqrt{1+\dfrac{16}{49}}}\\
=&-\dfrac{1}{\sqrt{\dfrac{49+16}{49}}}=-\dfrac{1}{\sqrt{\dfrac{65}{49}}}\\
=&-\dfrac{1}{\dfrac{\sqrt{65}}{7}}=-\dfrac{7}{\sqrt{65}}\\
=&-\dfrac{7\sqrt{65}}{65}\\
\sin\alpha=&\tan\alpha\cdot\cos\alpha\\
=&\left(\dfrac{4}{7}\right)\cdot\left(-\dfrac{7\sqrt{65}}{65}\right)\\
=&-\dfrac{4\sqrt{65}}{65}\\ 
\end{align*}
\begin{esempiot}{Trovare seno coseno e tangente}{}
	Se $\cot\alpha=-\dfrac{3}{5}$ con $\pi\leq\alpha<\dfrac{\pi}{2}$ determinare seno e coseno e tangente.
\end{esempiot}
Come dalla figura\nobs\vref{fig:esempio8} l'angolo è nel secondo quadrante quindi seno positivo e coseno negativo e tangente negativa.
\begin{align*}
\tan\alpha=&\dfrac{1}{\cot\alpha}=-\dfrac{5}{3}\\
\cos\alpha=&-\dfrac{1}{\sqrt{1+\tan^2\alpha}}=-\dfrac{1}{\sqrt{1+\dfrac{25}{9}}}\\
=&-\dfrac{1}{\sqrt{\dfrac{9+25}{9}}}=-\dfrac{1}{\sqrt{\dfrac{34}{9}}}\\
=&-\dfrac{1}{\dfrac{\sqrt{34}}{3}}=-\dfrac{3}{\sqrt{34}}\\
=&-\dfrac{3\sqrt{34}}{34}\\
\sin\alpha=&\tan\alpha\cdot\cos\alpha\\
=&\left(-\dfrac{5}{3}\right)\cdot\left(-\dfrac{3\sqrt{34}}{34}\right)\\
=&+
\dfrac{5\sqrt{34}}{34}\\ 
\end{align*}