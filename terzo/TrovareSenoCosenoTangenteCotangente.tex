\tcbstartrecording
\chapter{Formule derivate esercizi}
%\begin{exercise}[no solution]
%	Disegnare nel piano complesso il numero $z=+2\uimm$
%\end{exercise}
%\begin{exercise}
%	Convertire $\alpha=\ang{45;58;25}$ in radianti\index{Radianti}
%	\tcblower
%	Convertire $\alpha=\ang{45;58;25}$ in radianti\index{Radianti}
%	\begin{align*}
%	\alpha=&\ang{45}+\left(\dfrac{58}{60}\right)^{\si{\degree} }+\left(\dfrac{25}{3600}\right)^{\si{\degree} }\\
%	=&\ang{45}+\left(\dfrac{58\cdot60+25}{3600}\right)^{\si{\degree}}\\
%	=&\ang{45}+\left(\dfrac{3505}{3600}\right)^{\si{\degree}}\approx\ang[round-precision=\lungarrotandamento,round-mode=places]{45,97361111}
%	\end{align*}
%	\[\rho=\dfrac{\pi}{180}\alpha\approx\dfrac{\pi}{180}\cdot\ang[round-precision=\lungarrotandamento,round-mode=places]{45,97361111}\approx\SI[round-precision=\lungarrotandamento,round-mode=places]{0.802390882}{\radian}\approx\dfrac{1277}{5000}\pi\;\si{\radian}\]
%	%\SI[quotient-mode=fraction]{51/200}{\radian}\]
%\end{exercise}
\section{Funzione nota}
\begin{exercise}
Se $\cot\alpha=-5$ con $\ang{90}\leq\alpha\leq\ang{180}$ determinare seno, coseno, tangente.
\tcblower
Se $\cot\alpha=-5$ con $\ang{90}\leq\alpha\leq\ang{180}$ determinare seno, coseno, tangente.

Secondo quadrante quindi seno positivo, coseno negativo, tangente negativa.
\begin{align*}
\tan\alpha=&\dfrac{1}{\cot\alpha}=-\dfrac{1}{5}\\
\cos\alpha=&-\dfrac{1}{\sqrt{1+\tan^2\alpha}}=-\dfrac{1}{\sqrt{1+\dfrac{1}{25}}}\\
=&-\dfrac{1}{\sqrt{\dfrac{25+1}{25}}}=-\dfrac{1}{\dfrac{\sqrt{26}}{5}}\\
=&-\dfrac{5}{26}=-\dfrac{5\sqrt{26}}{26}\\
\sin\alpha=&\tan\alpha\cos\alpha=\left(-\dfrac{5\sqrt{26}}{26}\right)\cdot\left(-\dfrac{1}{5}\right)\\
=&\dfrac{\sqrt{26}}{26}
\end{align*}
\end{exercise}
\begin{exercise}[no solution]
Se $\cos\alpha=-\dfrac{1}{7}$ con $\pi\leq\alpha\leq\dfrac{3\pi}{2}$ determinare seno, tangente cotangente.
\end{exercise}
\begin{exercise}[no solution]
	Se $\tan\alpha=\dfrac{\sqrt{3}}{2}$ con $0\leq\alpha\leq\dfrac{\pi}{2}$ determinare seno, coseno cotangente.
\end{exercise}
\begin{exercise}
	Se $\tan\alpha=4$ con $\pi\leq\alpha\leq\dfrac{3\pi}{2}$ determinare seno, coseno cotangente.
\tcblower	
	Se $\tan\alpha=4$ con $\pi\leq\alpha\leq\dfrac{3\pi}{2}$ determinare seno, coseno cotangente.
	
Terzo quadrante quindi seno negativo, coseno negativo, tangente positiva.
\begin{align*}
\cot\alpha=&\dfrac{1}{\tan\alpha}=\dfrac{1}{4}\\
\cos\alpha=&-\dfrac{1}{\sqrt{1+\tan^2\alpha}}=-\dfrac{1}{\sqrt{1+16}}\\
=&-\dfrac{1}{\sqrt{17}}=-\dfrac{\sqrt{17}}{17}\\
\sin\alpha=&\tan\alpha\cos\alpha=4\cdot\left(-\dfrac{\sqrt{17}}{17}\right)\\
=&\dfrac{4\sqrt{17}}{17}
\end{align*}
\end{exercise}
\begin{exercise}[no solution]
	Se $\cot\alpha=5$ con $\ang{0}\leq\alpha\leq\ang{90}$ determinare seno, coseno, tangente.
\end{exercise}
\begin{exercise}[no solution]
	Se $\cos\alpha=-\dfrac{3}{5}$ con $\dfrac{\pi}{2}\leq\alpha\leq\pi$ determinare seno, tangente cotangente.
\end{exercise}
\begin{exercise}[no solution]
	Se $\sin\alpha=\dfrac{3}{5}$ con $\ang{0}\leq\alpha\leq\ang{90}$ determinare coseno, tangente cotangente.
\end{exercise}
\begin{exercise}
	Se $\cos\alpha=\dfrac{1}{2}$ con $0\leq\alpha\leq\dfrac{\pi}{2}$ determinare seno, tangente, cotangente.
\tcblower	
	Se $\cos\alpha=\dfrac{1}{2}$ con $0\leq\alpha\leq\dfrac{\pi}{2}$ determinare seno, tangente, cotangente.
	
Primo quadrante quindi seno positivo, tangente positiva cotangente positiva.
\begin{align*}
\sin\alpha=&\sqrt{1-\cos^2\alpha}=\sqrt{1-\dfrac{1}{4}}\\
=&\sqrt{\dfrac{4-1}{4}}=\dfrac{\sqrt{3}}{2}\\
\tan\alpha=&\dfrac{\sin\alpha}{\cos\alpha}=\dfrac{\dfrac{\sqrt{2}}{2}}{\dfrac{1}{2}}\\
&=\dfrac{\sqrt{2}}{2}\cdot 2=\sqrt{2}\\
\cot\alpha=&\dfrac{1}{\tan\alpha}=\dfrac{1}{\sqrt{2}}\\
=&\dfrac{\sqrt{2}}{2}
\end{align*}	
\end{exercise}
\begin{exercise}[no solution]
	Se $\cot\alpha=-4$ con $\dfrac{3}{2}\pi\leq\alpha\leq 2\pi$ determinare coseno, seno, tangente.
\end{exercise}
\begin{exercise}[no solution]
	Se $\sin\alpha=\dfrac{4}{7}$ con $\dfrac{\pi}{2}\leq\alpha\leq\pi$ determinare  coseno, tangente, cotangente.
\end{exercise}
\begin{exercise}[no solution]
	Se $\tan\alpha=-5$ con $\dfrac{\pi}{2}\leq\alpha\leq\pi$ determinare seno, coseno, cotangente.
\end{exercise}
\begin{exercise}
	Se $\sin\alpha=-\dfrac{4}{5}$ con $\ang{180}\leq\alpha\leq\ang{270}$ determinare  coseno, tangente, cotangente.
	\tcblower	
Se $\sin\alpha=-\dfrac{4}{5}$ con $\ang{180}\leq\alpha\leq\ang{270}$ determinare  coseno, tangente, cotangente.
	
Terzo quadrante quindi coseno negativo, tangente positiva cotangente positiva.
	\begin{align*}
	\cos\alpha=&\sqrt{1-\sin^2\alpha}=\sqrt{1-\dfrac{16}{25}}\\
	=&-\sqrt{\dfrac{25-16}{25}}=\sqrt{\dfrac{9}{25}}\\
	=&-\dfrac{3}{5}
	\tan\alpha=&\dfrac{\sin\alpha}{\cos\alpha}=\dfrac{-\dfrac{4}{5}}{-\dfrac{3}{5}}\\
	&=\left(-\dfrac{4}{5}\right)\cdot\left(-\dfrac{5}{3}\right)=\dfrac{4}{3}
	\cot\alpha=&\dfrac{1}{\tan\alpha}=\dfrac{1}{\dfrac{4}{3}}\\
	=&\dfrac{3}{4}
	\end{align*}	
\end{exercise}
\begin{exercise}[no solution]
	Se $\sin\alpha=-\dfrac{3}{8}$  con $\pi\leq\alpha\leq\dfrac{3\pi}{2}$ determinare coseno, tangente cotangente.
\end{exercise}
\begin{exercise}[no solution]
	Se $\cos\alpha=\dfrac{7}{8}$  con $\ang{270}\leq\alpha\leq\ang{360}$ determinare seno, tangente cotangente.
\end{exercise}
\begin{exercise}[no solution]
	Se $\cot\alpha=3$ con $\ang{180}\leq\alpha\leq\ang{270}$ determinare seno, coseno, tangente.
\end{exercise}
\begin{exercise}
	Se $\tan\alpha=-\sqrt{3}$ con $\ang{270}\leq\alpha\leq\ang{360}$ determinare seno, coseno, cotangente.
\tcblower	
	Se $\tan\alpha=-\sqrt{3}$ con $\ang{270}\leq\alpha\leq\ang{360}$ determinare seno, coseno, cotangente.

Quarto quadrante quindi seno negativo, coseno positivo, cotangente negativa.
\begin{align*}
\cot\alpha=&\dfrac{1}{\tan\alpha}=-\dfrac{1}{\sqrt{3}}\\
=&-\dfrac{\sqrt{3}}{3}\\
\cos\alpha=&-\dfrac{1}{\sqrt{1+\tan^2\alpha}}=-\dfrac{1}{\sqrt{1+3}}\\
=&-\dfrac{1}{\sqrt{4}}=-\dfrac{1}{2}\\
\sin\alpha=&\tan\alpha\cos\alpha=\dfrac{1}{2}\cdot\left(-\sqrt{3}\right)\\
=&-\dfrac{\sqrt{3}}{2}
\end{align*}	
\end{exercise}
\tcbstoprecording
\newpage
\section{Soluzioni seno coseno tangente}
\tcbinputrecords
