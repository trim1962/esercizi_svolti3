\section{Modulo e argomento}
\begin{esempiot}{Trovare modulo ed argomento di un numero complesso}{numcomarg1}\index{Numero!complesso!modulo}\index{Numero!complesso!argomento}
	Dato  il numero complesso \[z=2+3\uimm\] determinarne modulo $\abs{z}$ ed l'argomento $\phi$ con $\ang{0}\leq\phi<\ang{360}$
\end{esempiot}
Rappresento il numero complesso nel piano di Gauss
\begin{center}
	\includestandalone[width=.6\textwidth]{terzo/grafici/argomentocomplesso1}
	\captionof{figure}{Modulo ed argomento uno}\label{fig:moduloargomentouno}
\end{center}
Formule coinvolte:
\begin{align*}
\abs{z}=&\sqrt{a^2+b^2}\\
\phi=&\arctan\left(\dfrac{b}{a}\right)
\end{align*}
Quindi per il modulo
\begin{align*}
z=&2+3\uimm\\
\abs{z}=&\sqrt{2^2+3^2}\\
=&\sqrt{4+9}\\
=&\sqrt{13}\\
\simeq&\num[round-precision=\lungarrotandamento,round-mode=places]{3.6055551275}
\end{align*}
Utilizzando la calcolatrice
 \begin{center}
	\begin{tabular}{ll}
		\tasto{2}\tastoquadrato\tastopiu\tasto{3}\tastoquadrato\tastouguale&13\\
	\tastoradicequadrata\tastoans\tastouguale&$\simeq\num[round-precision=\lungarrotandamento,round-mode=places]{3.6055551275}$
		\end{tabular}
\end{center}
Per l'argomento

Verifico la calcolatrice \testgradi
\begin{align*}
z=&2+3\uimm\\
\phi=&\arctan\left(\dfrac{3}{2}\right)\\
\simeq&\ang[round-precision=\lungarrotandamento,round-mode=places]{56.30993247}
\end{align*}
Utilizzando la calcolatrice
\begin{center}
	\begin{tabular}{ll}
		\tasto{3}\tastodiv\tasto{2}\tastouguale&1.5\\
\tastoitan\tastoans\tastouguale&$\simeq\ang[round-precision=\lungarrotandamento,round-mode=places]{56.30993247}$
	\end{tabular}
\end{center}

\begin{esempiot}{Trovare modulo ed l'argomento di un numero complesso}{}
	Dato  il numero complesso \[z=-2+3\uimm\] determinarne modulo $\abs{z}$ ed argomento $\phi$ con $\ang{0}\leq\phi<\ang{360}$
\end{esempiot}
Rappresento il numero complesso nel piano di Gauss
\begin{center}
	\includestandalone[width=0.6\textwidth]{terzo/grafici/argomentocomplesso2}
	\captionof{figure}{Modulo ed argomento due}\label{fig:moduloargomentodue}
\end{center}
Formule coinvolte:
\begin{align*}
\abs{z}=&\sqrt{a^2+b^2}\\
\phi=&\arctan\left(\dfrac{b}{a}\right)
\end{align*}
Quindi per il modulo
\begin{align*}
z=&-2+3\uimm\\
\abs{z}=&\sqrt{(-2)^2+3^2}\\
=&\sqrt{4+9}\\
=&\sqrt{13}\\
\simeq&\num[round-precision=\lungarrotandamento,round-mode=places]{3.6055551275}
\end{align*}
Utilizzando la calcolatrice
\begin{center}
	\begin{tabular}{ll}
		\tastoparentesisin\tasto{-2}\tastoparentesides\tastoquadrato\tastopiu\tasto{3}\tastoquadrato\tastouguale&13\\
		\tastoradicequadrata\tastoans\tastouguale&$\simeq\num[round-precision=\lungarrotandamento,round-mode=places]{3.6055551275}$
	\end{tabular}
\end{center}
Per l'argomento

Verifico la calcolatrice \testgradi
\begin{align*}
z=&-2+3\uimm\\
\phi=&\arctan\left(-\dfrac{3}{2}\right)\\
\simeq&\ang[round-precision=\lungarrotandamento,round-mode=places]{-56.30993247}
\intertext{Correggo}
\simeq&\ang{180}\ang[round-precision=\lungarrotandamento,round-mode=places]{-56.30993247}\\
\simeq&\ang[round-precision=\lungarrotandamento,round-mode=places]{123.6900675}
\end{align*}
Utilizzando la calcolatrice
\begin{center}
	\begin{tabular}{ll}
		\tasto{3}\tastodiv\tastoparentesisin\tasto{-2}\tastoparentesides\tastouguale&\num{-1.5}\\
		\tastoitan\tastoans\tastouguale&$\simeq\ang[round-precision=\lungarrotandamento,round-mode=places]{-56.30993247}$\\
		\tasto{180}\tastopiu\tastoans\tastouguale&$\simeq\ang[round-precision=\lungarrotandamento,round-mode=places]{123.6900675}$
	\end{tabular}
\end{center}
\begin{esempiot}{Trovare modulo ed l'argomento di un numero complesso}{}
	Dato  il numero complesso \[z=-2-3\uimm\] determinarne modulo $\abs{z}$ ed argomento $\phi$ con $\ang{0}\leq\phi<\ang{360}$
\end{esempiot}
Rappresento il numero complesso nel piano di Gauss
\begin{center}
	\includestandalone[width=.6\textwidth]{terzo/grafici/argomentocomplesso3}
	\captionof{figure}{Modulo ed argomento tre}\label{fig:moduloargomentotre}
\end{center}
Formule coinvolte:
\begin{align*}
\abs{z}=&\sqrt{a^2+b^2}\\
\phi=&\arctan\left(\dfrac{b}{a}\right)
\end{align*}
Quindi per il modulo
\begin{align*}
z=&-2-3\uimm\\
\abs{z}=&\sqrt{(-2)^2+(-3)^2}\\
=&\sqrt{4+9}\\
=&\sqrt{13}\\
\simeq&\num[round-precision=\lungarrotandamento,round-mode=places]{3.6055551275}
\end{align*}
Utilizzando la calcolatrice
\begin{center}
	\begin{tabular}{ll}
		\tastoparentesisin\tasto{-2}\tastoparentesides\tastoquadrato\tastopiu\tastoparentesisin\tasto{-3}\tastoparentesides\tastoquadrato\tastouguale&13\\
		\tastoradicequadrata\tastoans\tastouguale&$\simeq\num[round-precision=\lungarrotandamento,round-mode=places]{3.6055551275}$
	\end{tabular}
\end{center}
Per l'argomento

Verifico la calcolatrice \testgradi
\begin{align*}
z=&-2-3\uimm\\
\phi=&\arctan\left(\dfrac{-3}{-2}\right)\\
\simeq&\ang[round-precision=\lungarrotandamento,round-mode=places]{56.30993247}
\intertext{correggo}
\simeq&\ang{180}+\ang[round-precision=\lungarrotandamento,round-mode=places]{56.30993247}\\
\simeq&\ang[round-precision=\lungarrotandamento,round-mode=places]{236.30993247}
\end{align*}
Utilizzando la calcolatrice
\begin{center}
	\begin{tabular}{ll}
		\tasto{-3}\tastodiv\tasto{-2}\tastouguale&1.5\\
		\tastoitan\tastoans\tastouguale&$\simeq\ang[round-precision=\lungarrotandamento,round-mode=places]{56.30993247}$\\
		\tasto{180}\tastopiu\tastoans\tastouguale&$\simeq\ang[round-precision=\lungarrotandamento,round-mode=places]{236.3099325}$\\
	\end{tabular}
\end{center}

\begin{esempiot}{Trovare modulo ed l'argomento di un numero complesso}{}
	Dato  il numero complesso \[z=2-3\uimm\] determinarne modulo $\abs{z}$ ed argomento $\phi$ con $\ang{0}\leq\phi<\ang{360}$
\end{esempiot}
Rappresento il numero complesso nel piano di Gauss
\begin{center}
	\includestandalone[width=.6\textwidth]{terzo/grafici/argomentocomplesso4}
	\captionof{figure}{Modulo ed argomento quattro}\label{fig:moduloargomentoquattro}
\end{center}
Formule coinvolte:
\begin{align*}
\abs{z}=&\sqrt{a^2+b^2}\\
\phi=&\arctan\left(\dfrac{b}{a}\right)
\end{align*}
Quindi per il modulo
\begin{align*}
z=&2-3\uimm\\
\abs{z}=&\sqrt{2^2+(-3)^2}\\
=&\sqrt{4+9}\\
=&\sqrt{13}\\
\simeq&\num[round-precision=\lungarrotandamento,round-mode=places]{3.6055551275}
\end{align*}
Utilizzando la calcolatrice
\begin{center}
	\begin{tabular}{ll}
		\tasto{2}\tastoquadrato\tastopiu\tastoparentesisin\tasto{-3}\tastoparentesides\tastoquadrato\tastouguale&13\\
		\tastoradicequadrata\tastoans\tastouguale&$\simeq\num[round-precision=\lungarrotandamento,round-mode=places]{3.6055551275}$
	\end{tabular}
\end{center}
Per l'argomento

Verifico la calcolatrice \testgradi
\begin{align*}
z=&2-3\uimm\\
\phi=&\arctan\left(\dfrac{-3}{2}\right)\\
\simeq&\ang[round-precision=\lungarrotandamento,round-mode=places]{-56.30993247}
\intertext{correggo}
\simeq&\ang{360}\ang[round-precision=\lungarrotandamento,round-mode=places]{-56.30993247}\\
\simeq&\ang[round-precision=\lungarrotandamento,round-mode=places]{303.6900675}
\end{align*}
Utilizzando la calcolatrice
\begin{center}
	\begin{tabular}{ll}
		\tasto{-3}\tastodiv\tasto{2}\tastouguale&-1.5\\
		\tastoitan\tastoans\tastouguale&$\simeq\ang[round-precision=\lungarrotandamento,round-mode=places]{-56.30993247}$\\
			\tasto{360}\tastopiu\tastoans\tastouguale&$\simeq\ang[round-precision=\lungarrotandamento,round-mode=places]{303.6900675}$\\
		\end{tabular}
\end{center}