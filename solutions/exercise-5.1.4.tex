$\tan x=\num[round-precision=\lungarrotandamento,round-mode=places]{1.414213562}$

 Controllare che la calcolatrice sia impostata in gradi sessagesimali\index{Grado!Sessagesimale}.
 Basta verificare che

\testgradi

In caso contrario modificare le impostazioni.

Le soluzioni sono \[x_1=\alpha+k\ang{180}\]

Converto in gradi sessagesimali\index{Grado!Sessagesimale} $x_1$
 \begin{center}
 \begin{tabular}{ll}
 \tastoitan\tasto{\num[round-precision=\lungarrotandamento,round-mode=places]{1.414213562}}
 \tastouguale&\SI[round-precision=\lungarrotandamento,round-mode=places]{54.73561032}{\si{\degree}}\\
 \end{tabular}
\end{center}

 Converto in gradi sessagesimali\index{Grado!Sessagesimale} $x_1$

 \begin{center}
 \begin{tabular}{ll}
 \tastoans\tastomeno\tasto{54}\tastouguale&\SI[round-precision=\lungarrotandamento,round-mode=places]{0.735610317}{\si{\degree}}\\
 \tastoans\tastoper\tasto{60}\tastouguale&\SI[round-precision=\lungarrotandamento,round-mode=places]{44.13661903}{\si{\arcminute}}\\
 \tastoans\tastomeno\tasto{44}\tastouguale&\SI[round-precision=\lungarrotandamento,round-mode=places]{0.136619034}{\si{\arcminute}}\\
 \tastoans\tastoper\tasto{60}\tastouguale&\SI[round-precision=\lungarrotandamento,round-mode=places]{8.197142083}{\si{\arcsecond}}\\
 \end{tabular}
 \end{center}
Le soluzioni sono \[x_1=\ang{54;44;8}+k\ang{180}\]
 
