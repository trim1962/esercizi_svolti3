	La retta $7x+6y+4=0$ incontra l'asse $y$ nel punto $A$. Trovare la retta perpendicolare e la retta parallela alla retta $8x+5y+1=0$ che passano per $A$.
	Troviamo le coordinate del punto $A$ scrivo la retta  $7x+6y+4=0$ in forma esplicita
	\begin{align*}
	6y=&-7x-4\\
	y=&-\dfrac{7}{6}x-\dfrac{4}{6}\\
	y=&-\dfrac{7}{6}x-\dfrac{2}{3}\\
	\end{align*}
	Quindi $q=-\dfrac{2}{3}$ per cui $A\coord{0}{-\dfrac{2}{3}}$. Per trovare la retta perpendicolare a $8x+5y+1=0$ devo calcolarne il coefficiente angolare.
	Scrivo la retta in forma esplicita.
		\begin{align*}
		5y=&-8x-1\\
		y=&-\dfrac{8}{5}x-\dfrac{1}{5}\\
		\end{align*}
	$m=-\dfrac{8}{5}$ utilizzando la formula \[m_1\cdot m_2=-1\] ottengo
	\begin{align*}
-\dfrac{8}{5}\cdot m_2=&-1\\
\dfrac{8}{5}\cdot m_2=&1\\
 m_2=&\dfrac{5}{8}\\
	\end{align*}
	Utilizzando l'equazione del fascio di rette, il valore di $m_2$ trovato e le coordinate di $A$ ottengo:
	\begin{align*}
	y+\dfrac{2}{5}=&\dfrac{5}{8}x\\
		y=&\dfrac{5}{8}x-\dfrac{2}{5}
	\end{align*}
	Cioè l'equazione della retta perpendicolare cercata.

	Due rette sono parallele se hanno lo stesso coefficiente angolare quindi \[m_1=m_2 \]
	quindi $m_2=-\dfrac{8}{5}$

	Utilizzando l'equazione del fascio di rette, il valore di $m_2$ trovato e le coordinate di $A$ ottengo:
		\begin{align*}
		y+\dfrac{2}{5}=&-\dfrac{8}{5}x\\
		y=&-\dfrac{8}{5}x-\dfrac{2}{5}
		\end{align*}
	Cioè l'equazione della retta parallela cercata.
	\begin{center}
\includestandalone[width=.5\textwidth]{terzo/grafici/retta_dis_13}
%\captionof{figure}{Grafico}\label{fig:EsRiedistanza13}
\end{center}
