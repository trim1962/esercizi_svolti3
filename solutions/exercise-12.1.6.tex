	Data la retta $-3x+14y-13=0$, scrivi l'equazione della retta parallela e della retta perpendicolare che passano per  $A\coord{-5}{-2}$

	Scrivo la retta in forma esplicita
	\begin{align*}
		-3x+14y-13=&0\\
		14y=&+3x+13\\
		y=&\dfrac{3}{13}x+\dfrac{13}{14}
	\end{align*}
	 	$m=\dfrac{3}{14}$ utilizzando la formula \[m_1\cdot m_2=-1\] ottengo
	 	\begin{align*}
	 		\dfrac{3}{14}\cdot m_2=&-1\\
	 		m_2=&-\dfrac{14}{3}\\
	 	\end{align*}
	 		Utilizzando l'equazione del fascio di rette, il valore di $m_2$ trovato e le coordinate di $A$ ottengo:
	 		\begin{align*}
	 			y+2=&-\dfrac{14}{3}(x+5)\\
	 			y=&-\dfrac{14}{3}x-\dfrac{70}{3}-2\\
	 			y=&-\dfrac{14}{3}x-\dfrac{76}{3}\\
	 		\end{align*}
	 		Cioè l'equazione della retta perpendicolare cercata.

	 		Due rette sono parallele se hanno lo stesso coefficiente angolare quindi \[m_1=m_2 \]
	 		quindi $m_2=\dfrac{3}{14}$

	 		Utilizzando l'equazione del fascio di rette, il valore di $m_2$ trovato e le coordinate di $A$ ottengo:
	 		\begin{align*}
	 			y+2=&\dfrac{3}{14}(x+5)\\
	 			y=&\dfrac{3}{14}x+\dfrac{15}{14}-2\\
	 			y=&-\dfrac{3}{14}x-\dfrac{13}{14}\\
	 		\end{align*}
	 		Cioè l'equazione della retta parallela cercata.

	 			\begin{center}
	 				\includestandalone[width=.5\textwidth]{terzo/grafici/retta_dis_16}
	 				%\captionof{figure}{Grafico}\label{fig:EsRiedistanza13}
	 			\end{center}
