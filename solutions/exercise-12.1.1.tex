Calcolo la distanza tra $AB$ come con l'\cref{exa:disuno}
quindi \[d(AB)=\abs{-3-3}=\abs{-6}=6\] Ora calcoliamo la distanza tra $BC$ come con \cref{exa:discinque}
\begin{align*}
d(BC)=&\sqrt{(3)^2+(2+3)^2}\\
=&\sqrt{34}
\end{align*}
Ora calcoliamo la distanza tra $AC$ come con \cref{exa:discinque}
\begin{align*}
d(AC)=&\sqrt{(-3)^2+(2+3)^2}\\
=&\sqrt{34}
\end{align*}
Conseguentemente il perimetro è
\[2P=6+\sqrt{34}+\sqrt{34}=6+2\sqrt{34}\]
\begin{center}
	\includestandalone[width=.5\textwidth]{terzo/grafici/retta_dis_11}
	\captionof{figure}{Perimetro triangolo}\label{fig:EsRieDistanza11}
\end{center}
