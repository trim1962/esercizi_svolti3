 $\sin x=\num[round-precision=2,round-mode=places]{-0.75}$

 Controllare che la calcolatrice sia impostata in gradi sessagesimali\index{Grado!Sessagesimale}\index{Seno}.

 Basta verificare che
 \testgradi

 In caso contrario modificare le impostazioni.

 Le soluzioni sono
 \[\begin{cases}
 x_1=\alpha+k\ang{360}\\
 x_2=\ang{180}-\alpha+k\ang{360}\\
 \end{cases}\]
 Calcolo $\alpha$

 \begin{center}
 \begin{tabular}{ll}
 \tastoisin\tasto{\num[round-precision=2,round-mode=places]{-0.75}}\tastouguale&\SI[round-precision=\lungarrotandamento,round-mode=places]{-48.59037789}{\si{\degree}}
 \end{tabular}
 \end{center}

 \[\begin{cases}
 x_1=\alpha+k\ang{360}=\SI[round-precision=\lungarrotandamento,round-mode=places]{-48.59037789}{\si{\degree}}+k\ang{360}=\SI[round-precision=\lungarrotandamento,round-mode=places]{311.4096221}{\si{\degree}}+k\ang{360}\\
 x_2=-\ang{180}+\alpha+k\ang{360}=\SI[round-precision=\lungarrotandamento,round-mode=places]{228.5903789}{\si{\degree}}+k\ang{360}\\
 \end{cases}\]

 Converto in gradi sessagesimali\index{Grado!Sessagesimale} $x_1$

 \begin{center}
 \begin{tabular}{ll}
 \tastoans\tastomeno\tasto{311}\tastouguale&\SI[round-precision=\lungarrotandamento,round-mode=places]{0.409622109}{\si{\degree}}\\
 \tastoans\tastoper\tasto{60}\tastouguale&\SI[round-precision=\lungarrotandamento,round-mode=places]{24.57732655}{\si{\arcminute}}\\
 \tastoans\tastomeno\tasto{48}\tastouguale&\SI[round-precision=\lungarrotandamento,round-mode=places]{0.577326552}{\si{\arcminute}}\\
 \tastoans\tastoper\tasto{60}\tastouguale&\SI[round-precision=\lungarrotandamento,round-mode=places]{34.63959312}{\si{\arcsecond}}\\
 \end{tabular}
 \end{center}
 \[x_1=\ang{311;24;34}\]

 Converto in gradi sessagesimali\index{Grado!Sessagesimale} $x_2$

 \begin{center}
 \begin{tabular}{ll}
 \tastoans\tastomeno\tasto{228}\tastouguale&\SI[round-precision=\lungarrotandamento,round-mode=places]{0.59037789}{\si{\degree}}\\
 \tastoans\tastoper\tasto{60}\tastouguale&\SI[round-precision=\lungarrotandamento,round-mode=places]{35.42267345}{\si{\arcminute}}\\
 \tastoans\tastomeno\tasto{48}\tastouguale&\SI[round-precision=\lungarrotandamento,round-mode=places]{0.422673448}{\si{\arcminute}}\\
 \tastoans\tastoper\tasto{60}\tastouguale&\SI[round-precision=\lungarrotandamento,round-mode=places]{25.36040688}{\si{\arcsecond}}\\
 \end{tabular}
 \end{center}
 \[x_2=\ang{228;35;25}\]

 le soluzioni sono quindi
 \[\begin{cases}
 x_1=\ang{311;24;34}+k\ang{360}\\
 x_2=\ang{228;35;25}+k\ang{360}\\
 \end{cases}\]
 
