$\sin(6x-\dfrac{\pi}{3})=\sin(4x +\dfrac{2}{3}\pi)$

	Le soluzioni sono
	\[\begin{cases}
	x_1=+\alpha+2k\pi\\
	x_2=\pi-\alpha+2k\pi\\
	\end{cases}\]
	Troviamo la prima

	\begin{align*}
	6x-\dfrac{\pi}{3}&=4x +\dfrac{2}{3}\pi+2k\pi\\
6x-4x&=\dfrac{\pi}{3} +\dfrac{2}{3}\pi+2k\pi\\
	2x&=\pi+2k\pi\\
	x&=\dfrac{\pi}{2}+k\pi
	\end{align*}
	Troviamo la seconda

	\begin{align*}
		6x-\dfrac{\pi}{3}&=\pi-4x-\dfrac{2}{3}\pi+2k\pi\\
		6x+4x&=\pi+\dfrac{\pi}{3}-\dfrac{2}{3}\pi+2k\pi\\
		10x&=\dfrac{2}{3}\pi+2k\pi\\
		x&=\dfrac{2}{30}\pi+k\dfrac{1}{5}\pi
	\end{align*}

	Le soluzioni sono

	\[\begin{cases}
		x=\dfrac{\pi}{2}+k\pi\\
		\\
	x=\dfrac{2}{30}\pi+k\dfrac{1}{5}\pi
	\end{cases}\]
