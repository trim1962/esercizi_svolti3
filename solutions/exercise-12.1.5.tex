	Trovare la retta che passa per $A\coord{7}{6}$ , parallela alla retta che passa per $B\coord{-13}{8}$ e $C\coord{5}{1}$

	Troviamo il coefficiente angolare della retta che passa per $B$ e $C$
	\begin{align*}
		y-8=&(x+13)\\
		1-8=&(5+13)\\
		-7=&18m\\
		m=&-\dfrac{7}{18}
	\end{align*}

		Due rette sono parallele se hanno lo stesso coefficiente angolare quindi \[m_1=m_2 \]
		quindi $m_2=-\dfrac{7}{18}$

		Utilizzando l'equazione del fascio di rette, il valore di $m_2$ trovato e le coordinate di $A$ ottengo:
		\begin{align*}
			y-6=&-\dfrac{7}{18}(x-7)\\
			y=&-\dfrac{7}{18}x+\dfrac{49}{18}+6\\
			y=&-\dfrac{7}{18}x+\dfrac{157}{18}\\
		\end{align*}
		Cioè l'equazione della retta parallela cercata.
			\begin{center}
				\includestandalone[width=.5\textwidth]{terzo/grafici/retta_dis_15}
				%\captionof{figure}{Grafico}\label{fig:EsRiedistanza13}
			\end{center}
