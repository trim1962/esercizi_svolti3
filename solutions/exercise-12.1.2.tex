	Calcoliamo la lunghezza tra $AB$ come con \cref{exa:discinque}
	\begin{align*}
		d(AB)=&\sqrt{(-3+1)^2+(1-4)^2}\\
		=&\sqrt{4+9}\\
		=&\sqrt{13}\\
	\end{align*}
	Calcolo la distanza tra $CB$ come con l'\cref{exa:disuno}
	quindi \[d(CB)=\abs{-1+6}=\abs{5}=5\]
	Ora calcoliamo la distanza tra $CD$ come con \cref{exa:discinque}
	\begin{align*}
		d(CD)=&\sqrt{(-6+8)^2+(4-1)^2}\\
			=&\sqrt{4+9}\\
		=&\sqrt{13}
	\end{align*}
	Calcolo la distanza tra $DA$ come con l'\cref{exa:disuno}
	quindi \[d(DA)=\abs{-3+8}=\abs{5}=5\]
	Conseguentemente il perimetro è
	\[2P=5+\sqrt{13}+5+\sqrt{13}=10+2\sqrt{13}\]
	\begin{center}
		\includestandalone[width=.5\textwidth]{terzo/grafici/retta_dis_12}
		\captionof{figure}{Perimetro}\label{fig:EsRieDistanza12}
	\end{center}
