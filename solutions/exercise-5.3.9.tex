 $\cos (3x+\ang{30;;})=\dfrac{1}{4}$

 Controllare che la calcolatrice sia impostata in gradi\index{Coseno}.

 Basta verificare che
 \testgradi

 In caso contrario modificare le impostazioni.

 Non resta che procedere con il calcolo.

 Le soluzioni sono
 \[\begin{cases}
 x_1=+\alpha+k\ang{360;;}\\
 x_2=-\alpha+k\ang{360;;}\\
 \end{cases}\]
 Calcolo $\alpha$
 \begin{center}
 \begin{tabular}{ll}
 \tastoicos\tasto{\num[round-precision=2,round-mode=places]{0.25}}
 \tastouguale&\num[round-precision=\lungarrotandamento,round-mode=places]{75.52248781}
 \end{tabular}
 \end{center}
 \[\alpha=\SI[round-precision=\lungarrotandamento,round-mode=places]{75.52248781}{\si{\degree}}\]
 \begin{align*}
 3x_1+\ang{30;;}&=\SI[round-precision=\lungarrotandamento,round-mode=places]{75.52248781}{\si{\degree}}+k\ang{360;;}\\
 3x_1&=\SI[round-precision=\lungarrotandamento,round-mode=places]{75.52248781}{\si{\degree}}-\ang{30;;}+k\ang{360;;}\\
 3x_1&=\SI[round-precision=\lungarrotandamento,round-mode=places]{45.52248781}{\si{\degree}}+k\ang{360;;}\\
 x_1&=\SI[round-precision=\lungarrotandamento,round-mode=places]{15.1741626}{\si{\degree}}+k\ang{120;;}\\
 \end{align*}
 \begin{align*}
 3x_2+\ang{30;;}&=-\SI[round-precision=\lungarrotandamento,round-mode=places]{75.52248781}{\si{\degree}}+k\ang{360;;}\\
 3x_2&=-\SI[round-precision=\lungarrotandamento,round-mode=places]{75.52248781}{\si{\degree}}-\ang{30;;}+k\ang{360;;}\\
 3x_1&=\SI[round-precision=\lungarrotandamento,round-mode=places]{-105.5224878}{\si{\degree}}+k\ang{360;;}\\
 x_2&=\SI[round-precision=\lungarrotandamento,round-mode=places]{-35.1741626}{\si{\degree}}+k\ang{180;;}\\
 \end{align*}

 Le soluzioni sono

 \[\begin{cases}
x_1=\SI[round-precision=\lungarrotandamento,round-mode=places]{15.1741626}{\si{\degree}}+k\ang{120;;}\\
x_2=\SI[round-precision=\lungarrotandamento,round-mode=places]{-35.1741626}{\si{\degree}}+k\ang{120;;}\\
 \end{cases}\]
 
