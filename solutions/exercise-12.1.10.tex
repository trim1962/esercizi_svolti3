	Date le rette $y=-2x+5$ e $y=x+1$ trovare,se esiste, il loro punto di intersezione $A$. Trovare l'equazione della parallela a $y=5x+4$ che passa per il punto $A$.
	\begin{align*}
	&\begin{cases}
	y=-2x+5\\
	y=x+1
	\end{cases}
	&&\begin{cases}
	-2x+5=x+1\\
	y=x+1
	\end{cases}\\
	&\begin{cases}
	-3x=-4\\
	y=x+1
	\end{cases}
	&&\begin{cases}
	y=\dfrac{4}{3}\\[.5em]
	y=\dfrac{4}{3}+1
	\end{cases}\\
	&\begin{cases}
y=\dfrac{4}{3}\\[.5em]
y=\dfrac{7}{3}
	\end{cases}
	\end{align*}
	$A\coord{\dfrac{4}{3}}{\dfrac{7}{3}}$
	Due rette sono parallele se hanno lo stesso coefficiente angolare quindi \[m_1=m_2 \]
	quindi $m_2=5$

	Utilizzando l'equazione del fascio di rette, il valore di $m_2$ trovato e le coordinate di $A$ ottengo:
	\begin{align*}
	y-\dfrac{7}{3}=&5(x-\dfrac{4}{3})\\
	y=&5x-\dfrac{20}{3}+\dfrac{7}{3}\\
	y=&5x-\dfrac{13}{3}\\
	\end{align*}
	Cioè l'equazione della retta parallela cercata.
	\begin{center}
		\includestandalone[width=.6\textwidth]{terzo/grafici/retta_dis_20}
		%\captionof{figure}{Grafico}\label{fig:EsRiedistanza13}
	\end{center}

