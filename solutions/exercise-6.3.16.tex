	$\tan (5x-\ang{70;;})=-5$

	Controllare che la calcolatrice sia impostata in gradi\index{Tangente}.

	Basta verificare che
	\testgradi

	In caso contrario modificare le impostazioni.

	Non resta che procedere con il calcolo.

	Le soluzioni sono
	\[x_1=+\alpha+k\ang{180;;}\]
	Calcolo $\alpha$
	\begin{center}
		\begin{tabular}{ll}
			\tastoitan\tasto{\num[round-precision=1,round-mode=places]{5}}
			\tastouguale&\num[round-precision=\lungarrotandamento,round-mode=places]{-78.69006753}
		\end{tabular}
	\end{center}
	\[\alpha=-\SI[round-precision=\lungarrotandamento,round-mode=places]{78.69006753}{\si{\degree}}\]
	\begin{align*}
	5x_1-\ang{70;;}&=-\SI[round-precision=\lungarrotandamento,round-mode=places]{76.69006753}{\si{\degree}}+k\ang{180;;}\\
	5x_1&=-\SI[round-precision=\lungarrotandamento,round-mode=places]{78.69006753}{\si{\degree}}+\ang{70;;}+k\ang{180;;}\\
	5x_1&=-\SI[round-precision=\lungarrotandamento,round-mode=places]{8.690067526}{\si{\degree}}+k\ang{180;;}\\
	x_1&=-\SI[round-precision=\lungarrotandamento,round-mode=places]{1.738013505}{\si{\degree}}+k\ang{36;;}
	\end{align*}

	La soluzione è
	\[x_1=-\SI[round-precision=\lungarrotandamento,round-mode=places]{1.738013505}{\si{\degree}}+k\ang{36;;}\]
