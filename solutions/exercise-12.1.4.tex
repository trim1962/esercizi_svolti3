Data la retta $r$ di equazione $y=7x+6$, trovarne le intersezioni con gli assi. Indicato con $A$ il punto di intersezione di $r$ con l'asse $x$ e con $B$ il punto di intersezione con $r$ l'asse $y$. Trovare la retta $s$ perpendicolare a $r$ che passa $A$.
Trovare la retta $t$ parallela a $s$ che passa per $B$. Disegnare la rette.

Intersezione con l'asse $x$ di $y=7x+6$
\begin{align*}
7x+6=&0\\
7x=&-6\\
x=&-\dfrac{6}{7}
\end{align*}
 $A\coord{-\dfrac{6}{7}}{0}$

 Intersezione con l'asse $y$
  $A\coord{0}{6}$

 	$m=7$ utilizzando la formula \[m_1\cdot m_2=-1\] ottengo
 	\begin{align*}
 	7\cdot m_2=&-1\\
 	m_2=&-\dfrac{1}{7}\\
 \end{align*}

 	Utilizzando l'equazione del fascio di rette, il valore di $m_2$ trovato e le coordinate di $A$ ottengo:
 	\begin{align*}
 	y=&-\dfrac{1}{7}(x+\dfrac{6}{7})\\
 	y=&-\dfrac{1}{7}x-\dfrac{6}{49}
 	\end{align*}
 	Cioè l'equazione della retta perpendicolare cercata.

	Due rette sono parallele se hanno lo stesso coefficiente angolare quindi \[m_1=m_2 \]
	quindi $m_2=-\dfrac{1}{7}$

	Utilizzando l'equazione del fascio di rette, il valore di $m_2$ trovato e le coordinate di $B$ ottengo:
	\begin{align*}
	y-6=&-\dfrac{1}{7}x\\
	y=&-\dfrac{1}{7}x+6\\
	\end{align*}
	Cioè l'equazione della retta parallela cercata.
		\begin{center}
			\includestandalone[width=.5\textwidth]{terzo/grafici/retta_dis_14}
			%\captionof{figure}{Grafico}\label{fig:EsRiedistanza13}
		\end{center}
