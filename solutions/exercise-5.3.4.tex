 $\cos 2x=\num[round-precision=3,round-mode=places]{0.128}$

 Controllare che la calcolatrice sia impostata in radianti\index{Radianti}\index{Coseno}.

 Basta verificare che
 \testradianti

 In caso contrario modificare le impostazioni.

 Non resta che procedere con il calcolo.

 Le soluzioni sono
 \[\begin{cases}
 x_1=+\alpha+2k\pi\\
 x_2=-\alpha+2k\pi\\
 \end{cases}\]
 Calcolo $\alpha$

 \begin{center}
 \begin{tabular}{ll}
 \tastoicos\tasto{\num[round-precision=3,round-mode=places]{0.128}}
 \tastouguale&\num[round-precision=\lungarrotandamento,round-mode=places]{1.442444199}
 \end{tabular}
 \end{center}
 \[\alpha=\SI[round-precision=\lungarrotandamento,round-mode=places]{1.442444199}{\radian} +2k\pi\]
 \begin{align*}
 2x_1&=\SI[round-precision=\lungarrotandamento,round-mode=places]{1.442444199}+2k\pi \si{\radian}\\
 x_1&=\SI[round-precision=\lungarrotandamento,round-mode=places]{0.721222099}+k\pi \si{\radian}\\
 \end{align*}
 Le soluzioni sono

 \[\begin{cases}
 x_1&=+\SI[round-precision=\lungarrotandamento,round-mode=places]{0.721222099}+k\pi \si{\radian}\\\

 x_2&=-\SI[round-precision=\lungarrotandamento,round-mode=places]{0.721222099}+k\pi \si{\radian}\\
 \end{cases}\]
 
